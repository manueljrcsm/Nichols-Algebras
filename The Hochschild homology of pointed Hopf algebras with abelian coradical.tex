\documentclass{amsart}
\usepackage{amsaddr}
\usepackage{amsthm}
\usepackage{amsmath}
\usepackage{amssymb}
\usepackage{tikz-cd}
\usepackage{tikz}
\usetikzlibrary{calc}



\usepackage[square,numbers]{natbib}
\bibliographystyle{abbrvnat}

% ---- Environenments ---
\newtheorem{theorem}{Theorem}[section]
\newtheorem{lemma}[theorem]{Lemma}
\newtheorem{corollary}[theorem]{Corollary}
\newtheorem*{conjecture}{Conjecture}

\theoremstyle{definition}

\newtheorem{definition}[theorem]{Definition}
\newtheorem{question}[theorem]{Question}
\newtheorem*{question*}{Question}
\newtheorem*{remark}{Remark}
\newtheorem*{convention}{Convention}
\newtheorem{example}[theorem]{Example}

% --- Text stylings

\newcommand{\emphaut}[1]{\textsc{#1}} % use this to emphasise names/authors.

% --- Mathematical Symbols 
\newcommand{\from}{\colon}
\newcommand{\defeq}{{\colon =}}
\newcommand{\eqdef}{{= \colon}}

\newcommand{\YD}[1]{\ensuremath{{}^{#1}_{#1}\mathcal{YD}}}
\newcommand{\Nichols}[1]{\ensuremath{\mathcal{B}(#1)}}
\DeclareMathOperator{\Hom}{Hom}
\DeclareMathOperator{\Gr}{Gr}
\DeclareMathOperator{\gr}{gr}
\DeclareMathOperator{\id}{id}



\author{}
\address[]{
	Technische Universität Dresden,
	Institut f\"ur Geometrie,
	\\
	Zellescher Weg 12-14, 01062 Dresden}
\email{}
\date{\today}
\subjclass[2010]{}
\keywords{}	
\title{A BGG type resolution for pointed Hopf algebras with abelian coradical.}
\begin{document}
	\begin{abstract}
		We define a projective resolution for a certain type of Hopf algebras and use it to study some homological properties.
	\end{abstract}
	\maketitle

\section{Introduction}
	The Bernstein-Gelfand-Gelfand resolution of the universal enveloping of a (complex) Lie algebra $\mathfrak g$ was introduced in TODO:REFERENCE in order to TODO:REFERENCE. 
	It principal ingredients are a triangular decomposition of $\mathfrak g \eqdef \mathfrak n^- \oplus \mathfrak h \oplus \mathfrak n^+$ and the action of the Weyl group on the weight space of $\mathfrak g$.
	Later it was realised that for quantised universal enveloping algebras at a non-root of unity an almost identical resolution can be given.
	
	\subsection{Motivation and main results}
	
	In this paper we intend to generalise this procedure further to the setting of pointed Hopf algebras with abelian coradical. 
	
	\subsection{Outline}
	
\section{Pointed Hopf algebras with abelian coradical}
	We work over an algebraically closed field $k$ of characteristic zero; `$\dim$' and `$\otimes$' ought to be understood as dimension and tensor product over $k$. 
	Standard notation for Hopf algebras, as in e.g.\ \cite{Montgomery1993,Radford2012}, is freely used. 
	Given a Hopf algebra $H$ we write $\Gr(H)$ for its group of group-likes, $\Pr(H)$ for its space of primitive elements and $H^\circ$ for its (finite) dual Hopf algebra. 
	The antipode of $H$ is denoted by   $S\from H \to H$, its counit by $\epsilon \from H \to k$ and its coproduct by $\Delta \from H \to H\otimes H$. 
	For calculations involving the coproduct of $H$ or the coaction of some $H$ comodule $M$ we rely on reduced Sweedler notation. 
	For example we  write $h_{(1)} \otimes h_{(2)} \defeq \Delta (h)$ for $h \in H$.
	An element $x\in H$ whose coproduct is $\Delta(x) = 1 \otimes x + x \otimes g$, with $g\in \Gr(H)$ is called a twisted primitive.
	\\
	
	The \emph{Cartier-Kostant-Milnor-Moore theorem}, see
	%TODO CHECK SOURCE
	\cite{Sweedler1969, Montgomery1993}, states, that any cocommutative Hopf algebra $H$ over an algebraically closed field of characteristic zero is isomorphic to the smash product $U(P)\#kG$ of the universal enveloping algebra of its Lie algebra of primitive elements with the group algebra of its group-like elements.
	This has some remarkable consequences like the existence of a $PBW$-basis. 
	%TODO Paraphrase the result. Something like: By the artice the Hochschild cohomology ring of a cocommutative Hopf algebra $H$ is isomorphic to the cohomology ring of $U(P)$.
	The Hochschild cohomology ring of such smash products was study in \cite{Burciu2007}.
	
	In this paper we want to consider a generalisation of the previous class of Hopf algebras and study their Hochschild (co-)homology.
	
	\begin{definition}[\cite{Radford2012}] \label{def: Pointed} %TODO: PRECISE SOURCE
		A Hopf algebra $H$ is called \emph{pointed} if every simple comodule is one-dimensional.
	\end{definition}

	The connection between (finite-dimensional) pointed Hopf algebras and cocommutative Hopf algebras was establised by 
	\emphaut{Andruskiewitsch} and \emphaut{Schneider} in conjectural form.
	\begin{conjecture}
		Every finite-dimensional pointed Hopf algebra is generated by group-likes and twisted primitives.
	\end{conjecture}
	%TODO: CHECK WHETEHER THIS HOLDS  
	As a consequence of \cite{Angiono2019} and the classification of Nichols algebras of diagonal type \cite{Heckenberger2009, Andruskiewitsch2017} the conjecture holds in case the group of group-like elements is abelian.
	
	An important tool for studying pointed Hopf algebras is the coradical filtration. It is defined in analogy with the Jacobson radical of a ring.
	
	%TODO: IS THIS DEF IN LINE WITH THE WAY WE DEFINE BOSONISATION?
	\begin{definition}[\cite{Radford2012}]\label{def: Coradical} %TODO: PRECISE SOURCE
		Let $C$ be a coalgebra. The \emph{coradical} of $C$ is
		\begin{align}\label{eq: Coradical}
			C_0 \defeq \sum D, \text{ with } D\subset C \text{ simple comodule}.
		\end{align}
		
		The \emph{coradical filtration} of $C$ is the ascending filtration
		$C_0 \subseteq C_1 \subseteq C_2 \subseteq \dotsc$ whose $n {+} 1$-th term is inductively defined by
		\begin{align*}
			C_{n+1} \defeq \{ c\in C \mid \Delta(c) \in C_n \otimes C + C \otimes C_0 \}
		\end{align*} 
		
		We say $C$ is \emph{coradically graded} if  as vector spaces
		\begin{align*}
		 	C= \sum _{n \geq 0} C(n) \text{ with } \sum _{n = 0}^m C(n)= C_m
		\end{align*}
	\end{definition}

	We note two important observations about the coradial filtration without a proof.
	
	%TODO: BETTER SOURCE
	\begin{lemma}[{\cite[Definition 1.13]{Andruskiewitsch2002}}] \label{lem: CoradFiltExhaustive}
		The coradical filtration of a coalgebra $C$ is an exhaustive filtration. That is, $C= \cup_{n \geq 0} C_n$.
	\end{lemma}

	\begin{definition}
	The coradically graded coalgebra associated to a coalgebra $C$ is  
		\begin{align*}
			 \gr C = \oplus_{n \geq 0} C_{n} \setminus C_{n-1} \text{ with } C_{-1} \defeq \{0\}.
		\end{align*}
	\end{definition}

	In case $H$ is a pointed Hopf algebra its coradical $H_0$ is a sub-Hopf algebra and the coradical filtration is a coalgebra as well as an algebra filtration. This equips $\gr H$ with the structure of a graded Hopf algebra containing $H_0$ as a sub-Hopf algebra in degree zero.
	
	Similarly to the aforementioned Cartier-Kostant-Milnor-Moore theorem 
	we want to write the coradically graded Hopf algebra $\gr H$ associated to a pointed Hopf algebra $H$ as a (version of) a smash product. 
	\begin{lemma}
		Let $H$ be a pointed Hopf algebra. The inclusion $\iota \from H_0 \hookrightarrow \gr H$ has a retraction $\pi \from \gr H \twoheadrightarrow H_0$.
	\end{lemma}
	
	\begin{lemma}
		Let $H$ be a pointed Hopf algebra.
		The vector space 
		$R \defeq \{ h \in \gr H \mid (\id \otimes \pi)\Delta(h) = h\otimes 1 \}$ can be equipped with the structure of a braided Hopf algebra in $\YD{H_0}$.
	\end{lemma}
	
	
	
	
	\subsection{Nichols algebras}
	

\section{Weyl groupoids and their geometry}

\section{The Hochschild homology of pointed Hopf algebras with abelian coradical}



-------TODO: READ articles-------
\begin{enumerate}
	\item Hochschild and cyclic homology: 
		\cite{Andruskiewitsch2020}, 
		\cite{Burciu2007},
		\cite{Erdmann2019}, 
		\cite{Heckenberger2007}, 
		\cite{Kowalzig2014}, 
		\cite{Loday1998},
		\cite{Mastnak2009, Mastnak2010},
		\cite{Witherspoon2019}
	\item (Pointed) Hopf algebras: 
		\cite{Andruskiewitsch2002, Andruskiewitsch2010, Andruskiewitsch2017},
		\cite{Angiono2013},
		\cite{Angiono2019}
		\cite{Laugwitz2016}
		\cite{Montgomery1993}
		\cite{Pogorelsky2016}
		\cite{Radford2012}
		\cite{Vay2019}
	\item Quantum groups: 
		\cite{Joseph1995}, 
		\cite{Kassel1998} 
	\item Kac-Moody algebras: 
		\cite{Kac1990}, 
		\cite{Kumar2002}, 
	\item Weyl group(oids): 
		\cite{Cuntz2015, Cuntz2019}
		\cite{Angiono2018}, 
		\cite{Heckenberger2009, Heckenberger2010, Heckenberger2011} 
	\item Tensor categories: 
		\cite{Etingof2015} 
\end{enumerate}
%TODO: REMOVE NOCITE!
\nocite{*}

	
%	\section{Preliminaries - Nichols algebras of Cartan type}
%	
%	In this section main definitions of Nichols algebras of diagonal/Cartan type are recalled according to the Andruskiewitsch survey. 
%	
%	\begin{convention} $ $ \\
%		\begin{enumerate}
%			\item $k$ denotes an algebraically closed field of characteristic zero
%			\item $G$ denotes a finite abelian group (finitely generated should be fine too!)
%			\item $\YD G$ denotes the category of Yetter-Drinfeld modules over the group algebra of $G$.
%			\item For a natural number $n\in \mathbb N$ we denote $[n]:= \{1,\dotsc, n\}$.
%		\end{enumerate}
%	\end{convention}
%	
%	\subsection{Nichols algebras}
%	
%		In short, Nichols algebras are braided Hopf algebras which can be seen as generalisations of exterior or symmetric algebras. Even though it is possible to define them abstractly, one usually considers them as Hopf algebra objects in the category of Yetter-Drinfeld modules over some Hopf algebra $H$.
%	\begin{lemma}
%		Let $H$ be some group. The category $\YD{H}$ of $H$-Yetter-Drinfeld modules is, as a braided monoidal category, equivalent (in fact, isomorphic) to the category of $H$-graded $H$-modules. That is the category $\mathcal C$ whose
%		\begin{itemize}
%			\item  objects are $H$-graded $H$-modules $V:= \oplus_{h\in H} V_h$ such that
%			$n \triangleright V_h \subset V_{nhn^{-1}}$,
%			\item morphisms are $H$-linear maps $f: \oplus_{h\in H} V_h \rightarrow \oplus{h\in H} W_h$, which respect the grading,
%			\item tensor product is defined via
%			$$
%			\oplus_{h\in H} V_h \bigotimes \oplus_{n \in H} W_n := \oplus_{h \in H}(\oplus_{n \in H} V_{hn^{-1}} \otimes W_n).
%			$$
%			\item braiding is given by
%			$$
%			\sigma: \oplus_{h\in H} V_h \bigotimes \oplus_{n \in H} W_n \rightarrow 
%			\oplus_{n \in H} W_n \bigotimes \oplus_{h\in H} V_h, \qquad  v\otimes  w \mapsto \left(h \triangleright w\right) \otimes v \text{ for } v\in V_h.
%			$$	
%		\end{itemize}
%	\end{lemma}
%%	\begin{proof}
%%		A direct calculation shows that every $H$-graded $H$-module is a Yetter-Drinfeld module. This yields a functor of braided monoidal categories.
%%		To prove the converse, consider any Yetter-Drinfeld module $M$ over $H$. 
%%		Write $M_h :=\{m \in M \mid \delta(m) = h \otimes m\}$. One shows $M \cong \oplus M_h$ as comodules and modules and the Yetter-Drinfeld condition implies 
%%		$n \triangleright M_h \subset M_{nhn^{-1}}$.
%%		This defines another functor of braided monoidal categories. 
%%		It is proven directly that these functors are mutually inverse to each other.
%%	\end{proof}
%	
%	There exist various definitions of Nichols algebras. The most conceptually compelling one is given in terms of  their universal property.
%	
%	\begin{definition}
%		Let $V \in \YD{H}$ be some Yetter-Drinfeld module over a Hopf algebra $H$.
%		The Nichols algebra of $V$ is the braided graded Hopf algebra $\Nichols V= \oplus_{n \in \mathbb N_0} \Nichols V_n\in \YD{H}$ uniquely determined by
%		\begin{enumerate}
%			\item It is a connected algebra, i.e. $\Nichols V_0 = \{0\}$ and $\Nichols V_1 = V$.
%			\item The set  $Pr(\Nichols V)$ of primitives is given by $V$. 
%			\item It is as an algebra generated in degree one, i.e. it is generated by $V$. 
%		\end{enumerate}
%		The dimension of $V$ is called the rank of the Nichols algebra $\Nichols V$.
%	\end{definition}
%	
%	There is a more hands-on description, which realises the Nichols algebra as a quotient of the tensor algebra of $V$. 
%	
%	\begin{lemma}
%		Let $V \in \YD{H}$ be some Yetter-Drinfeld module over a Hopf algebra $H$.
%		The Nichols algebra $\Nichols V$ of $V$ is, up to unique isomorphism, given by the quotient $T(V)/J$ of the tensor algebra $T(V)$ of $V$ by the maximal two-sided ideal $J$ which satisfies
%		\begin{enumerate}
%			\item $J$ is a Yetter-Drinfeld submodule of $T(V)$,
%			\item $J$ is a two-sided coideal,
%			\item $J$ is generated by homogenous elements of degree $\geq 2$.
%		\end{enumerate}
%	\end{lemma}
%	
%	
%		\begin{theorem}[Bosonisation]
%		Let $H$ be a Hopf algebra with invertible antipode and $R$ a braided Hopf algebra in $^H_H \mathcal{YD}$. 
%		The bosonisation of $R$ by $H$ is the Hopf algebra $R\#H$ whose underlying vector space is $R \otimes H$ and whose multiplication, comultiplication and antipode are defined by
%		\begin{align*}
%		\begin{aligned}
%		& (r\#g)(s\#h)=r(g_{(1)}\triangleright s) \# g_{(2)}h
%		\\
%		& \Delta(r\#g):= r^{(1)}\#r^{(2)}_{(-1)}g_{(1)} \otimes r^{(2)}_{(0)}\# g_{(2)}
%		\\
%		& S(r\# g) = S_H(r_{(-1)(2)} g_{(2)})\triangleright S_R(r_{(0)})\#S_H(r_{(-1)(1)} g_{(1)}).
%		\end{aligned}
%		\end{align*}		
%	\end{theorem}
%	
%	A fact which will be of utter importance to us is that there is a section of Hopf algebras 
%	$H\overset{\iota}{\underset{\pi}{\rightleftarrows}} R\#H$ given by
%	$$
%	\iota:H \rightarrow R\#H, \quad h \mapsto 1\#h, \qquad \pi: R\#H\rightarrow H, \quad r\#h \mapsto \varepsilon(r)h.
%	$$
%	
%	Commonly, Nichols algebras are distinguished in terms of the braiding.
%	
%	\begin{definition}
%		Let $V$ be a finite dimensional vector space and $c: V\otimes V \rightarrow V \otimes V$ be a solution of the YBE. We say that $(V,c)$
%		\begin{enumerate}
%			\item has \emph{group type} if there exists a basis $v_1, \dotsc, v_\theta$ of $V$ and elements $g_i(v_j)\in V$ such that
%			$
%			c(v_i \otimes v_j) = g_i(v_j) \otimes v_i.
%			$
%			Necessarily $g_i \in GL(V)$.
%			\item has \emph{finite} (resp. \emph{abelian}) group type if it is of group type and the subgroup of $GL(V)$ spanned by $g_1,\dotsc g_\theta$ is finite (resp. abelian).
%			\item is of \emph{diagonal} type if there there exists a basis $v_1, \dotsc, v_\theta$ of $V$ and scalars $q_{ij} \in k$ such that
%			$
%			c(v_i \otimes v_j) = q_{ij} v_j \otimes v_i.
%			$
%			The matrix $(q_{ij})$ is called the matrix of the braiding.
%		\end{enumerate}
%	\end{definition}
%
%	\subsection{Nichols algebras of diagonal type}
%	
%	\begin{lemma}
%		Let $(V,c)$ be of group type and $v_1,\dotsc v_\theta$ a basis such that there are $g_1,\dotsc, g_\theta \in GL(V)$ satisfying $c(v_i\otimes v_j) = g_i(v_j) \otimes v_i$. 
%		Then $V$ becomes a Yetter-Drinfeld module over the subgroup of $GL(V)$ generated by $\{ g_1, \dotsc, g_\theta \}$ 
%		by defining
%		$$
%		g_i \triangleright v_j := g_i(v_j), \qquad \delta(v_i)= g_i \otimes v_i.
%		$$ 
%		
%		If $(V,c)$ is of diagonal type it is necessarily of abelian group type.
%		It is also of finite group type if every entry of the matrix of the braiding is a root of unity.
%	\end{lemma}
%	
%	\begin{example}
%		Let $N\geq 2 $ and $C_N$ be the cyclic group of order $N$. Write $g$ for a generator of $C_N$.
%		Fix some primitive $N$-th root of unity $q\in k$. 
%		Then there exists a unique character $\xi: C_N \rightarrow k$, $g \mapsto  q$.
%		We call integers $a_1, a_2, b_1, b_2 \in \mathbb Z_N$ satisfying 
%		$$
%		a_1 b_1 \neq 0 \quad a_2 b_2 \neq 0 , \quad a_1 b_2 + a_2 b_1 = 0 \mod N
%		$$
%		\emph{parameters of a generalised Taft algebra}.
%		The vector space $V:=\langle v_1, v_2\rangle$ becomes a Yetter-Drinfeld module over $C_N$ by defining 
%		$$
%		g\triangleright v_i := q^{b_i} v_i,\qquad \delta(v_i) = g^{a_i} \otimes v_i.
%		$$
%		Its Nichols algebra $\mathcal B(V)$ is generated, as an algebra, in degree one by the primitive elements $v_1$ and $v_2$, together with the relations
%		$$
%		v_i^{N_i} = 0, \qquad v_1 v_2 = q^{a_1b_2} v_2 v_1,
%		$$ 
%		where $N_i:=|\langle a_i b_i \rangle |$ is the order of the subgroup of $\mathbb Z_N$ generated by $a_i b_i$.
%
%		The bosonisation of the above described Nichols algebra is (the cooposite) of a generalised Taft algebra, i.e. a Hopf algebra $H_q(a_1, a_2, b_1, b_2)$ parametrised by some primitive $N$-th roof of unity $q$  and parameters of a generalised Taft algebra $a_1, a_2, b_1, b_2$.
%		More explicitly, it is generated by three generators $g$, $x$ and $y$ and relations
%		\begin{gather}
%		\begin{gathered} \label{eq:AlgRelations}
%		g^N = 1, \qquad
%		x^{N_x} = 0, \qquad
%		y^{N_y} = 0,	
%		\\
%		gx = q^{b_1}xg, \qquad
%		gy = q^{b_2}yg, \qquad
%		xy = q^{a_1b_2} yx,
%		\end{gathered}
%		\\
%		\begin{gathered}\label{eq:CoalgRelations}
%		\Delta(x) = x \otimes 1 + g^{a_1} \otimes x,  \qquad
%		\Delta(y) = y \otimes 1 + g^{a_2} \otimes y,
%		\\
%		\Delta(g) = g \otimes g , \;\;
%		\end{gathered}
%		\\
%		\begin{gathered}\label{eq:AntipodeRelations}
%		S(g) = g^{-1}, \qquad
%		S(x) = - q^{a_1b_2}xg^{-a_1}, \qquad
%		S(y) = - q^{a_2b_2}yg^{-a_2}.
%		\end{gathered}
%		\end{gather}		
%	\end{example}
%
%
%
%	\begin{definition}
%		Let $(V,q)$ be a braided vector space of diagonal type. That is there exists a fixed ordered basis
%		$v_1,\dotsc v_n$ s of $V$ such that the braiding of $V$ is implemented by the $n \times n$-matrix $q$. The \emph{bilinear form associated to the braiding} is defined as the map 
%		\begin{align}
%		\begin{aligned}
%		&\mathfrak{q}: \mathbb Z^{[n]} \times \mathbb Z^{[n]} \rightarrow k^\times \\
%		&\mathfrak{q}(\alpha, \beta)= \prod_{i,j=1}^{n}q_{ij}^{\alpha_i\beta_j}, \quad \text{ where } \alpha= (\alpha_1,\dotsc,\alpha_n), \beta = (\beta_1, \dotsc,\beta_n)
%		\end{aligned}
%		\end{align} 	
%	\end{definition}
%
%	\begin{definition}
%		Let $(V,q)$ be a braided vector space of diagonal type with dimension $\dim V =n$ and $R$ a quotient of $T(V)$ by a $\mathbb Z^{[n]}$ homogeneous ideal.
%		Then the braiding $c: R\otimes R \rightarrow R \otimes R$ satisfies 
%		\begin{align} 
%			c(x\otimes y) = \mathfrak{q}(\alpha,\beta) y \otimes x,\qquad  \text{ for } x\in R_\alpha, y \in R_\beta \text{ homgenous elements of degree $\alpha$ and $\beta$}.
% 		\end{align}	
%	\end{definition}
%
%	\begin{definition}
%		Let $(R,c)$ be any braided Hopf algebra. The adjoint action of $R$ on itself is defined by
%		\begin{align}
%			ad: R\rightarrow GL(R),  ad(x)(y):= 	\mu(id \otimes \mu)(id \otimes id \otimes S)(id \otimes c)(\Delta \otimes id)(x \otimes y)
%		\end{align}
%		The braided commutator is defined as the map
%		\begin{align}
%			[-,-]_c: R \otimes R \rightarrow R, [x,y]_c := \mu (id-c)(x\otimes y)
%		\end{align}
%	\end{definition}
%
%	We fix the ordered standard basis $e_1, \dotsc e_n \in\mathbb Z^n \subset k^n$.
%
%	\begin{lemma}
%		If $(V,q)$ is a braided vector space of diagonal type with dimension $\dim V =n$ and $R$ a quotient of $T(V)$ by a $\mathbb Z^{n}$ homogeneous ideal which is also a coideal (i.e. $R$ is a braided Hopf algebra then:
%		\begin{enumerate}
%			\item $[v_i,x]_c = v_ix - \mathfrak{q}(e_i,\alpha)x v_i \text{ for $v_i \in V$ and $x_\alpha \in R_\alpha$}$
%			\item $[v_i,y]_c = ad(x)(y)$.
%		\end{enumerate}
%	\end{lemma}
%
%	\begin{lemma}
%		If $(V,q)$ is a braided vector space of diagonal type with dimension $\dim V =n$ and $R$ a quotient of $T(V)$ by a $\mathbb Z^{n}$ homogeneous ideal then the braided commutator satisfies:
%		\begin{align}
%			[u,vw]_c & = [u,v]_c w + \mathfrak{q}(\alpha, \beta) v[u,w] \\	
%			[uv,w]_c & = u[v,w]_c + \mathfrak{q}(\beta, \gamma) [u,w]v \\
%			[u,[v,w]]_c & = [[u,v]_c , w]_c  - \mathfrak q (\beta, \gamma) [u, w]_c w + \mathfrak q (\alpha, \beta) v [u, w]_c\\ 
%		\end{align}
%	\end{lemma}
%
%	\subsection{Nichols algebras of Cartan type}
%	
%	We fix a braided vector space $(V,q)$ of dimension $n$.
%	
%	\begin{definition}
%		Let $(V,q)$ be a braided vector space, $\Nichols V$ its Nichols algebra and $B$ a generating set of the  PBW basis of $\Nichols V$.
%		We define the set of positive roots as
%		\begin{align}
%			\Delta^+_q := \{deg(b) \mid b \in B \} \subset k^n,
%		\end{align}
%		where $deg$ denotes the degree with respect to the $\mathbb Z^{[n]}$ braiding.
%		The Nichols algebra $\Nichols B$ is called arithmetic if $|\Delta^+_q|$ is finite.
%	\end{definition}
%
%	\begin{remark}
%		If $\Nichols B$ is arithmetic every root has multiplicity one, i.e.
%		there is a bijection $\Delta^+_q \rightarrow B$, where $B$ is a generating set of the PBW basis of $\Nichols B$.
%	\end{remark}
%	
%	\begin{definition}
%		A generalised Cartan matrix is a matrix $A= (a_{ij})_{1\leq i,j \leq n}$ such that
%		\begin{gather}
%			a_{ii}=2, \qquad, a_{ij}\leq 0 \text{ for } i \neq j, \qquad a_{ij}=0 \Leftrightarrow a_{ji}=0.			
%		\end{gather}
%		(See Kumar)
%	\end{definition}
%	
%	\begin{definition}
%		We call $(V,q)$ of \emph{Cartan type}, if there exists a generalised Cartan matrix $A=(a_{ij})_{1\leq i,j \leq n}$ such that
%		\begin{align}
%			q_{ij}q_{ji}= q_{ii}^{q_{ij}} \qquad \text{ for } 1\leq i,j\leq n.
%		\end{align}
%	\end{definition}
%	 If a generalised Cartan matrix implements the braiding we choose it such that 
%	 $a_{ij} \neq ord(q_{ii})$ and $a_{ij}$ maximal.
%
%	\begin{lemma}
%		Let $(V,q)$ ne of Cartan type with generalised Cartan matrix $A$ associated to the braiding.
%		Then $A$ induces a set of reflection on $\subset k^n$ via 
%		\begin{align}
%			s_i(\delta_j)= e_j - a_{ij}e_i
%		\end{align}
%		The group generated by $\{s_i \mid 1 \leq i \leq n\}$ is called the Weyl group $W$ associated to $A$.
%		
%		If $A$ is symmetrizable and indecomposable and finite the associated Weyl group is finite.
%	\end{lemma}
%
%	We will concentrate on the case of Nichols algebras of Cartan type whose associated Cartan matrix is symmetrizable(, indecomposable) and finite.
%	
%	\begin{definition}
%		Given a positive root system $\Delta^+$  a convex order is defined to be a total order $<$ on $\Delta^+$ such that, if
%		$a<b$ and $a+b \in \Delta^+$ we have $a< a+b< b$.
%	\end{definition}
%	
%	\begin{lemma}
%		Let $(V,q)$ be a braided vector space of Cartan type sand let $\Delta^+$ denote the set of positive roots associated to $\Nichols V$. Assume that $\Delta^+$ is finite (i.e.) that $\Nichols B$ is arithmetic then $\Nichols B$ is generated by $v_1, \dotsc v_n$ and the relations
%		\begin{align}
%			[v_i,v_j]c &= \sum_{n_{i+1},\dotsc, n_{j-1} \in \mathbb N_+} c_{n_{i+1},\dotsc, n_{j-1}} v_{j-1}^{n_{j-1}} \dotsc v_{i+1}^{n_{i+1}}, \quad v_i < v_j \in \Delta^+, \\
%			v_i^{N_\beta} &= 0,
%		\end{align}
%		where $c_{n_{i+1},\dotsc, n_{j-1}} =0 $ if $n_{i+1}v_{i+1} + \dotsc n_{j-1} v_{j-1} \neq v_i + v_j$.
%	\end{lemma}
%
%	\paragraph{Outline} To obtain a projective resolution of a Nichols algebra of diagonal type $\Nichols(V)$ we proceed as follows:
%	\begin{enumerate}
%		\item We start with a braided vector space $(V,q)$ od dimension $n$
%		\item Associate to it the pre-Nichols algebra $\Nichols{V}'$ (same relations except unrestricted heights of the PBW- generators)
%		\item The PBW-generators (viewed as vectors in $\mathbb R^n$) form a root system $\Delta^+$
%		For a positive root $\mu$ we write $x_\mu$ for the corresponding generator of the $PBW$-basis.
%		\item We denote by $(W,S)$ the Weyl-group generated by the (reflections along) simple positive roots $S \subset \Delta^+$ (i.e. the generators of the Nichols algebra).
%		\item Every element $w\in W$ has a well-defined length $l(w)$ - its minimal representation in terms of letters in $S$. We obtain a decomposition of $W$ in terms of the length function, i.e. 
%		$W = \cup_{k=0}^m W_k$ with $W_k:= \{ w \in W \mid l(w)=k\}$.
%		\item We claim that we obtain a projective resolution of the trivial $\Nichols V'$ module via 
%		\begin{center}
%			\begin{tikzcd}
%				0 
%				\arrow{r}& 
%				P_m 
%				\arrow{r}{d_m}&
%				P_{m-1}
%				\arrow{r}{d_{m-1}} &
%				\dotsc
%				\arrow{r}{d_1} &
%				P_0 
%				\arrow{r} & 
%				k \arrow {r} & 
%				0
%			\end{tikzcd}
%		\end{center}
%		where  $P_j = \oplus_{w\in W_j }\Nichols V'$ and $d_j= \sum_{o \in W_j, p\in W_{j-1}} \iota_p d_k^{(o,p)}\rho_o$ where $\iota_p$ denotes the identification of $\Nichols V'$ wth the $p-th$ copy of itself inside of $P_{j-1}$ and $\rho_o$ the projection of $P_j$ onto its $o$-th component identified with $\Nichols V'$ and
%		$$
%			d_j^{(o,p)}: \Nichols V' \rightarrow \Nichols V', d_j^{(o,p)} = 
%			\left\{ \begin{aligned}
%				&\cdot \xi x_\mu^a  	& \text{with } \xi \in \mathbb C  \text{  and } \mu \in \Delta^+ \text{s.t. } o \tau_\mu =p \\
%				&0 				&\text{else}
%			\end{aligned} \right.
%		$$
%	\end{enumerate}
%	Sketch of the proof:
%	\begin{enumerate}
%		\item We consider the Bruhat order on $(W,S)$.
%			An arrow wrt to the bruhat order is a  pair $(o,p)\in W$ such that 
%			$l(o)-1 = l(p)$ and there exists a transposition $t\in T\subset W$ such that $ot=p$.
%			(Note that these arrows are in 1-to-1 correspondence with the non-zero $d^{(o,p)}_j$ maps.) 
%		\item To obtain the integers $a \in \mathbb N$ we construct from the Bruhat order the Bruhat polytope.
%		\item The Bruhat polytope $\mathcal P$ consists of a set of vertices $\{v_w \mid w \in W\}\subset \mathbb R^n$ and a set $E$ of (directed) straight lines such that there is an straight line $s$ between $v_o,v_p \in V$ if and only if there is an arrow $\tau_\mu$ wrt the Bruhat order between $o,p$ and
%		$s$ is (up to translation) an itegral multiple of the root $\mu$ implementing the arrow between $o$ and $p$.
%		\item Conjecture: The Bruhat polytope (as the minimal polytope satisfying this property) is uniquely defined by these building rules and does always exist.
%		\item Given an arrow $\tau_\mu: o \rightarrow p$ in the Bruhat order we associate to it the integer $a$ which is determined as the unique integer which implements the line $a\mu$ between $o$ and $p$ in the Bruhat polytope.		
%		\item Conjecture: Given two elements $u,w\in W$ such that $l(u)+2= l(w)$ there exist exactly $2$ or $0$ pairs of arrows	$w \rightarrow v \rightarrow u$. If two such arrows exist we call this a square in the Bruhat order.
%		\item Conjecture: Given a square $(u,v,v',w)$ in the Bruhat the corresponding (closed) polygon chain in the Bruhat polytope is contained in a (affine) $2d-$ plane.
%		\item Conjecture There exist scalars such that the maps in the complex corresponding to a square in the Bruhat order are zero (i.e. the complex is a chain complex.)
%	\end{enumerate}
%
%
%	
%	
%	\section{A projective resolution for generalised Taft algebras}
%	
%	In this section the construction of a projective resolution of generalised Taft algebras is reviewed.
%	It is done as conceptually as possible in order to guide towards a more general principle.
%	
%	\subsection{Projective resolutions of Nichols algebras}
%	
%	In this section we discuss a projective resolution for the above mentioned Nichols algebras $\Nichols V$ and `lift' this to a projective resolution of the bosonisation.
%	
%	Let us start by defining the algebra $X$ which is generated by $v_1$ and $v_2$  and the relation
%	$$
%	v_1 v_2 = p v_2 v_1, \qquad \text{where } p:= q^{a_1 b_2}.
%	$$
%	Note that $X$ sits in the following exact sequence $T(V) \twoheadrightarrow X\twoheadrightarrow \Nichols V$ of graded vector spaces.
%	
%	We define a projective resolution of $k$ as an $X_1$-module via
%	\begin{center}
%		\begin{tikzcd}
%		0
%		\arrow[r]
%		& X_1
%		\arrow[r, "f_2"]
%		&
%		X_1^2
%		\arrow[r, "f_1"]
%		& 
%		X_1 \arrow[r, "\varepsilon"]
%		&
%		k \arrow[r]
%		& 
%		0,
%		\end{tikzcd}
%	\end{center}
%	with the morphisms
%	\begin{align*}
%	f_2: & X_1 \rightarrow X_1^2,  && x \mapsto
%	x\begin{pmatrix}
%	p v_2 \\
%	- v_1
%	\end{pmatrix}
%	\\
%	f_1: & X_1^2\rightarrow X_1,  && \begin{pmatrix} x \\ y\end{pmatrix} \mapsto
%	x v_1 + y v_2
%	\end{align*}
%	Let us give a graphical representation of this resolution that will be illustrative of what we will do next.
%	\begin{center}
%		\begin{tikzcd}
%		&& X_1 \arrow[rd, "\cdot v_1"]
%		\\
%		0
%		\arrow[r]
%		& X_1
%		\arrow[ru, "\cdot pv_2"]
%		\arrow[rd, "\cdot(-v_1)"']
%		&& 
%		X_1 \arrow[r, "\varepsilon"]
%		&
%		k \arrow[r]
%		& 
%		0,
%		\\
%		&& X_1 \arrow[ru,"\cdot v_2"']
%		\end{tikzcd}
%	\end{center}
%	where we read each column as a direct sum \footnote{
%		\begin{conjecture}
%			For these types of algebras (i.e.) disconnected Dynkin diagrams we get resolution that look like Hasse-diagrams of partitions.
%			
%		\end{conjecture} 
%	}.
%	To get a resolution of $\Nichols V$, we stitch this resolution together with  the relations $v_1^{N_1} = v_2^{N_2} =0$. This yields
%	\begin{center}
%		\begin{tikzcd}
%		\mathcal{B}(V_1) \arrow[rd, "\cdot v_1^{N_1-1}"]
%		\\
%		&
%		\mathcal{B}(V_1) \arrow[rd,"\cdot v_1"]
%		\\
%		\mathcal{B}(V_1) \arrow[ru, "\cdot p v_2"] \arrow [rd,"\cdot(-v_1)"']		&&
%		\mathcal{B}(V_1) \arrow[rd,"\cdot v_1^{N_{1}-1}"]
%		\\
%		&
%		\mathcal{B}(V_1) \arrow[rd,"\cdot (-v_1^{N_{1}-1})"'] \arrow[ru,"\cdot v_2"]	&&\mathcal{B}(V_1) \arrow[rd,"\cdot v_1"] 
%		\\
%		\mathcal{B}(V_1) \arrow[ru, "\cdot v_2^{N_2-1}"] \arrow[rd,"\cdot v_1^{N_1-1}"']		&&
%		\mathcal{B}(V_1) \arrow[ru,"\cdot p v_2"] \arrow[rd,"\cdot(-v_1)"'] 	&& 
%		\mathcal{B}(V_1) \arrow[r,"\varepsilon"]  & 
%		k \arrow[r]& 
%		0.
%		\\
%		& 
%		\mathcal{B}(V_1)\arrow[rd,"\cdot v_1"'] \arrow[ru,"\cdot p^{-1}v_2^{N_2-1}"]&& \mathcal{B}(V_1) \arrow[ru,"\cdot v_2"']
%		\\
%		\mathcal{B}(V_1)\arrow[ru, "\cdot pv_2"] \arrow[rd,"\cdot(-v_1)"']		&&
%		\mathcal{B}(V_1) \arrow[ru,"\cdot v_2^{N_{2}-1}"']
%		\\
%		&
%		\mathcal{B}(V_1) \arrow[ru,"\cdot v_2"']
%		\\
%		\mathcal{B}(V_1) \arrow[ru,"\cdot v_2^{N_2-1}"']
%		\end{tikzcd}
%	\end{center}
%	Notice that the resolution of $X$ still appears as `tiles' in this resolution.
%	
%	\textbf{Question:} How can we generalise this `stitching' process? Gut feeling is to look at the canonical long exact sequence in homology. 
%	
%	
%	\subsection{Lifting resolutions from the Nichols algebra to its bosonisation}
%	
%	Let $H$ be Hopf algebra (over any field $k$) and $R$ a braided Hopf algebra in $\YD H$. Denote by $\YD{H}$ the category of $H$-Yetter-Drinfeld modules and by $R$-$H$-Mod the category of modules over the ($H$-module algebra) $R$. Finally, denote by 
%	$R$-$\YD{H}$ the subcategory of $R$-modules in $\YD{H}$. That is, objects of 
%	$R$-$\YD{H}$ are pairs $(\mathbb M,\blacktriangleright)$ comprising a $H$-Yetter-Drinfeld module $\mathbb M := (M,\triangleright,\delta)$ and a $H$-linear, $H$-colinear map $\blacktriangleright: R \otimes M \rightarrow M$, which implements an action of $R$ on $\mathbb M$.
%	
%	Similarly objects in $R$-$H$-Mod are pairs $(\mathbb M,\blacktriangleright)$ comprising a $H$-module $\mathbb M := (M,\triangleright)$ and a $H$-linear map $\blacktriangleright: R \otimes M \rightarrow M$, which implements an action of $R$ on $\mathbb M$.
%	Note that there is a forgetful functor $R$-$\YD H\rightarrow R$-$H$-Mod.
%	This functor is exact (since kernels and cokernels come from the kernels and cokernels of the underlying $k$-linear maps). 
%	
%	
%	\begin{lemma}
%		Let $H$ be a f.d. Hopf algebra such that $\text{tr } S^2 \neq 0$. TFAE
%		\begin{enumerate}
%			\item $H$ is semisimple.
%			\item $D(H)$ is semisimple.
%			\item The trivial $H$-module $k_\varepsilon$ is projective.
%			\item The trivial $D(H)$-module $k_\varepsilon$ is projective.
%		\end{enumerate}
%	\end{lemma}
%	\begin{proof}
%		We prove $(1) \Leftrightarrow (3)$ and $(2) \Leftrightarrow (4)$:
%		Assume that $L$ is some f. d.  semisimple Hopf algebra. By Maschke's theorem, there exists a normalised integral $\Lambda \in L$ (i.e. $\varepsilon(\Lambda) =1$). 
%		One shows that $L:= \Lambda L \oplus (1-\Lambda)L$ thus $\Lambda L$ is projective.  Moreover $\Lambda L\cong k_\varepsilon$ implying that the trivial module is projective.
%		Conversely, assume $k_\varepsilon$ to be projective.
%		Now consider the surjective morphism of $L$ modules $\varepsilon: L \rightarrow k_\varepsilon$ . The projectivity implies that there exists a morphism of $L$ modules $\xi: k_\varepsilon \rightarrow L$ such that $\varepsilon \xi = id_k$. Define $\Lambda:= \xi(1)$. By 
%		definition we have $\varepsilon(\Lambda) = \varepsilon\xi (1)= 1$.
%		Moreover for any $l\in L$ we have $l \Lambda = l \xi(1) = \xi (l\triangleright 1) = \varepsilon(l) \Lambda$. 
%		
%		The equivalence between $(1)$ and $(2)$ follows from the following consideration.
%		If $H$ is semisimple and $\text{tr } S^2 \neq 0$, its dual is semisimple. One then shows, using integrals, that $D(H)$ is semisimple. Conversely, if $D(H)$ is semisimple, there exists a non-trivial integral $\mu \in D(H)$.  Define $\Lambda:= (\varepsilon \otimes id)(\mu) \in H$. As $\varepsilon(\Lambda)= \varepsilon\otimes \varepsilon (\mu)=1$, we have $\Lambda \neq 0$. Moreover, a direct computation shows that it is a (non-trivial) integral for $H$. By Maschke's theorem, we then know that $H$ is semisimple.
%		
%	\end{proof}
%	
%	\begin{lemma}
%		Let $H$ be finite dimensional and semisimple.The algebra $R$ is projective in $R$-$\YD H$ (and in $R$-$H$-Mod).
%	\end{lemma}
%	\begin{proof}
%		We show that $R$ is projective in $\YD H$-Mod.
%		There is a natural isomorphism 
%		$\Hom_{R\text{-}\YD H}(R, -) = \Hom_{R\text{-}\YD H}(R\otimes k_\varepsilon, -) = \Hom_{\YD H}(k_\varepsilon, -)$ coming from an adjoint pair of functors, i.e., ``free" and ``forgetful" functors.
%		As $k_\varepsilon$ is projective in $\YD H$ by the previous lemma,  $Hom_{\YD H}(k_\varepsilon, -)$ is exact, implying that $\Hom_{R-\YD H}(R, -)$ is exact. Hence, $R$ is projective in {{$R\text{-}\YD H$-Mod}}.
%	\end{proof}
%	
%	\begin{lemma}
%		Let $H$ f.d. and semisimple and let $M$ be a projective object in $R\text{-}\YD H$-Mod. Then there exists a $Q\in R\text{-}\YD H$-Mod, such that $M \oplus Q  = \oplus_{i \in I} R$.
%	\end{lemma}
%	\begin{proof}
%		Consider the free $R$-module generated by $M$, $R^M:= \oplus_{m \in M} R$
%		Let $\pi: R^M \rightarrow M$, $r_m \rightarrow r\triangleright m$ . Since $M$ is projective there exists a map $\iota: M \rightarrow R^M$, such that $\pi \iota  = id_M$.
%		We claim that $R^M = \text{im } \iota \oplus \ker \pi$.
%		Let $r \in R^M$ and write $r= s + t$, with $s= \iota \pi (r)$ and $t= r-s$.  Then we have $s \in \text{im } \iota$ and $\pi(t) = \pi(r) - \pi \iota \pi(r)= \pi(r) -\pi(r) = 0$. Thus $t \in \ker \pi$.
%		To see that the sum is direct, assume that there exists some $x \in R^M$ such that $x\in \ker \pi$ and $x \in \text{im } \iota$.  As $x \in \text{im } \iota$, there exists $y \in M $ such that $\iota y= x$.
%		As $y= \pi\iota (y) = \pi(x) =0$, it follows that $x=0$.
%	\end{proof}
%	
%	\begin{theorem}
%		Let $H$ be a finite dimensional semisimple Hopf algebra  such that $\text{tr } S^2 \neq 0$ and let $R$ be a braided Hopf algebra over $H$. Write $\Lambda$ for the unique left integral in $H$ such that $\varepsilon(\Lambda)=1$.
%		For any $H$-module $N$, the following constitutes an exact functor:
%		\begin{align}
%		\begin{aligned}
%		F_N\colon &R\text{-}\YD{H} \rightarrow R\#H\text{-}\text{Mod} \\
%		& ( M,\blacktriangleright) \mapsto 
%		\left(M\#N\right)^\Lambda:=((R\# \Lambda)\triangleright(M\#N),\triangleright)\\
%		&f \mapsto \left(f\# id_N\right) |_{	\left(M\#N\right)^\Lambda},
%		\end{aligned}
%		\end{align}
%		with $\triangleright$ given by $\left(r\#h\right) \triangleright \left(m\#n\right) :=  
%		r\blacktriangleright(h_{(1)}\triangleright m) \otimes h_{(2)}\triangleright n$, for ${r \in R}, {h \in H}, {m\in M}$ and  $n \in N$.
%		
%		Moreover, if $N$ is such that there exists a surjective $H$-linear map $N \rightarrow k$, then $F_N$ preserves projective modules.
%	\end{theorem}
%	\begin{proof}
%		To prove that $F_N$ is well-defined, we first define the functor
%		\begin{align}
%		\begin{aligned}
%		F_N'\colon &R\text{-}\YD{H} \rightarrow R\#H\text{-}\text{Mod} \\
%		& (\mathbb M,\blacktriangleright) \mapsto 
%		M\#N:=(M\otimes N,\triangleright)\\
%		&f \mapsto f\# id_N:= f\otimes id_N.
%		\end{aligned}
%		\end{align}
%		where $\triangleright$ is given by $\left(r\#h\right) \triangleright \left(m\#n\right) :=  
%		r\blacktriangleright(h_{(1)}\triangleright m) \otimes h_{(2)}\triangleright n$.
%		
%		Let $M$ be in $R\text{-}\YD H$. We compute, for $r,s \in R$, $g,h,\in H$, $m\in M$ and $n\in N$
%		\begin{align*}
%		(r\#g)&\triangleright((s\#h) \triangleright(m\# n))
%		=(r\# g) \triangleright (s\blacktriangleright (h_ {(1)}\triangleright m) \# h_{(2)} \triangleright n) \\
%		& = r\blacktriangleright( g_{(1)}\triangleright(s\blacktriangleright (h_{(1)}\triangleright m))) \#  g_{(2)}h_{(2)}\triangleright n \\
%		& =  r\blacktriangleright((g_{(1)}\triangleright s)\blacktriangleright (g_{(2)} h_{(1)}\triangleright m)) \#  g_{(3)}h_{(2)}\triangleright n \\
%		& =  (r(g_{(1)}\triangleright s))\blacktriangleright (g_{(2)(1)} h_{(1)}\triangleright m) \#  g_{(2)(2)}h_{(2)}\triangleright n \\
%		& = (r(g_{(1)}\triangleright s)\# g_{(2)}h)\triangleright (m\#n) \\
%		& = (r\# g)(s \# g) \triangleright (m \# n).
%		\end{align*}
%		This shows that $(F_N'(M),\triangleright)$ is a $R\#H\text{-}$module. Moreover, one checks that the action is unital. 	We also compute, given a morphism $f: M \rightarrow  M'$ in $R\text{-}\YD H$,
%		\begin{align*}
%		(f\#\text{id}_N)((r\#g) \triangleright (m \# n)) &= (f\#\text{id}_N)(r\blacktriangleright(g_{(1)}\triangleright m) \otimes g_{(2)}\triangleright n) \\
%		& = r\blacktriangleright(g_{(1)}\triangleright f(m)) \otimes g_{(2)}\triangleright n \\
%		&  = \left(r\#g\right) \triangleright (f\#\text{id}_N)(m\#n).
%		\end{align*}
%		Hence, we conclude that the functor $F_N'$ is well-defined.
%		
%		Next, define the endofunctor
%		\begin{align}
%		\begin{aligned}
%		(-)^\Lambda \colon &R\# H\text{-Mod} \rightarrow R\#H\text{-Mod} \\
%		&X \mapsto X^\Lambda :=(R\#\Lambda) \triangleright X \\
%		&f:X\rightarrow Y \mapsto f|_{X^\Lambda} \colon X^\Lambda \mapsto Y^\Lambda
%		\end{aligned}
%		\end{align}
%		
%		We compute, for some $x\in X$, $r,s\in R$ and $g\in H$,
%		$$
%		(r\#g)\triangleright ((s\#\Lambda) \triangleright x)
%		= (r(g\triangleright s)\# \Lambda)\triangleright x
%		$$
%		and further, for a morphism $f \colon X \to Y$ in $R\#H$-Mod,
%		$$
%		f((s\# \Lambda)\triangleright x) = (s\# \Lambda) f(x),
%		$$
%		hence, $\text{im } f|_{X^\Lambda} \subseteq Y^\Lambda$ and thus, $(-)^\Lambda$ is well defined and we conclude that so is $F_N$, as $F_N= (-)^\Lambda  \circ F_N' $.
%		
%		To prove that $F_N$ preserves exactness, note that $F_N'$ does, since on the level of  morphisms $F_N'(f)= f\otimes \text{id}_N$ and the tensor product over $k$ is exact.
%		By definition, $(-)^\Lambda(f)$ preserves kernels. Moreover, if  that ${f\colon X\rightarrow Y}$  is surjective, then $(-)^\Lambda(f) = f|_{X^\Lambda}$ is surjective too, since: if $y\in Y^\Lambda$, there is $r\in R$ such that $y= (r\#\Lambda) \triangleright y'$. 
%		Letting $x'\in X$ such that $f(x') =y'$, then $f((r\# \Lambda) \triangleright x')=
%		r\# \Lambda \triangleright y' = y$.	Therefore, we conclude that $(-)^\Lambda$ is exact and it follows that $F_N$ is exact too.
%		
%		It remains to see that $F_N$ preserves projectivity, if there exists a surjective $H$-linear map $\widehat \varepsilon: N \rightarrow k$.		 		
%		Let $M$ be  projective in $R\text{-}\YD H$. Then there exists a $Q$ in $R\text{-}\YD H$ such that ${M \oplus Q \cong \oplus_{j\in J} R}$.
%		Applying the forgetful functor $\text{Forg} \colon  {R\text{-}\YD H} \rightarrow R\text{-}H\text{-Mod}$, we see that $M$ is projective as an $R$-$H$-module.
%		
%		
%		
%		Fix a surjection $\beta: X \rightarrow Y$ of $R\#H$-modules and a morphism  $\gamma: F(M) \rightarrow Y$. As there are algebra inclusions of $R$ and $H$ in  $R\#H$, we have that $\beta$ and $\gamma$ are $H$- and $R$-linear.
%		Thus, we can consider the following diagram in $R$-$H$-Mod:		
%		\begin{center}
%			\begin{tikzcd}
%			& M 
%			\arrow[d,dashed, bend left =10,"\lambda"] 
%			\arrow[ddl, dashed, bend right =30 , "\alpha"]\\
%			& F_N(M)  \arrow[d,"\gamma"] 
%			\arrow[u, dashed, bend left =10,"\delta"] \\
%			X \arrow[r,two heads, "\beta"]
%			& Y
%			\end{tikzcd},
%		\end{center}
%		where  $\delta: F_N(M) \rightarrow M$, $m \#  n \mapsto 
%		\widehat \varepsilon (n)  m$. 
%		IF(?) it is a surjective morphism of $R\text{-}H$-modules, there exists a section $\lambda$ since $M$ is projective in $R$-$H$-Mod.  Again by the projectivity of $M$, there exists $\alpha \colon M \to X$ in $R$-$H$-Mod which makes the outer triangle commute\footnote{
%			IDEA for surjectivity.
%			
%			Let $Q$ be such that $M\oplus Q = \oplus_{i\in I} R$ consider the $\YD H$ map $\eta: k \rightarrow \oplus_{i\in I}$, $1_k \mapsto (1,0,...,0)$. 
%			Let $\pi: \oplus R \rightarrow M$ be the projection onto  $M$ and write $\eta' = \pi \circ \eta$.
%			Now let $n\in N$ be such that $\hat \varepsilon(n) =1$ and choose some $m\in M$.
%			Then $m = \pi(r_1,\dotsc r_k)$. Let wlog $r_2\dotsc r_k =0$. Then consider the element
%			$m'= \eta'(1) \in M$. We compute
%			$$
%			\delta( r_1\Lambda_{(1)}m' \# \Lambda_{(2)}n ) = \varepsilon(\Lambda n) r_1 m' = m.
%			$$
%		}.
%		We show that the composition $\alpha \delta$ extends in fact to a morphism of $R\# H$-modules. We compute 
%		\begin{align*}
%		(\alpha \delta) ((r\#g) \triangleright (s \blacktriangleright (\Lambda_{(1)} \triangleright m)\#\Lambda_{(2)}n)) 
%		& = (\alpha \delta) (r(g_{(1)} \triangleright s) \blacktriangleright ((g_{(2)}\Lambda_{(1)})\triangleright m) \# g_{(3)} \Lambda_{(2)} n)	\\
%		&= \widehat \varepsilon(n) r(g \triangleright s) \blacktriangleright(\Lambda \triangleright \alpha (m)) \\
%		%			& = \widehat \varepsilon(n) (r(g \triangleright s)\# 1)(1 \# \Lambda) \triangleright \alpha(m) \\
%		& = \widehat\varepsilon(n)  (r(g \triangleright s)\# \Lambda) \triangleright \alpha (m)).
%		\end{align*}
%		and
%		\begin{align*}
%		(r\#g)  \triangleright ((\alpha \delta)(s \blacktriangleright (\Lambda_{(1)} \triangleright m)\#\Lambda_{(2)}n))
%		& = \widehat\varepsilon(n) (r\#g)  \triangleright (s \blacktriangleright (\Lambda \triangleright \alpha(m) )) \\
%		& = \widehat\varepsilon(n)  (r\# g)(s\# \Lambda) \triangleright \alpha (m)	 \\
%		& = \widehat\varepsilon(n)  (r(g \triangleright s)\# \Lambda) \triangleright \alpha (m)).
%		\end{align*}
%		Since $\lambda \delta = \text{id}_M$ and $\beta \alpha =  \gamma\lambda$, we have
%		$\beta \alpha\delta =  \gamma \lambda \delta = \gamma$.
%		Thus, $F_N(M)$ is projective.
%	\end{proof}
%	
%	
%	---TODO:CHECK WHETHER THE ARGUMENTS REALLY HOLD AND HOW THEY CAN BE GENERALISED ---
%	
%	\begin{theorem}
%		Let $H$ be a fd semisimple Hopf algebra over some field $k$ such that $\text{tr }S^2 \neq 0$ and $R$ a braided Hopf algebra over $H$. 
%		Write $\Lambda$ for the unique left integral of $H$ such that $\varepsilon(\Lambda) =1$.
%		Given any projective resolution $(P_n, d_n)$ of $k$ viewed as the trivial $R$ module and any simple $H$ module $N$ we obtain a projective resolution $F_N(P_n, d_n)$ of the trivial $R\#H$-module $k$.
%	\end{theorem}
%	\begin{proof}
%		As simplicity implies projectivity $N$ is in particular projective.
%		Choose any $n\in N$ such that $Hn =N$, and define $\widehat \varepsilon(g n):=\varepsilon(g)$. This is a morphism of $H$ modules and  as $\widehat \varepsilon(n) \neq 0$ it is surjective. 
%		By the previous theorem we know that $F_N(P_n,d_n)$ is a projective resolution of $F_N(k\varepsilon)$. 
%		Let us therefore study $F_N(k_\varepsilon)$
%		As 
%		$
%		\widehat \varepsilon(\Lambda \triangleright n)= \varepsilon(\Lambda) \widehat \varepsilon(n) = \varepsilon(n) \neq 0.
%		$
%		we have $\Lambda \triangleright n \neq 0$.
%		Now consider any $n' \in N$. By definition of $n$ there exists a $g\in H$ such that $n' = g \triangleright n$ and we have
%		$\Lambda \triangleright n' = \Lambda g \triangleright n = \alpha (g) \Lambda \triangleright n$, where $\alpha \in H^*$ is the distinguished group-like of $H^*$.  Thus $\Lambda \triangleright N \cong k_\varepsilon$.
%		
%		Note: We did not use simplicity but being generated by one element for the most part. Does being generated by one element imply projectivity?
%	\end{proof}
%	
%	
%	SKETCH  TO FURTHER THIS PROJECT:
%	Let $(V,c)$ be a braided vector space of diagonal type with $q$ as matrix of the braiding 
%	Alter the diagonal entries of $q$ such that the newly obtained matrix $q'$ is of Cartan type.
%	Denote by $V'$ the corresponding braided  vector space. Produce a proj res of the trivial module $\Nichols {V'}$ show that it can be turned into a projective resolution  of the trivial $\Nichols V$ module.
%	Use the above thm to obtain a proj res of the trivial module over the bosonisation.
%	
%	
%	\begin{enumerate}
%		\item Let $\mathfrak q$ be the matrix of a braiding. Associate to it the generalised Cartan matrix $C$ as constructed in Definition 2.22 in the Andrus-Angiono survey.
%		\item BGG resolution is build around Verma module. We need to understand those in our setting. This we should do on the level of a Pre-Nichols algebra $\tilde B(V)$.
%		\item There exists a short exact sequence of (?) $k$-vector spaces
%		$$
%			0 \rightarrow X \rightarrow \tilde B(V) \rightarrow \Nichols V \rightarrow 0.
%		$$
%		Where $X$ is generated by the set $x_1^{N_1} \dotsc x_n^{N_n}$.
%		Assume we have a nice resolution of the algebras $X$ (coming from $\tilde B(V)$ ?) and $\tilde B(V)$ (here should be the BGG part) we obtain a long exact sequence of homology groups via the snake lemma.
%	\end{enumerate}
%
%
%
%	Idea: Consider a braided vector space $(V,c)$, such that $\Nichols V$ is of Cartan type.
%	Consider its root system and denote by $W$ its Weyl-group. Its Weyl group is a coxeter group generated by $S$, the set of reflections which is indexed by the elements constituting a PBW basis of $\Nichols V$.
%	
%	Let $w\in W$ be the longest word of $W$.
%		
%	\begin{conjecture}
%		Let $w\in W$ as before and denote by $l(w)$ the length of $w$. 
%		Then
%		\begin{center}
%		\begin{tikzcd}
%			0 \arrow[r]&
%			B \arrow[r, "d_{l-1}"]&
%			\oplus_{w_1} B &
%			\dotsc
%		\end{tikzcd}
%		\end{center}
%	Yields a free resolution of the pre-Nichols algebra of $V$.
%	\end{conjecture}
%	\begin{proof}
%		Freeness is clear it is exactness we need to care about
%		
%	\end{proof}
%
%	Idea: The distinguished pre-Nichols algebra plays the $n$-part in the triangular decompostion. Now  the $h$ part coms from a vector space which is (if its f.d) dual to a space containing the roots.
%	So we should get it out of our Cartan matrix $C$.
%	We should try to define the borel algebra $U(b)$ in terms of $h$ and the distinguished Nichols algebra.
%	\begin{lemma}
%		Let $(V,c)$ be a finite-dimensional braided vector space of diagonal type with matrix of the brading $(q_{ij})_{1 \leq i,j \leq \theta}$.
%		Let $G$ be any group such that $(V,c)$ is a Yetter-Drinfeld module over $kG$.
%		Then
%		\begin{itemize}
%			\item There exists a Yetter-Drinfeld module $(V',c')$ over $G$ such that
%			the matrix of the braiding $q'$ satisfies:
%			$q_{ij} = q'_{ij}$ for $i \neq j$ and $q'$ is of Cartan type i.e. there exists a generalised Cartan matrix $A$ such that $q'_{ij}= q{ii}^{A_{ij}}$.
%			\item There exists (unique?) pre-Nichols algebras $B(V)$ and $B(V')$ such that $B(V)$ and $B(V')$ are isomorphic as $k$-algebras. 
%		\end{itemize}		
%	\end{lemma}
%	
%	\begin{proof}
%		Consider the matrix of the braiding $q_{ij}$ and let $Q_i:=\{q_{ij}|1 \leq j \leq \theta\}$.
%		As $Q_i$ generates acyclic subgroup of $k^\times$ there exists some generator $x_{i}$. Now set $q'_{ij}= q_{ij}$ if $i\neq j$ and $q'_{ii}=x_i$ and fix 
%		$A_{ij}= \max{n\in -\mathbb N | x_i^n=q_{ij} }$ if $i \neq j$ and  $A_{ii}=2$. This is by construction a generalised Cartan matrix.
%		Moreover $(V',c')$ is constructed in the obvious way.
%	\end{proof}
%
%\section{Coxeter groups}
%
%\begin{definition}
%	Let $S$ be a set of order $N>1$ and $M:=(m_{ij})$ an integral $N\times N$ matrix such that
%	\begin{enumerate}
%		\item $m_{ii}=1$
%		\item $m_{ij} = m_{ji} \geq 2 $ for $i \neq j$.
%	\end{enumerate} 
%	The group $W$ is the quotient of the free group generated by $S$ by the relations
%	$(s_i s_j)^{m_{ij}}=e$. The pair $(W, S)$ is called a Coxeter system.
%	
%	A reflection in $W$ is an element $t:= wsw^{-1}\in W$, with $w \in W$ and $s\in S$. We write $T$ for the set of reflections of $W$. 
%\end{definition}
%
%\begin{definition}
%	Let $(W,S)$ be a Coxeter system and $s_1\dotsc s_n \in W$. Write
%	$t_{n-i}:= s_n s_{n-1} \dotsc s_{n-i} \dotsc s_n \in T$ for $1 \leq i \leq n$. The inversion order of $s_1\dotsc s_k$ is given by the tuple $(t_n, \dotsc , t_1)$.
%\end{definition}
%
%\begin{remark}
%	It holds that
%	$$
%		s_1 \dotsc s_n t_{n-i} = s_1 \dotsc s_{i-1} s_{i+1} \dotsc s_n.
%	$$
%	and
%	$$
%		s_1 \dotsc s_i = t_i \dotsc t_1.
%	$$	
%	If $w= s_1\dotsc s_k$ with $k$ minimal then $t_i \neq t_j$ for $i\neq j$.
%\end{remark}
%	
%	
%\begin{definition}
%	Let $(W,S)$ be a Coxeter system.
%	We define the length of an element $w\in W$ via
%	$$
%		l_S(w)= \min_{k\in \mathbb N}\{ s_1 \dotsc s_k= w \mid s_1,\dotsc s_k \in S\}.
%	$$
%	We call a word $s_1\dotsc s_k$ reduced if it is of minimal length.
%\end{definition}
%
%\begin{remark}
%	It holds that
%	$$
%		l_S(u)+l_S(v) \geq l_S(uv), \qquad l_S(u) = l_S (u^{-1}).
%	$$
%\end{remark}
%
%\begin{definition}
%	Consider a Coxeter system $(W,S)$ and $u,v \in W$.
%	We write
%	\begin{enumerate}
%		\item $u \leftarrow v$ iff $l_S(u) +1= l_S(v)$ and there exists a $t\in T$ such that $ut=v$.
%		\item $u\leq v$ if $u=u_0 \leftarrow u_1 \leftarrow \dotsc \leftarrow u_n =v$.
%	\end{enumerate}
%\end{definition}
%
%\begin{definition}
%	Let $(W,S)$ be a Coxeter system.
%	Its Bruhat graph is the direct labeled graph which is defined by:
%	\begin{enumerate}
%		\item Its vertices are the elements of $W$.
%		\item There is an edge form $v$ to $u$ whenever $u \leftarrow v$. It is labeled by the transposition $t$ such that $ut=v$
%	\end{enumerate}
%\end{definition}
%
%\begin{example}
%	Consider the Coxeter-group $S_4$ with three generators $a,b,c$ and relations
%	$$
%		(ab)^3=(bc)^3=(ac)^2 = e \quad\quad \text{ and } \quad\quad a^2 = b^2 = c^2 = e. 
%	$$
%	The 24 elements of $S_4$ are
%	\begin{gather*}
%		\begin{matrix}
%			e 		&			&			&			&			& 		\\
%			a 		& b 		& c 		&			&			&		\\
%			ab 		& ac 		& ba		& bc		& cb 		&		\\
%			aba		& abc		& acb		& bac		& bcb		& cba	\\
%			abac	& abcb	 	& acba 		& bacb 		& bcba		&		\\
%			abacb	& abcba		& bacba		&			&			&		\\
%			abacba
%		\end{matrix}
%	\end{gather*}
%	
%	Let us now determine all edges in the bruhat graph:
%	\begin{gather*}
%		e \overset{a}\rightarrow a, \quad
%		e \overset{b}\rightarrow b, \quad
%		e \overset{c}\rightarrow c, \quad
%		\\
%		\\
%		a \overset{b}\rightarrow ab, \quad
%		a \overset{aba}\rightarrow ba, \quad
%		a \overset{c}\rightarrow ac, \quad
%		\\
%		b \overset{a}\rightarrow ba, \quad
%		b \overset{aba}\rightarrow ab, \quad
%		b \overset{c}\rightarrow bc, \quad
%		b \overset{bcb}\rightarrow cb, \quad
%		\\
%		c \overset{a}\rightarrow ac, \quad
%		c \overset{b}\rightarrow cb, \quad
%		c \overset{bcb}\rightarrow bc, \quad
%		\\
%		\\
%		ab \overset{a}\rightarrow aba, \quad
%		ab \overset{c}\rightarrow abc, \quad
%		ab \overset{bcb}\rightarrow acb, \quad
%		\\
%		ac \overset{b}\rightarrow acb, \quad
%		ac \overset{aba}\rightarrow cba, \quad
%		ac \overset{bcb}\rightarrow abc, \quad
%		ac \overset{abcba}\rightarrow bac, \quad
%		\\
%		bc \overset{abcba}\rightarrow abc, \quad
%		bc \overset{a}\rightarrow bac, \quad
%		bc \overset{b}\rightarrow cbc, \quad 
%		\\
%		cb \overset{aba}\rightarrow acb, \quad
%		cb \overset{a}\rightarrow cba, \quad
%		cb \overset{c}\rightarrow bcb, \quad 
%	\end{gather*}
%	The squares in this graph are
%	
%	\begin{center}
%		\begin{tikzcd}
%			&\circ
%			\arrow{dr}{b}\\
%			\circ 
%			\arrow{ur}{a}
%			\arrow{dr}{b}
%			&& \circ \\
%			& \circ 
%			\arrow{ur}{aba}
%			\\
%			&\circ
%			\arrow{dr}{aba}\\
%			\circ 
%			\arrow{ur}{a}
%			\arrow{dr}{b}
%			&& \circ \\
%			& \circ 
%			\arrow{ur}{a}
%			\\
%			&\circ
%			\arrow{dr}{bcb}\\
%			\circ 
%			\arrow{ur}{a}
%			\arrow{dr}{bcb}
%			&& \circ \\
%			& \circ 
%			\arrow{ur}{abcba}\\
%			&\circ
%			\arrow{dr}{abcba}\\
%			\circ 
%			\arrow{ur}{aba}
%			\arrow{dr}{c}
%			&& \circ \\
%			& \circ 
%			\arrow{ur}{aba}
%		\end{tikzcd}
%		\begin{tikzcd}
%			&\circ
%			\arrow{dr}{c}\\
%			\circ 
%			\arrow{ur}{a}
%			\arrow{dr}{c}
%			&& \circ \\
%			& \circ 
%			\arrow{ur}{a}
%			\\
%			&\circ
%			\arrow{dr}{c}\\
%			\circ 
%			\arrow{ur}{aba}
%			\arrow{dr}{c}
%			&& \circ \\
%			& \circ 
%			\arrow{ur}{abcba}\\
%			&\circ
%			\arrow{dr}{abcba}\\
%			\circ 
%			\arrow{ur}{a}
%			\arrow{dr}{bcb}
%			&& \circ \\
%			& \circ 
%			\arrow{ur}{a}
%		\end{tikzcd}
%		\begin{tikzcd}
%			&\circ
%			\arrow{dr}{c}\\
%			\circ 
%			\arrow{ur}{b}
%			\arrow{dr}{c}
%			&& \circ \\
%			& \circ 
%			\arrow{ur}{bcb}
%			\\
%			&\circ
%			\arrow{dr}{bcb}\\
%			\circ 
%			\arrow{ur}{b}
%			\arrow{dr}{c}
%			&& \circ \\
%			& \circ 
%			\arrow{ur}{b}
%			\\
%			&\circ
%			\arrow{dr}{bcb}\\
%			\circ 
%			\arrow{ur}{aba}
%			\arrow{dr}{bcb}
%			&& \circ \\
%			& \circ 
%			\arrow{ur}{aba}
%		\end{tikzcd}
%	\end{center}
%	
%	
%	
%	
%	The Bruhat graph then is
%	\begin{center}
%		\begin{tikzcd}[column sep= tiny, row sep= 45pt]
%			&&&&& abacba 
%				\arrow[blue]{dll}[near start, description]{a}
%				\arrow[red]{d}[near start, description]{b}
%				\arrow[teal]{drr}[near start, description]{c}
%			\\
%			&&& abacb
%				\arrow[red]{dll}[near start, description]{b}
%				\arrow[orange]{d}[near start, description]{aba}
%				\arrow[teal]{drr}[near end, description]{c}
%			&& abcba
%				\arrow[orange]{dllll}[near start, description]{aba}
%				\arrow[blue]{dll}[near start, description]{a}
%				\arrow[purple!70!black]{drrrr}[near end, description]{bcb}
%				\arrow[teal]{drr}[near start, description]{c}
%			&& bacba
%				\arrow[blue]{dll}[near start, description]{a} 
%				\arrow[red]{drr}[near start, description]{b}
%				\arrow[purple!70!black]{d}[near start, description]{bcb}\\
%			& abac
%				\arrow[teal]{ddl}[near start, description]{c}
%				\arrow[blue]{ddr}[near start, description]{a}
%				\arrow[purple!70!black]{ddrrrrr}[near start, description]{bcb}
%			&& abcb
%				\arrow[red]{ddl}[near end, description]{b}
%				\arrow[teal]{ddr}[near start, description]{c}
%				\arrow[magenta!70!green]{ddrrrrrrr}[near start, description]{abcba}
%			&& bacb 
%				\arrow[purple!70!black]{ddlllll}[near start, description]{bcb}
%				\arrow[magenta!70!green]{ddl}[near end, description]{abcba}
%				\arrow[red]{ddr}[near end, description]{b}
%				\arrow[orange]{ddrrrrr}[near start, description]{aba}
%			&& acba 
%				\arrow[magenta!70!green]{ddlllllll}[near end, description]{abcba}
%				\arrow[blue]{ddlll}[near end, description]{a}
%				\arrow[red]{ddr}[near start, description]{b}
%			&& bcba
%				\arrow[blue]{ddr}[near start, description]{a}
%				\arrow[teal]{ddl}[near start, description]{c}
%				\arrow[orange]{ddlll}[near start, description]{aba}
%			\\ \\
%			aba 
%				\arrow[blue]{ddr}[near start, description]{a} 
%				\arrow[red]{ddrrr}[near start, description]{b}
%%				
%			&& abc 
%				\arrow[teal]{ddl}[near start, description]{c} 
%				\arrow[purple!70!black]{ddrrr}[near start, description]{bcb}
%				\arrow[magenta!70!green]{ddrrrrr}[near start, description]{abcba}
%			&& acb
%				\arrow[purple!70!black]{ddlll}[near end, description]{bcb}
%				\arrow[red]{ddr}[near start, description]{b} 
%				\arrow[orange]{ddrrrrr}[near end, description]{aba}
%%				
%		 	&& bac 
%		 		\arrow[teal]{ddlll}[near start, description]{c} 
%		 		\arrow[magenta!70!green]{ddl}[near start, description]{abcba}
%		 		\arrow[blue]{ddr}[near start, description]{a}
%		 	&& cba
%		 		\arrow[blue]{ddr}[near start, description]{a} 
%		 		\arrow[orange]{ddlll}[near start, description]{aba}
%		 		\arrow[magenta!70!green]{ddlllll}[near start, description]{abcba}
%		 	&& bcb 
%		 		\arrow[red]{ddlll}[near start, description]{b} 
%		 		\arrow[teal]{ddl}[near start, description]{c} \\ \\
%			& ab 
%				\arrow[orange]{drrrr}[near start, description]{aba}
%				\arrow[red]{drr}[near start, description]{b}
%			&& ba 
%				\arrow[blue]{drr}[near start, description]{a}
%				\arrow[orange]{d}[near start, description]{aba}
%			&& ac 
%				\arrow[teal]{dll}[near start, description]{c}
%				\arrow[blue]{drr}[near start, description]{a}
%			&& bc 
%				\arrow[purple!70!black]{d}[near start, description]{bcb}
%				\arrow[teal]{dll}[near start, description]{c}
%			&& cb 
%				\arrow[red]{dll}[near start, description]{b}
%				\arrow[purple!70!black]{dllll}[near start, description]{bcb}\\
%			&&& a 
%				\arrow[blue]{drr}[near start, description]{a}
%			&& b
%				\arrow[red]{d}[near start, description]{b}
%			 && c 
%				\arrow[teal]{dll}[near start, description]{c}\\
%			&&&&& e
%		\end{tikzcd}
%	\end{center}
%
%	\begin{center}
%		\begin{tikzpicture}[x=2cm, y=2cm, z=1cm]
%		\tikzset{-Circle}
%			% Axes
%			\draw [->, dashed] (-4,0,0) -- (4,0,0) node [at end, right] {$x$};
%			\draw [->, dashed] (0,-4,0) -- (0,4,0) node [at end, left] {$y$};
%			\draw [->, dashed] (0,0,-4) -- (0,0,4) node [at end, left] {$z$};
%			
%			\coordinate (e) at (0,0,0);
%			 
%			\coordinate (a) at (1,0,0); 
%			\coordinate (b) at (-0.5,  0.7071, -0.5); 
%			\coordinate (c) at (0,0,1); 
%			
%			\coordinate (a+b) at (0.5,  0.7071, -0.5);
%			\coordinate (a+b+c) at at (0.5,  0.7071, 0.5);
%			\coordinate (b+c) at (-0.5,  0.7071, 0.5);
%			
%			\coordinate (ab) at ($(b)+(a+b)$);
%			\coordinate (ba) at ($(a)+(a+b)$);
%			\coordinate (ac) at ($(a)+(c)$);
%			\coordinate (bc) at ($(c)+(b+c)$);
%			\coordinate (cb) at ($(b)+(b+c)$);
%			
%			\coordinate (aba) at ($(ab)+(a)$);
%			\coordinate (acb) at ($(ab)+(b+c)$);
%			\coordinate (abc) at ($(bc)+(a+b+c)$);
%			\coordinate (cba) at ($(ac)+(a+b)$);
%			\coordinate (bac) at ($(ac)+(a+b+c)$);
%			\coordinate (bcb) at ($(cb)+(c)$);
%			
%			
%			\draw[blue] (e) -- (a) node [at end, right] {$a$};
%			\draw[red] (e) -- (b) node [at end, right] {$b$};
%			\draw[teal] (e) -- (c) node [at end, right] {$c$};
%			
%			\draw[blue!50!red] (b) -- (ab) node [at end, right] {$ab$};
%			\draw[red] (a) -- ($(a)+(b)+(b)$);
%			
%			\draw[blue!50!red] (a) -- (ba) node [at end, right] {$ba$};
%			\draw[blue] (b) -- ($(b)+(a)+(a)$);
%			
%			\draw[teal] (a) -- (ac) node [at end, right] {$ac$};
%			\draw[blue] (c) -- ($(c)+(a)$);
%			
%			\draw[red!50!teal] (c) -- (bc) node [at end, right] {$bc$};
%			\draw[teal] (b) -- ($(b)+(c)+(c)$);
%			
%			\draw[red!50!teal] (b) -- (cb) node [at end, right] {$cb$};
%			\draw[red] (c) -- ($(c)+(b)+(b)$);
%			
%			\draw[blue] (ab) -- (aba) node [at end, right] {$aba$};
%			\draw[red] (ba) -- ($(ba)+(b)$);
%			
%			\draw[red!50!teal] (ab) -- (acb) node [at end, right] {$acb$};
%			\draw[red] (ac) -- ($(ac)+3*(b)$);
%			\draw[blue!50!red] (cb) -- ($(cb)+(a+b)$);
%			
%			\draw[orange] (bc) -- (abc) node [at end, right] {$abc$};
%			\draw[teal] (ab) -- ($(ab)+3*(c)$);
%			\draw[red!50!teal] (ac) -- ($(ac)+2*(b+c)$);
%			
%			
%			
%%			\draw (a) -- ($(a)+(a+b+c)$)node [at end, right] {$x$};
%		\end{tikzpicture}
%	\end{center}
%\end{example}
%
%
%
%%\begin{definition}
%%	Given any $w \in W$ we define the inversion set and descending set of $w$ via
%%	$$
%%		Inv(w) = \{t \in T | l_S(tw) < l_S(w) \}, \qquad Des(w) = Inv(w) \cap S.
%%	$$
%%\end{definition}
%%
%%\begin{theorem}
%%	Given a Coxeter system $(W,S)$ and a reduced word $s_1 \dotsc s_i$ we have
%%	$t\in Inv(w)$ iff $t= s_1\dotsc s_{i-1} s_i s_{i-1} \dotsc s_1$.
%%\end{theorem}
%%\begin{proof}
%%	$\Leftarrow$ is clear.
%%	$\Rightarrow$ is a bit more involved, see below.
%%\end{proof}
%%
%%\begin{definition}
%%	The root system assigned to $(W,S)$ is the set $R:= T \times\{\pm 1\}$.
%%\end{definition}
%%
%%\begin{definition}
%%	Given any word $s_1 \dotsc s_k$ and any reflection $t\in T$ we define 
%%	$$
%%		\#(s_1\dotsc s_k ; t):= \{ 1\leq i \leq k \mid t_i = t\}, \qquad \eta(s_1\dotsc s_k; t) := (-1)^{\# (s_1\dotsc s_k ;t)}.
%%	$$.
%%\end{definition}
%%
%%\begin{lemma}
%%	There is a unique embedding of groups $\pi: W \rightarrow Sym(R)$ such that
%%	$\pi_s((t,\epsilon)) = (sts, \epsilon \eta (s,t))$.
%%\end{lemma}
%%\begin{proof}
%%	We first check that $\pi$ is well defined on $S$.
%%	Note that $\eta(s, sts)= \eta(s,t))$ thus
%%	$$
%%		\pi_s^2(t,\epsilon) = \pi_s(sts, \epsilon \eta(s,t)) =(t, \epsilon).
%%	$$
%%	Now consider $u, v\in S$ and let $p:= \min_{n\geq 1}\{(uv)^n = 0\}$.
%%	Observe that $(uv)^p$ gives rise to a word of length $2p$ with alternating letters $u$ and $v$ and any $t_i \in T((uv)^p)$ is of the form $(uv)^{i-1}u$.
%%	Moreover $(uv)^p=0$ implies $t_{i+p} =(uv)^{p+i-1}u =(uv)^{i-1}u = t_i$ for $1\leq i \leq p$, implying that $\eta((uv)^p; t)\in 2 \mathbb N$.
%%	Thus, by the universal property of the quotient, 
%%	$\pi$ is a morphism of groups.
%%	
%%	To see that $\pi$ is injective consider any reduced word $W:=s_1\dotsc s_k$.
%%	Fix an element $(s_1,\epsilon) \in R$. As $\#(s_1\dotsc s_k ; s_1) =1$ it follows that $\pi_w(s_1,\epsilon) = (w s_1 w, -\epsilon) \neq (s_1, \epsilon)$.
%%\end{proof}
%%
%%\begin{remark}
%%	For any $t\in T$ we have $\pi_t (t,\epsilon) = (t,-\epsilon)$.
%%\end{remark}
%%
%%We can now proof the theorem
%%\begin{proof}
%%	Assume $t\in Inv(w)$, that is $l_S(tw)< l_S(w)$. One shows that $\eta(w,t) = -1$, i.e. that $t = t_i$ for some $i\in \{1\dotsc ,k\}$ thus $l_S(tw)< l_S(w)$.
%%\end{proof}
%%
%%\begin{remark}
%%	It holds that $\# Inv(w) = l_S(w)$ since if $w= s_1\dotsc s_k$ is a reduced word $Inv(w)$ consists of $k$ reflections. Conversely let $\# Inv(w) =k$ then $w$ has a least  $k$ reflections and in a reduced word $t_i \neq t_j$ for $i\neq j$ implies it has at most $k$ reflections.
%%\end{remark}
%%
%%
%%TODO----
%
%
%\begin{definition}
%	Let $(W,S)$ be a Coxeter system and $u,v\in W$ we write
%	$u < v$ iff $l(u) < l(v)$ and there exists a sequence of reflections $t_1,\dotsc t_n$ such that
%	$u t_1\dotsc t_n =v$. This partial order is called the Bruhat order.
%	An arrow $u\rightarrow v$ is a tuple $(u,v)$ with $u<v$ such that there exists a $t\in T$ with $ut=v$
%\end{definition}
%
%\begin{conjecture}
%	Let $u$, $v$ be such that $l(u)+2 = l(v)$ then 
%	$$ \{ w\in W | u\rightarrow w \rightarrow v \} $$ has cardinality either two or zero.
%\end{conjecture}
%
%
%\begin{example}
%	Fix the generators $a= (12)$, $b=(13)$ of the group $S_3$ and observe that 
%	$(ab)^3=id$.
%	The elements of $S_3$ are
%	$$
%		id, \quad a, \quad b, \quad ab, \quad ba, \quad aba=bab.
%	$$
%	Observe that $a < ab$ and $a< ba$ since $a(aba) = ba$.
%	\begin{center}
%	\begin{tikzcd}
%		& aba = bab	\\
%		ab 		\arrow[ur,"a"]				
%		&& ba	\arrow[ul,"b"'] \\
%		a 		\arrow[u,"b"]
%				\arrow[urr,"aba"']			
%		&& b	\arrow[u,"a"']
%				\arrow[ull,"aba"] \\
%		&id		\arrow[ul,"a"]
%				\arrow[ur,"b"']
%	\end{tikzcd}
%	\end{center}
%\end{example}	
%
%Now consider a Nichols algebra $B$ of diagonal type corresponding to type $A_2$ i.e. whose Weyl group is the above group $S_3$.
%Let it be generated by $x_1, x_2$ by AngAndr the PBW basis is generated by $x_1, x_{12}, x_2$ and the relations are
%$$
%	[x_1,x_{12}]= 0, \quad
%	[x_1,x_2] = x_{12} \quad
%	[x_{12},x_2]= 0.
%$$
%The following complex is a projective resolution.
%\begin{center}
%	\begin{tikzcd}
%		& & B \arrow[r, "\cdot x_1^2"] \arrow[rdd, "\cdot x_{12}"'] & B \arrow[rd,"\cdot x_2"]\\
%		0 \arrow[r] & B\arrow[ur,"\cdot x_2"] \arrow[dr,"\cdot x_1"'] & && B \arrow[r,"\epsilon"] & k \arrow[r] & 0\\
%		& & B \arrow[r, "\cdot x_2^2"'] \arrow[ruu, "\cdot x_{12}"'] & B \arrow[ru,"\cdot x_1"']\\
%	\end{tikzcd}
%\end{center}
%\begin{proof}
%	First note that it is a free complex therefore projective.
%	Exactness can be checked for each square (in the inner). that the last map is injective is clear and that the first map generates the kernel of $\epsilon$ is clear as well.
%	
%	Idea for the squares:  Multiplication with a root vector corresponds in some way to the cooresponding reflection in the Weyl group.
%
% 	Roughly speaking:  ?
%	
%\end{proof}
%
%
%A more involved example
%
%Let $B$ be the Nichols algebra corresponding to $G_2$-type.
%That is the root system looks like
%\begin{center}
%	\begin{tikzpicture}
%	\foreach\ang in {60,120,...,360}{
%		\draw[->,black,thick] (0,0) -- (\ang:2cm);
%	}
%	\foreach\ang in {30,90,...,330}{
%		\draw[->,black,thick] (0,0) -- (\ang:3cm);
%	}
%	\node[anchor=south west,scale=0.6] at (2,0) {$\alpha$};
%	\node[anchor=south east,scale=0.6] at (150:3cm) {$\beta$};
%	\node[anchor=south east,scale=0.6] at (120:2cm) {$\alpha+\beta$};
%	\node[anchor=south west,scale=0.6] at (60:2cm) {$2\alpha+\beta$};
%	\node[anchor=south west,scale=0.6] at (30:3cm) {$3\alpha+\beta$};
%	\node[anchor=south,scale=0.6] at (90:3cm) {$3\alpha+2\beta$};
%	\end{tikzpicture}
%\end{center}
%We fix an convex ordering on the positive roots, i.e. an ordering such that
%$x< x+y<y$ whenever $x+y\in \Delta^+$.
%Such an  ordering is given for example by
%$\alpha < 3 \alpha + \beta < 2\alpha + \beta < 3\alpha +2 \beta  <  \alpha+\beta < \beta$.
%Let us compute its Weyl group. We denote its generators by $s_a$ and $s_b$.
%We write the reflections along simple roots in a matrix form
%$$ 
%	s_a:=
%	\begin{pmatrix}
%		\alpha &
%		3 \alpha + \beta & 
%		2 \alpha + \beta & 
%		3 \alpha + 2 \beta & 
%		\alpha + \beta & 
%		\beta \\
%		-\alpha &
%		\beta &
%		\alpha + \beta &
%		3 \alpha + 2 \beta &
%		2 \alpha + \beta &
%		3 \alpha + \beta
%	\end{pmatrix}
%$$
%$$ 
%	s_b:=
%	\begin{pmatrix}
%		\alpha &
%		3 \alpha + \beta & 
%		2 \alpha + \beta & 
%		3 \alpha + 2 \beta & 
%		\alpha + \beta & 
%		\beta \\
%		\alpha + \beta &
%		3 \alpha+ 2 \beta &
%		2 \alpha + \beta &
%		3 \alpha +  \beta &
%		 \alpha &
%		- \beta
%	\end{pmatrix}
%$$
%
%From this we compute every element of the Weyl group explicitly: 
%\begin{center}
%	\begin{tikzcd}
%		& a \arrow[r, "b"] \arrow[rdd, bend left =10, "aba"] &
%		ab  \arrow[r, "a"] \arrow[rdd, bend right =10, "babab"']  &
%		aba \arrow[r, "b"] \arrow[rdd, bend left =10, "babab"] &
%		abab \arrow[r, "a"] \arrow[rdd, bend right =10, "aba"'] &
%		ababa \arrow[rd,"b"]
%		\\
%		e \arrow[ru, "a"]\arrow[rd, "b"'] & & & & & & ababab = bababa 
%		\\
%		& b \arrow[r, "a"'] \arrow[ruu, bend left =10, "bab"] & 
%		ba \arrow[r, "b"'] \arrow[ruu, bend right =10, "ababa"'] & 
%		bab \arrow[r, "a"'] \arrow[ruu, bend left =10, "ababa"]& 
%		baba \arrow[r, "b"'] \arrow[ruu, bend right =10, "bab"']&
%		babab \arrow[ru,"a"'] 
%	\end{tikzcd}
%\end{center}
%

\section{The weight space and root system of a finite-dimensional Nichols algebra of Cartan type}

We put ourselves into a nice world and work over an algebraically closed field of characteristic zero.
Within this setting - as a consequence of the theorems of Maschke and Schur- every module over (the group algebra of ) an abelian group is semisimple and the simple modules are one-dimensional. 

\subsection{Generalised Cartan matrices}

In the following we want to introduce a setup which allows us to describe the representation theory of a (special) finite abelian group by geometric means. The geometric information is governed by a root system obtained from a generalised Cartan matrix.

\begin{definition}{\cite[1.1.1 and 1.1.5]{Kumar2002}}
	A \emph{generalised Cartan matrix} (short GCM) of size $N$ is a square matrix
	$A := (a_{ij})_{1 \leq i,j \leq N}$ with integral coefficients such that for each pair $(i,j)$ the following conditions hold:
	\begin{enumerate}
		\item $a_{ii}= 2$
		\item $a_{ij}\leq 0 $ if $i \neq j$
		\item $a_{ij} = 0 \Leftrightarrow a_{ji} = 0$.
	\end{enumerate}

	The matrix $A$ is called \emph{symmetriseable} if an invertible diagonal matrix $D$ with entries in $\mathbb Q$ exists such that
	$D^{-1} A$ is a symmetric matrix. 
\end{definition}


\begin{lemma}{\cite[1.1.5]{Kumar2002}}
	Let $A$ be a symmetrisable generalised Cartan matrix of size $N$.
	There exists a unique invertible diagonal matrix $D= \text{diag}(e_1,\dotsc, e_N)$ such that
	\begin{enumerate}
		\item $D^{-1} A$ is symmetric
		\item every entry of $D$ is a non-negative integer,
		\item Ifx $D'=(e_1',\dotsc, e_N') $ is another matrix satisfying the first two conditions 
		then $e_i\leq e_i'$ for all $1 \leq i \leq N$.
	\end{enumerate}
	We call such a matrix the \emph{minimal diagonal} matrix associated to the symmetrisable Cartan matrix $A$.
\end{lemma}

\begin{definition}{\cite[p. 4 ]{Kumar2002}}
	Let $A:= (a_{ij})_{1 \leq i,j \leq N}$ be a generalised Cartan matrix of size $N$. It is called \emph{indecomposable} if no partition of $\{1,\dotsc N\}$ into two disjoint subsets $I_1$ and $I_2$ exists such that
	$ a_{ij} = 0 = a_{ji} $ whenever $i \in I_1$ and $j\in I_2$.
	Otherwise $A$ is called \emph{decomposable}.
\end{definition}

For the following definition we need a good source(!). Maybe \cite[Chapter 4]{Kac1990}.
\begin{definition}
	We say that $A$ is of finite type if $D^{-1}A$ is positive definite.
	In this case we call $A$ an \emph{Cartan matrix}.
\end{definition}

\begin{lemma}
	If the generalised Cartan matrix $A$ of size $N$ is of finite type the symmetric matrix $D^{-1}A$ induces a inner product on the space $\mathbb R^N$.
\end{lemma}

Let us recall the definition of an abstract root system $\Delta$.
\begin{definition}
	Let $E$ be a finite-dimensional $\mathbb R$ vector space with an inner product denoted $( \cdot | \cdot )$.
	A subset $\Delta\subset E$ is called a root system of $E$ if
	\begin{enumerate}
		\item $\Delta$ spans $E$,
		\item $\alpha \in \Delta$ implies that $\mathbb R \alpha \cap \Delta = \{ \pm \alpha \}$,
		\item For $\alpha, \beta \in \Delta$ the reflection of $\beta$ along $\alpha$ is contained in $\Delta$. In formulas this means
		$\beta - 2 \frac{(\beta, \alpha)}{(\alpha, \alpha)} \alpha \in \Delta$ and
		\item  for $\alpha,\beta \in \Delta$ we have  $2 \frac{(\beta, \alpha)}{(\alpha, \alpha)}  \in \mathbb Z$.
	\end{enumerate}
	We refer to $E$ as the abstract \emph{weight space} associated to $\Delta$. Its elements are called \emph{weights}.
\end{definition}

\begin{remark}
	Given a weight space $E$ and weights $\alpha, \beta \in E$ we write
	\begin{align}
		\langle \alpha, \beta \rangle := 2 \frac{(\beta, \alpha)}{(\alpha, \alpha)}
	\end{align}
\end{remark}

\begin{definition}
	A system of \emph{simple roots} of a root system $\Delta\subset E$ is a subset $\Phi \subset \Delta$ such that
	\begin{enumerate}
		\item $\Phi$ is a basis of  the weight space $E$ and
		\item every root  $\alpha \in \Delta$ can be written solely non-negative or non-positive integral linear combination of simple roots.
	\end{enumerate}
	The subset of roots obtainable by non-negative integral combinations of simple roots is called the set of \emph{positive roots} $\Delta^+ \subset \Delta$.
\end{definition}

\begin{definition}
	Let $\Delta \subset \mathbb R^N$ be a root system and $\Phi= \{ \alpha_1, \dotsc , \alpha_N\}$ a system of simple roots.
	The \emph{Weyl group} of $\Delta$ is the group of reflections along roots $W \subset Aut (\mathbb R^N)$.
	It is generated by the reflections $s_{\alpha_1}, \dotsc , s_{\alpha_N}$ along simple roots. 
\end{definition}

There is a correspondence between abstract root systems and Cartan matrices of finite type.

\begin{theorem}
	Let $A$ be a Cartan matrix.
	Then $D^{-1}A$ induces an inner product on $\mathbb R^N$ and the 
	standard basis $\{ e_1, \dotsc, e_N\}$ of $\mathbb R^N$ form a set of simple roots.
\end{theorem}
\begin{proof}
	TODO
\end{proof}
The root system associated to a Cartan matrix $A$ is obtained by considered the closure of the reflection of the simple roots.

\begin{theorem}
	Suppose $E$ is a weight space of dimension $N$ for the root system  $\Delta \subset E$. Given a system of simple roots $\Phi = \{ \alpha_1, \dotsc, \alpha_N\}$ we obtain a Cartan matrix $A:= (a_{ij})_{1\leq i,j \leq N}$ via
	\begin{align}
		a_{ij} = \langle \alpha_j , \alpha_i \rangle \,\qquad \qquad \text{ for } 1 \leq i,j\leq N.
	\end{align}
	Moreover any other choice of system of simple roots yields the same Cartan matrix up to permutation.
\end{theorem}
\begin{proof}
	TODO
\end{proof}

\begin{theorem}
	Suppose that $E, F$ are weight spaces of the root systems $\Delta \subset E$ and $\Delta' \subset F$.
	There exists an isomorphism of inner product spaces $f:E \rightarrow F$ such that $f(\Delta) = f(\Delta')$ if and only if 
	the Cartan matrices of $\Delta$ and $\Delta'$ agree up to permutation.
\end{theorem}
\begin{proof}
	TODO
\end{proof}

By the previous theorems the geometry of the weight space is governed by its Cartan matrix.
Let us consider some weights with special properties

\begin{definition}
	Let $E$ be the weight space of a root system $\Delta$ and $\Phi$ a system of simple roots.
	Suppose $\beta \in E$ to be a weight. 
	We call $\beta$
	\begin{enumerate}
		\item \emph{integral} if $\langle \beta,\alpha \rangle  \in \mathbb Z$ for every root $\alpha \in \Delta$.
		\item \emph{fundamental} if $2\langle \beta, \alpha \rangle$ is $1$ for exactly one simple root $\alpha_i \in \Phi$ and $0$ otherwise. 
	\end{enumerate}
	The set $\Omega = \{ \omega_1, \dotsc \omega_N\}$ of fundamental weights forms another basis of $E$. In particular they generate the lattice of integral weights.
	
	We write  $\rho := \omega_1+ \dotsc \omega_N$ for the sum of fundamental weights.
\end{definition} 

A choice of positive roots gives us a nice partial order on the weight space.

\begin{definition}
	Let $\Phi\subset  \Delta$ be a sytem of simple roots of a  root system $\Delta$ of the weight space $E$.
	Given two weights $\beta, \gamma \in E$ we say
	$\beta \leq \gamma$ if $\gamma-\beta$ can be written as a positive integral combination of simple roots
\end{definition}


\subsection{Nichols algebras of Cartan type}

We fix a complex $N$-dimensional vector space $V$, together with an ordered basis $\{x_1, \dotsc, x_N\}$  of $V$ an invertible matrix $\mathfrak q := (q_{ij})_{1 \leq i,j \leq N} \in \text{GL}(V)$, called the \emph{matrix of the braiding} whose entries lie on the unit circle $S_1 \subset \mathbb C$.

For every entry $q_{ij}$ of $\mathfrak q$ exists a unique number $r_{ij}\in [0,1)$ such that $q_{ij}= e^{2\pi i r_{ij}}$. 

\begin{definition}
	We call the matrix $\mathfrak h$ whose entries are
	\begin{align}
		h_{ij} =  e^{\pi i r_{ij}}.
	\end{align}
	the \emph{root of the braiding matrix}. 
\end{definition}

\begin{definition}
	A realisation of $\mathfrak q$ is a triple $(\Gamma, (K_i)_{1\leq i \leq N}, (\chi_i)_{1 \leq i \leq N})$ comprising an abelian group $\Gamma$, a collection of elements $(K_i)_{1 \leq i \leq N} \in \Gamma$ and a collection  $(\chi_i)_{1 \leq i \leq N} \in \widehat \Gamma$ of characters of $\Gamma$ such that
	\begin{align}
		q_{ij} = \chi_j(K_i), \qquad \qquad \text{ for } 1 \leq i,j \leq N.
	\end{align}
\end{definition}

\begin{remark}
	Any realisation $(\Gamma, (K_i)_{1\leq i \leq N}, (\chi_i)_{1 \leq i \leq N})$ of $\mathfrak q$ gives $V$ the structure of a left-left Yetter-Drinfeld module over $\Gamma$ by letting
	$$
		V= \oplus_{1 \leq i \leq N} kx_i,
	$$
	where $kx_i$ denotes the one-dimensional $\Gamma$-Yetter-Drinfeld module whose action is given by $\chi_i$ and whose coaction is given by $K_i$.
\end{remark}

In the following we want to work with a particularly nice realisation which allows us to use Cartan matrices to describe the representation theory of the given group.
\begin{definition}
	The \emph{canonical representation theoretic} realisation of $\mathfrak q$ is the triple $(\mathbb Z^N,(K_i)_{1\leq i \leq N}, (\chi_i)_{1 \leq i \leq N})$, where
	the $K_i$ for the standard basis of $\mathbb Z^N$ and the $\chi_i$ are uniquely determined by
	\begin{align}
		h_{ij} = \chi_j(K_i), \qquad \qquad \text{ for } 1 \leq i,j \leq N.
	\end{align}
\end{definition}

Before we continue we wish to make our lives easier and impose further conditions on the matrix of the braiding $\mathfrak q$.

\begin{enumerate}
	\item We assume that $\mathfrak q$ is in block-diagonal form. Where each block is indecomposable (To be defined in analogy to the definition for Cartan algebras.)
	\item The matrix $\mathfrak q$ is symmetric.
	By \cite[Prop. 3.9]{Andruskiewitsch2002} we might restrict ourselves to this setting.(Every matrix of a braiding is symmetric up to twist-equivalence.)
	\item We assume that $\mathfrak q$ is of Cartan type. 
	That is there exists a generalised Cartan matrix $A = (a_{ij})_{1 \leq i,j \leq N}$ such that
	$$
		q_{ij}q_{ji}=q_{ii}^{a_{ij}}
	$$ 
	\item The diagonal entries of $\mathfrak q$ are different from $1$.
\end{enumerate}

Note that by no means the GCM $A$ can be assumed to be unique. Rather we fix a choice of $A$ by the conditions that for $i\neq j$
\begin{align}
	-N_i \leq a_{ij}\leq 0  \text{ where } N_i = \text{ord} \; (q_{ii})
\end{align}

\begin{remark}
	Let us stress that under these assumptions we have the following identity:
	\begin{align}
		q_{ij}= \pm h_{ii}^{a_{ij}}
	\end{align}
	It is not clear whether we can choose the $h_{ii}$ in such a fashion that all signs are positive.
	However every matrix of a braiding is twist-equivalent to one where all signs can be chosen positive. Thus we will restrict ourselves to this case.
\end{remark}

From \cite[Theorem 2.16]{Andruskiewitsch2017} we know that 
the generalised Cartan matrix $A$ is of finite type if and only if the Nichols algebra $\Nichols V$ of $V$ is finite-dimensional.

We want to restrict ourselves to this setting and assume $A$ to be a proper Cartan matrix in the following 
\footnote{Probably its fine if the Cartan matrix is affine (i.e. arithmetic root system in the language of Nichols algebras). What we really want and need is not a inner product but the non-degenerate bilinear form $\langle \cdot, \cdot \rangle $, which also exists for the affine type, see\cite{Kumar2002}}.

Before we enter the representation-theoretic realm of this work we need to introduce the Drinfeld double of a Nichols algebra as defined in \cite{Andruskiewitsch2010}.

\subsection{A version of the Drinfeld double and Verma modules}

\begin{definition}{{\cite[Definition 3.1]{Andruskiewitsch2010}}}
	A \emph{reduced YD datum} of size $N$ is a quadruple $ D:= (\Gamma, (K_i)_{1\leq i \leq N}, (L_i)_{1\leq i \leq N}, (\xi_i)_{1\leq i \leq N} )$ comprising a group $\Gamma$, two collections of elements $K_i, L_i\in \Gamma$ and a collection of characteres $\xi_i \in \widehat \Gamma$ such that
	\begin{align}
	K_i L_i \neq 1, \qquad \qquad
	\xi_j(K_i) = \xi_i(L_j)
	\end{align}
	We call $D$ symmetric if $\xi_i(K_j)= \xi_j(K_i)$ for all $1\leq i,j \leq N$.
\end{definition}

Note that we ignore the \emph{linking parameter}.

\begin{remark}
	Let $D = (\Gamma, (K_i)_{1\leq i \leq N}, (L_i)_{1\leq i \leq N}, (\xi_i)_{1\leq i \leq N})$ be a reduced YD datum.
	We obtain two $\Gamma$ Yetter-Drinfeld modules by
	\begin{align}
	V &:= \oplus_{1\leq i \leq N} kx_i, \qquad kx_i = V_{K_i}^{\xi_i} \\
	W &:= \oplus_{1\leq i \leq N} ky_i, \qquad ky_i = V_{L_i}^{\xi_i^{-1}} 
	\end{align} 
\end{remark}

\begin{definition}{{\cite[Definition 3.3]{Andruskiewitsch2010}}}
	The Drinfeld $D(V)$ double associated to a reduced YD datum $D = (\Gamma, (K_i)_{1\leq i \leq N}, (L_i)_{1\leq i \leq N}, (\xi_i)_{1\leq i \leq N})$ is the quotient of 
	$T(V \oplus W) \# k \Gamma$ by the two-sided ideal coideal spanned by
	\begin{enumerate}
		\item the relations of the Nichols algebra of $V$,
		\item the relations of the Nichols algebra of $W$,
		\item $x_i y_j - \xi_j^{-1}(K_i) y_j x_i = \delta_{i=j} (K_iL_i -1)$.
	\end{enumerate}
\end{definition}


\begin{lemma}
	Given our pair $(V,\mathfrak q)$ considered in the beginning we can build a symmetric YD datum by letting 
	$\Gamma := \mathbb Z^N$ be the free abelian group of dimension $N$,
	$K_1,\dotsc K_N$ the standard basis of $\mathbb Z^N$, $L_i =K_i$ for $1 \leq i \leq N$ and $\chi_i= \xi_i^2$ where
	$\xi_j(K_i) = h_{ij}= h_{ji} = \xi_i(L_j)$.
\end{lemma}

\begin{convention}
	From now onwards we  write $U:=D(V)$ for the Drinfeld double associated to this particular YD datum.
\end{convention}

\begin{theorem}
	The Drinfeld double $U$ has a triangular decomposition. That is,
	the multiplication map
	$$
	\Nichols W \otimes k \Gamma \otimes \Nichols V \rightarrow D(W),
	w\otimes g \otimes v \mapsto wgv
	$$
	is bijective.
	
	We write $U_0 := k \Gamma$ , $U^+ = \Nichols V$, $U^- := \Nichols W$ and $U_{\geq 0} = \Nichols V \# k \Gamma$ to stress the similarities between the Drinfeld double and the universal enveloping algebras.
\end{theorem}

We will need the following relation:
\begin{theorem}{{\cite[3.2]{Andruskiewitsch2010}}}
	Let $D$ be our fixed YD datum with underlying group $\Gamma$, $V, W$ the Yetter-Drinfeld modules associated to $D$ with the bases $x_1,\dotsc, x_N$ and $y_1, \dotsc, y_N$ as before. 
	Write
	$$
	E_i = x_i ,\dotsc 
	F_i = y_iL_i^{-1}.
	$$
	Then $U^+$ is generated by the $E_i$'s, $U^-$ is generated by the $F_i$'s and 
	\begin{align}
		g F_i &= \xi_i^{-2}(g) F_i g \text{ for all } g \in \Gamma
		\label{eq: grpExchange}
		\\
		E_i F_j - F_j E_i &= \delta_{i=j} (K_i - L_i^{-1})
		\label{eq: EF-exchange}
	\end{align}
\end{theorem}

\begin{lemma}{{\cite[4.3]{Joseph1995}}} \label{lem: ItteratedEFExchange}
	For $n>0$ we have
	\begin{align}
	E_i F_i^n - F_i^n E_i = [n]_{h_{ii}} F_i^{n-1} \left( h_{ii}^{-(n-1)} K_i- h_{ii}^{n-1} L_i^{-1} \right), 
	\end{align} 
	where $[n]_h:= \frac{h^n-h^{-n}}{h-h^{-1}}$.
\end{lemma}
Note that for each $1\leq i\leq N$ we have 
$\text{ord} h_{ii} \geq 2$ thus 
$[n]_{h_{ii}}$ is well-defined.
\begin{proof}	
	First let us recall that a q-natural is defined as
	$$
	(n)_q := q^{0} + q^1 + \dotsc +q^{n-1} = \frac{q^{n} - 1}{q -1} \text{ for } n>0
	$$
	
	Now using a telescope sum argument and the relation \eqref{eq: EF-exchange} we see that
	\begin{align*}
	E_i F_i^n - F_i^n E_i &= \sum_{s=0}^{n-1} F_i^{s} (E_i F_i - F_iE_i)F_i^{n-1-s} =
	\sum_{s=0}^{n-1} F_i^{s} (K_i- L_i^{-1})F_i^{n-1-s} \\
	& = F_i^{n-1} \sum_{s=0}^{n-1} \xi_i^{-2(n-1-s)}(K_i) K_i - 
	\xi_i^{-2(n-1-s)}(L_i^{-1}) L_i^{-1}  \\
	& =  F_i^{n-1} \sum_{s=0}^{n-1} h_{ii}^{-2s} K_i - h_{ii}^{2s} L_i^{-1}
	= F_i^{n-1} \left ((n)_{h_{ii}^{-2}} K_i- (n)_{h_{ii}^2} L_i^{-1} \right)\\
	& = [n]_{h_{ii}} F_i^{n-1} \left( h_{ii}^{-(n-1)} K_i- h_{ii}^{n-1} L_i^{-1} \right).
	\end{align*}
	In the last line we used that
	$$
	\frac{h_{ii}^n-h_{ii}^{-n}}{h_{ii}-h_{ii}^{-1}}h_{ii}^{-(n-1)} =  
	\frac{h_{ii}- h_{ii}^{-2n +1}}{h_{ii} - h_{ii}^{-1}} = 
	\frac{h_{ii}^{-2n} - 1}{h_{ii}^{-2} - 1}
	$$	
	and
	$$
	\frac{h_{ii}^n-h_{ii}^{-n}}{h_{ii}-h_{ii}^{-1}}h_{ii}^{(n-1)} =  
	\frac{h_{ii}^{2n -1}- h_{ii}^{-1}}{h_{ii} - h_{ii}^{-1}} = 
	\frac{h_{ii}^{2n} - 1}{h_{ii}^{2} - 1}.
	$$
\end{proof}

\subsection{Representation theory of Nichols algebras by geometric means}

\begin{definition}
	Let $D(V)$ be the Drinfeld double of $V$ and $\gamma$ a character of $\Gamma$.
	The \emph{Verma module of weight $\gamma$} is the induced module 
	\begin{align}
		M(\gamma) := D(V) \otimes_{U_{\geq 0}} k_\gamma,
	\end{align}
	where the action of $U_{\geq 0}$ on $k$ is given by 
	$$
		g\otimes v \triangleright 1 = \epsilon(v)\gamma(g).
	$$
	
	A vector $0 \neq v \in M(\gamma)$ is called highest weight vector if
	$x \cdot v =0 $ for every $x \in U^+$.
\end{definition}


Here something `weird' happens which feels somewhat analogous to the situation in Galois - theory. 
Verma modules are generated by essentially unique highest weight vectors.
However given a Verma module $M$ we find n isomorphic submodule $M' \subset M$ such that $M/M' \neq \{0\}$. 
How does this fit with the claim that the generating highest weight vector is unique?
Simply put the isomorphism
$M' \rightarrow M$ does not extend to an isomorphism $M \rightarrow M$, i.e. is not contained in the "Galois group". This is something we might want to look into.

\begin{lemma}
	Every Verma module $V(\gamma)$ of $U$  is generate by a highest weight vector $\nu(\gamma)$ and this vector is unique up to scalar multiplication.
\end{lemma}
\begin{proof}
	Generation is clear.
	To see uniqueness suppose that $\nu$ and $\nu'$ are highest weight vectors of weight $\gamma$.
	Now using the fact that we have a triangular composition we observe that there are elements $wgv, w'g'v' \in U$ with $w,w' \in U^-$, $g,g' \in U_0$ and $v,v' \in U^+$ such that
	$$
		wgv \triangleright \nu = \nu' \text{ and } w'g'v' \triangleright \nu' = \nu.
	$$
	As $\nu$ and $\nu'$ are highest weight vectors we have $v=v'=1$
	It follows from \eqref{eq: grpExchange} that there exists a scalar $\lambda_g'$ such that
	$\nu= w'g'wg \nu = \lambda{g'} w'w gg'\nu = \lambda{g'} \gamma(gg') w'w \nu $.
	Since $V(\gamma)$ is free as a $U^-$ module we observe that $w'w = \lambda'1$ for some scalar $\lambda'$.
	As $U^-$ is a graded, connected algebra $w$ and $w'$ needed to be scalar multiples of the unit.
	Thus we have a non-zero scalar $\lambda''$ such that 
	$\nu= \lambda'' \nu'$.
\end{proof}

\begin{theorem}[Universal property of Verma modules]
	Let $M$ be some module over $U$. Given a weight $\gamma$ there exists a non-zero morphism $\psi: V(\gamma) \rightarrow M$ if and only if $M$ contains a highest weight vector $m$ of weight $\gamma$.
	Moreover $\psi$ is uniquely determined by mapping the highest weight vector of $\nu(\gamma) \in V(\gamma)$ to $m$. 
\end{theorem}
\begin{proof}
	If such a vector exists its a direct calculation to show that $\psi$ exists and that it is uniquely determined by the image of $\nu(\gamma)$.
	
	Conversely assume that such a morphism of modules $\psi: V(\gamma)\rightarrow M$ is given.
	The image of $m := \psi(\nu(\gamma))$ needs to be a heighest weight vector of some weight $\gamma'$ or zero.
	If it were zero $\psi$ would be zero. Thus $0 \neq m$.
	Now assume $g\in U_0$ to be a group-like element. Then
	$\gamma'(g) m =  g \triangleright m = \psi( g \triangleright \nu(\gamma))  = \gamma(g) m$.
	Thus $\gamma' = \gamma$.
\end{proof}


\begin{lemma}\label{lem: HigWeightInVermMods}
	Let $V(\gamma)$ be a Verma module of weight $\gamma$ and $\nu$ a highest weight vector of weight $\gamma$.
	Then $F_i^n \nu$ is a highest weight vector of weight 
	$
		\xi_i^{-2n}\gamma
	$
	if and only if
	$
		\gamma(K_i) = \pm h_{ii}^{(n-1)}
	$ or $n=0 \mod \text{ord} (h_{ii})/2$ .
\end{lemma}
\begin{proof}	
	First we compute the weight of $F_i^n \nu$.
	Let $g \in U_0$ be a group-like element. We have
	$$
		g F_i^n \nu = \xi_i^{-2n}(g) F_i^n  g \nu  = (\xi_i^{-2n}\gamma)(g) F_i^n  \nu.
	$$
	
	
	Next we have to check that for every $1\leq j\leq N$ the identity
	$
		E_jF_i^n \nu = 0
	$ holds.
	If $i\neq j$ this holds trivially.
	Otherwise we compute 
	\begin{align*}
		E_iF_i^n \nu 
		& = 
		([n]_{h_{ii}}F_i^{n-1}(h_{ii}^{-(n-1)} K_i - h_{ii}^{n-1}(L_i)^{-1}) + F_i^n E_i) 
		\nu \\
		& = [n]_{h_{ii}}F_i^{n-1}(h_{ii}^{-(n-1)} K_i - h_{ii}^{n-1}(L_i)^{-1}) \nu \\
		& = \left( 
		h_{ii}^{-(n-1)} \gamma (K_i) - h_{ii}^{(n-1)}\gamma(L_i^{-1}) \right)	[n]_{h_{ii}}F_i^{n-1} \nu  \overset{!}{=} 0
	\end{align*}
	There are two cases which might occur:
	\begin{enumerate}
		\item $[n]_{h_{ii}}=0$ This is the case if and only if 
		$h_{ii}^n =  h_{ii}^{-n}$ which holds if and only if $n=0 \mod \text{ord} (h_{ii})/2$.
		\item 	$
		\left( 
		h_{ii}^{-(n-1)} \gamma (K_i) - h_{ii}^{(n-1)}\gamma(L_i^{-1}) \right)
		\overset{!}{=} 0.
		$
		Therefore
		$$
			\gamma(K_iL_i) = \left( h_{ii}^{(n-1)}\right) ^2. 
		$$
		In particular if $K_i = L_i$ we have
		$$
			\gamma(K_i) = \pm h_{ii}^{(n-1)}
		$$
	\end{enumerate}	
\end{proof}


Next we define a abstract weight space. The idea is that elements in this space are supposed to govern the (real) powers of our fixed characters $\xi_i$. 

\begin{definition}
	The \emph{abstract weight space} of $U$ is $\mathbb R^N$.
	The simple roots correspond to the standard basis $\alpha_1, \dotsc \alpha_N$. 
	The root system $\Delta$ of $U$ is the root system obtained by the Cartan matrix $A$ and the choice of simple roots.
\end{definition}


\begin{theorem}
	Given any vector $w\in E$ we define the character
	\begin{align}
		\xi_w: \mathbb Z^N \rightarrow S_1, \qquad  \xi_w(K_i) = h_{ii}^{ \langle w, \alpha_i \rangle}.
	\end{align}
	Conversely given any character $\xi$ there exists at least one vector $w' \in E$ such that $\xi_{w'} = \xi$.
\end{theorem}
\begin{proof}
	The first part of the theorem is shown by a direct calculation.
	To proof the second part observe that $\xi$ is uniquely determined by its values on the $K_i$.
	We thus have to construct the vector $w'$. Our approach resembles Grahm-Schmidt to some degree.
	
	We begin by choosing $w_1$ such that $\xi(K_1) = h_{11}^{\langle w_1, \alpha_1 \rangle }$.
	Next let $E'$ be the orthogonal complement of $\mathbb R w_1$.
	Since the $\alpha_i$ form a basis of $E$ and the bilinear form is non-degenerate there needs to be a vector $w_2$ in $E'$ such that
	$\xi(K_2) = h_{22}^{\langle w_1 + w_2, \alpha_2 \rangle }$.
	Set $w_2' := w_1+ w_2$.
	Continue this process until the desired vector $w'$ is obtained.
\end{proof}


\begin{lemma}\label{lem: EqualWeights}
	Let $w, w' \in E$ be weight vectors.
	Ten $\xi_w = \xi_{w'}$ if and only if for all $1 \leq i \leq N$ the identity 
	\begin{align}
		\langle w- w', \alpha_i \rangle  \in \text{ord}(h_{ii}) \mathbb Z
	\end{align}
	holds.
\end{lemma}
\begin{proof}
	Suppose the identity holds and fix any $1 \leq i \leq N$. Then
	there exists an $m \in \mathbb Z $ such that $	\langle w- w', \alpha_i \rangle = m\;\text{ord}(h_{ii})\alpha_i$ and 
	$$
	\xi_{w}(K_i) = 
	h_{ii}^{\langle w, \alpha_i \rangle } =
	h_{ii}^{\langle w- w' + w', \alpha_i \rangle } =
	h_{ii}^{m \;\text{ord}(h_{ii}) + \langle w', \alpha_i \rangle } =
	h_{ii}^{\langle w' , \alpha_i \rangle } =
	\xi_{w'}(K_i).
	$$
	
	Assume conversely that $\xi_w = \xi_{w'}$.
	In this case we have for any $1\leq i \leq N$ that
	$$
	\xi_w(K_i) \xi_{w'}(K_i)^{-1} = 
	h_{ii}^{\langle w, \alpha_i \rangle - \langle w', \alpha_i \rangle } =
	h_{ii}^{\langle w - w', \alpha_i \rangle } =		
	1	
	$$
	This is equivalent to 
	$ \langle w - w', \alpha_i \rangle \in \text{ord}(h_{ii}) \mathbb Z$.
\end{proof}


\begin{remark}
	The numbers $h_{ii}$ are not transcendental. Thus the vectors $w'$ are not unique. 
	In this setting we have a projection from the (abstract) weight space $E$ onto the simple representations of $\mathbb Z^N$. 
	It might be the case that, by considering  $\mathbb Z^N$-graded modules, we obtain a bijection.
\end{remark}

\begin{lemma}\label{lem: Fis-weights}
	Let $M$ be a $U$ module and $m\in M$ a weight vector whose abstract weight is represented by a vector $w \in E$.
	Then
	$F_i^n m$ has the weight $w - n\alpha_i$.
\end{lemma}
\begin{proof}
	Let $\xi_w$ be the character assigned to $w$.
	We compute 
	$$
		K_j F_i^n m = \xi_i^{-2n}(K_j) F_i^n K_j m 
		= \xi_j^{-2n}(K_i) \xi_w(K_j) F_i^n m 
		=  h_{jj}^{-n a_{ji} } h_{jj}^{\langle w, \alpha_j \rangle }F_i^n m.
	$$
	Conversely we have
	$$
		\xi_{w-n\alpha_i}(K_j)
		= h_{jj}^{\langle w -n\alpha_i, \alpha_j \rangle} 
		= h_{jj}^{\langle w , \alpha_j \rangle
			-n\langle \alpha_i, \alpha_j \rangle}
		=  h_{jj}^{-n a_{ji}} h_{jj}^{\langle w, \alpha_j \rangle }
	$$
\end{proof}

\begin{definition}
	The affine action of $W$ on the abstract weight space $\mathbb R^N$ is defined by
	$$
	w\cdot \lambda = w(\lambda+\delta)-\delta
	$$
	where $\delta := \omega_1+\dotsc+ \omega_n$ is the sum of fundamental weights.
\end{definition}

\begin{lemma}\label{lem: weightsAffACtion}
	Let $s_i\in W$ be a simple reflection and $\gamma \in \mathbb R^N$ any weight.
	Then 
	$$ 
	\gamma- s_i \cdot \gamma = n \alpha_i, \text{ with } n :=  \langle \gamma, \alpha_i \rangle + 1
	$$ 
	where $\alpha_i$ is the simple root corresponding to the reflection $s_i$.
	If $\gamma$ is integral either 
	$\gamma \leq s_i\cdot\gamma$ or $s_i \cdot \gamma \leq \gamma$.
\end{lemma}
\begin{proof}
	We compute
	\begin{align*}
		\gamma- s_i \cdot \gamma 
		& = \gamma -( s_i (\gamma+ \delta) -\delta) 
		= \gamma + \delta - s_i(\gamma +\delta)
		\\
		&= \gamma + \delta - (\gamma +\delta) + \alpha_i \langle \gamma + \delta, \alpha_i \rangle 
		= \alpha_i \langle \gamma+ \delta,\alpha_i \rangle
		\\
		&= \alpha_i 
		\left( \langle \gamma, \alpha_i \rangle + \langle \delta, \alpha_i \rangle \right)
		= \alpha_i 
		\left( \langle \gamma, \alpha_i \rangle + 1 \right)
		= n \alpha_i.
	\end{align*}
	In the last step we used that $\delta$ is the sum of fundamental weights.
\end{proof}


\begin{theorem}[{\cite[4.4.7]{Joseph1995}}]
	Let $\gamma \in \mathbb R^N$ be a positive integral weight and $w,v\in W$.
	Then there exists a (necessarily injective) morphism between Verma modules
	$$
		V(v\cdot \gamma) \rightarrow V(w\cdot\gamma)
	$$
	if (and only if ?) $v \geq w$.
\end{theorem}
\begin{proof}
	The proof works by an induction over the length $l(v)$ of $v$.
	If $l(v) = 0$ we have $w =v$ and nothing needs to be shown.
	Assume $l(v) \geq 1$. There exists a simple root $\alpha$ such that 
	$s_\alpha v < v$. 
	ASSUMPTION: BRUHAT YOGA IMPLIES THAT $v \cdot \gamma < s_\alpha v \cdot \gamma$.
	
	Define $ n-1:= \langle v\cdot \lambda, \alpha \rangle$ and write $\nu$ for a highest weight vector generating $V(s_\alpha v \cdot \gamma)$.
	By Lemma \ref{lem: weightsAffACtion} we have
	$v\cdot \gamma - s_\alpha v \cdot \gamma = n \alpha_i$
	As $v \cdot \gamma < s_\alpha v \cdot \gamma$ we have $n<0$.
	Using the fact that $U^-$ is a domain we set $ 0 \neq \tilde \nu := F_i^{-n} \nu$.
	As a consequence of Lemma \ref{lem: Fis-weights} the weight of $\tilde\nu$ is 
	$s_\alpha v \cdot \gamma + n\alpha_i = v\cdot \gamma$.
	Moreover it follows from Lemma \ref{lem: HigWeightInVermMods} that 
	$\tilde \nu $ is a highest weight vector. The universal property of Verma modules now implies that the required injective morphism
	$$
			V(v\cdot \gamma) \rightarrow V(s_\alpha v\cdot\gamma)	
	$$
	exists.
	TODO:MORE BRUHAT YOGA. Either $s_\alpha v >w$ or $s_\alpha v \geq s_\alpha w$ and $s_\alpha w \geq w$.
	If the first case holds we're done by the induction hypothesis.
	Thus let $s_\alpha v \geq s_\alpha w$ be true. Again the induction hypothesis grants us the existence of a morphism
	$\psi:M(s_\alpha v\cdot \gamma )\rightarrow M(s_\alpha w\cdot \gamma)$
	and a injective morphism $\iota: M(w \cdot \gamma) \rightarrow M(s_\alpha w\cdot \gamma)$.
	We identify $M(s_\alpha w\cdot \gamma)$ with its image in $M(s_\alpha w\cdot \gamma)$ under $\iota$.
	Let $M := M(s_\alpha w\cdot \gamma)/ M(w\cdot \gamma)$.
	We write $\mu := \psi(\tilde\nu)$ and $[\mu]$ for its equivalence class in $M$. 
	We claim that $M$ is locally finite under the action $F_i$, see \cite[4.3.5]{Joseph1995} \footnote{
	This means that for every $[X] \in M$ there is a non-negative integer $r$ such that $[F_i^r X] = 0$. 
	}.
	Therefore it somehow follows (?) that an  $r\in \mathbb N$ exists such that $[F_i^r \mu] =0$.
	Now different cases might occur.
	\begin{enumerate}
		\item $r = 0 \mod \text{ord}(h_{ii})/2$ or
		\item $r \neq 0 \mod \text{ord}(h_{ii})/2$.
	\end{enumerate}
	Observe that by Lemma \ref{lem: Fis-weights} and \ref{lem: EqualWeights} we have that $\mu':= F_i^{m \text{ord}(h_{ii})} \mu$ has the same weight as $\mu$.
	Moreover by Lemma \ref{lem: HigWeightInVermMods} it is a highest weight vector.
	Thus choose $m\in \mathbb N$ minimal such that $[F_i^{m \text{ord}(h_{ii})} \mu] = 0$. 
\end{proof}

\begin{remark}
	Some questions:
	Can we find the "biggest" copy of $M(v \cdot\gamma)$ in $M(w \cdot \gamma)$ ?
	How do vectors of weight zero enter the picture/ what is their role?
	
	Does the above theorem hold in the graded setting. This means that we need to be able to show that $r$ can be choosen smaller than $\text{ord } h_{ii}/2$.
\end{remark}



\begin{example}
	We show with a minimal counterexample that the above claims do not work in the case of Nichols algebras.	
	Fix the matrix 
	$\mathfrak q:= 
	\begin{bmatrix}
		-1 & i \\
		i & -1
	\end{bmatrix}
	$
	Its generalised Dynkin-diagram is
	\begin{equation}
		\begin{tikzcd}[arrows=-]
		 -1 \arrow[r, "-1"] &
		 -1
		\end{tikzcd}
	\end{equation}
	The associated Nichols algebra has two generators $F_1, F_2$ and the 
	PBW-basis $\{F_2^{j_2} F_{12}^{j_{12}} F_1^{j_1} \mid 0 \leq j_1, j_{12}, j_2 \leq 2\}$.
	Accordingly the set of roots is $\{\alpha_1, \alpha 1+ \alpha 2, \alpha 2\}$ and the fundamental weight is $\alpha_1 + \alpha_2$.
	Its Weyl group is $\{e, s_1, s_2, s_1 s_2, s_2 s_1, s_1 s_2 s_1, s_2 s_1 s_2, s_1 s_2 s_1 s_2\}$.
	Fix the weight $0$  and the elements of the Weyl group 
	$v := s_1 s_2 > s_1 =: w$.
	We compute
	\begin{gather}
		w \cdot 0 = - \alpha_1, \qquad \qquad s_2 \cdot 0 = - \alpha_2, \qquad v \cdot 0 = s_1 \cdot (- \alpha_2) = -2 \alpha_1 - \alpha_2.
	\end{gather}
	Let us write $x_w$, $x_v$ and $x_{s_2}$ for the generating highest weight vectors of the Verma modules $V(w\cdot 0)$, $V(v\cdot 0 )$ and $V(s_2 \cdot 0)$ respectively.
	We have $s_1v = s_2 < v$ and $v\cdot 0 -s_2 \cdot 0 = -2 \alpha_1$.
	Now we claim that $F_1 x_{s_2}$ is the highest weight vector implementing $V(v\cdot 0) \rightarrow V(s_1 \cdot 0)$.
	Note that the above map is not injective!
	
	Now we want to find a map going from $V(s_1 \cdot 0)$ to $V(s_1 w \cdot 0) = V(0)$.
	The highest weight vector is $F_1x_0$ 
\end{example}
\begin{enumerate}
	\item Rough strategy: Starting with an integrable weight $\lambda$ we obtain a sequence of inclusions of Verma modules which is esentially unique up to scalar multiplication. 
	\item This would allow us to build a BGG-styled complex.
\end{enumerate}




% 
%-------- Check what we need from here onwards ---
%
%
%\begin{enumerate}
%	\item A complex $N$-dimensional vector space $V$,
%	\item an ordered basis $\{x_1, \dotsc, x_N\}$  of $V$ a
%	\item generalised Cartan matrix $A := (a_{ij})_{1\leq i,j \leq N}$ and
%	\item a symmetric invertible matrix $\mathfrak q := (q_{ij})_{1 \leq i,j \leq N} \in \text{GL}(V)$, such that
%	\begin{align}\label{eq: Cartan type}
%		q_{ij}q_{ji} = q_{ii}^{a_{ij}} \quad \text{ and } \quad
%		a_{ij} = \max_{n\in \mathbb Z_{\leq 0}}\{ q_{ij}q_{ji} = q_{ii}^{n} \} \text{ if } i \neq j
% 	\end{align}	
%	holds for $1 \leq i,j \leq N$ and the Nichols algebra $\Nichols V$ of the pair $(V, \mathfrak q)$ is finte-dimensional.	
%\end{enumerate}
%
%	The Nichols algebra of $V$ is said to be of \emph{Cartan type}, see \cite[Definition 2.15]{Andruskiewitsch2017}.
%
%\begin{remark}
%	We should be able to see that $A$ is necessarily symmetrisable. At least this is assumed/noted in \cite{Andruskiewitsch2017}.
%\end{remark}
%
%By Theorem 2.16 of \cite{Andruskiewitsch2017} we have
%
%\begin{theorem}
%	The generalised Cartan matrix $A$ is of finite type.
%\end{theorem}
%
%Now that we understand what type of Cartan matrix we have to consider we can define a suiting  realisation of the Nichols algebra of $V$.
%We need some preparatory work before we can do so.
%\begin{remark}
%	Let $B_1, \dotsc, B_N$ be the canonical basis of $\mathbb Z^N$.
%	There exists a system of characters $\chi_1,\dotsc \chi_N$ of $\mathbb Z^N$ such that
%	\begin{align}
%		\begin{aligned}
%		\chi_i (B_j) &= \delta_{ij} \\
%		2 \text{ord}\; (q_{ii}) &\mid \text{ord}\; (\chi_i).
%		\end{aligned}
%	\end{align}
%\end{remark} 
%
%\begin{definition} 
%	The \emph{canonical realisation} of $V$ is the subgroup $H$ of $\mathbb Z^N$ spanned by the elements $K_1, \dotsc K_N$ which are uniquely determined by the condition that
%	\begin{align}
%		q_{ij} &= \chi_{j}(K_i^2) 
%	\end{align}  
%	and 
%	$0 \leq K_{ij} \leq  \text{ord}\; (\chi_j)$  where $K_{ij}$ denotes the $j$-th entry of $K_i$.
%\end{definition}
%
%\begin{remark}
%	We obtain a $H$-Yetter-Drinfeld module on $V$ by letting
%	$x_i$ span the one-dimensional submodule whose action is implemented by $\chi_i$ and whoses coaction is given by $K_i$.  
%\end{remark}
%
%
%
%We want to set up a root system in a way that it governs the representation theory of $H$ but is also compatible with the root system of the Nichols algebra $\Nichols V$ of $V$.
%
%Let $\Delta^+ \subset \mathbb R^N$ be the generalised root system of $\Nichols V$.
%We equip $\mathbb R^N$ with the inner product $(\cdot | \cdot )$ induced by $S = D^{-1} A$.
%
%
%\begin{definition}
%	 Let $\mathfrak h^*$ be the dual inner space of $(\mathbb R^N, (\cdot | \cdot ))$. The \emph{root system} of $\Nichols V$ is the dual root system $\Phi$ of $\Delta$. 
%\end{definition}
%
%We write $\alpha_i$ for the simple roots corresponding to the generators $x_i$ of $\Nichols V$. 
%
%
%\begin{lemma}
%	Let $\xi$ be a character of $H$. Then there exists a vector $\omega \in \mathbb R^N$ such that 
%	
%	$\xi = \chi_1^{2 \frac{(\omega_1,\alpha_1)}{(\alpha_1, \alpha_1)}}\dotsc \chi_N^{2 \frac{(\omega_N,\alpha_N)}{(\alpha_N, \alpha_N)}}$
%\end{lemma}
%
%\begin{proof}
%	The vectors $\alpha_i$ span $\mathfrak h^*$. Thus such coefficients exist. They might not be unique though.
%\end{proof}
%
%Since the projection from $\mathfrak h^*$ to the characters of $H$ is surjective and the fibers are discrete we might choose for any character $\xi$ a entry-wise non-negative minimal representant $\widehat \xi$ in $\mathfrak h^*$. 
%
%
%
%----- OLD STUFF---
%
%
%	
%	
%
%	\begin{definition}
%		A reduced YD-datum consists of
%		\begin{enumerate}
%			\item an abelian group $\Gamma$,
%			\item a natural number $N>0$,
%			\item a collection of elements $K_1,\dotsc K_N \in \Gamma$ plus
%			another collection $L_1, \dotsc L_N\in \Gamma$ of the same size
%			\item and a collection of characters $\xi_1, \dotsc \xi_N$
%		\end{enumerate}
%		such that
%		\begin{enumerate}
%			\item $K_i L_I \neq 1$
%			\item $\xi_i(K_j) = \xi_j(L_i)$
%		\end{enumerate}
%	\end{definition}
%
%	We fix for the remainder of this section the group $\Gamma = \mathbb Z^N$ with the ordered standard basis $K_1, \dotsc K_n$.
%	There is a unique collection of characters $\xi_1, \dotsc \xi_N \in S_1^N$ such that
%	$$
%		q_{ij}= \xi_j(K_i).
%	$$
%	See (ARSchneider - Definition 1.4 for this convention).
%	
%	Since our matrix $\mathfrak q$ is symmetric we might extend this to a YD-Datum by fixing $L_i = K_i$.
%	
%	On $V$ we obtain a structure of a $\Gamma$- YD module by setting
%	$$
%		V = \oplus kV_{K_i}^{\xi_i} 
%	$$
%	
%	Similarly define the $\Gamma$ YD-module
%	
%	$$
%		W = \oplus kW_{L_i}^{\xi_i^{-1}}
%	$$
%	and write $y_1,\dotsc y_N$ for its canonical basis. (SEE ARSchneider Definition 3.3).
%	
%	Note that the matrix of the braiding associated to $W$ is obtained from $\mathfrak q$ by inverting every entry. This specifically means that it is implemented by the same GCM.
%	
%	The Drinfeld double $D(V)$ of $\Nichols V$ is 
%	$\Nichols W\otimes k \Gamma \otimes \Nichols V$, i.e it is the bosonisation of $T(V\oplus W)\# k \Gamma$ modulo the ideal spanned by
%	\begin{enumerate}
%		\item the generating ideal $I(V)$ of the Nichols algebra of $V$,
%		\item the generating ideal $I(W)$ of the Nichols algebra of $W$,
%		\item the relation 
%			$$
%				x_iy_j - \xi_j^{-1}K_i y_j x_i =\delta_{i=j}(K_iL_i -1)
%			$$
%	\end{enumerate}	
%
%	We define another generating set of $D(V)$, namely
%	$E_i = x_i$ and $F_i = y_iL_i^{-1}$.
%	wrt this generating set we have the relations
%	
%	\begin{align}
%%	TODO \label{eq: EF-exchange}
%		g F_i &= \xi_i^{-1}(g) F_i g \text{ for all } g \in \Gamma\\
%		E_i F_j - F_j E_i &= \delta_{i=j} (K_i - L_i^{-1})
%	\end{align} 
%	
%
%
%	Let us now set up a definition of weight space, root system and root lattice.
%	Roughly what we want to do is the following:
%	The group of characters ( i.e. simple reps ) of $\mathbb Z^N$ is isomorphic to the $N$-torus  $S_1^N$.
%	To describe any element on the torus we might choose a point $p$ a basis for the tangent space $k^N$ at $p$ and determine now any new point in $p$ as a translation in direction of the tangent vector. 
%	Whilst offering us a nice mental picture this is not necessarily a rigid mathematical principle. As a guideline it seems however helpful.
%	
%	
%	Write $\mathfrak h := \text{span}_\mathbb R \{K_1, \dotsc K_N\}$ be the $N$-dimensional $\mathbb R$-vector space whose standard basis is labelled by the group-likes of $D(V)$.
%	Now choose a basis $\{\chi_1, \dotsc \chi_N \}$ of $\mathfrak h^*$ such that
%	\begin{enumerate}
%		\item $\chi_i$ is a character
%		\item $\xi_i = \chi_1^{i_1}\dotsc \chi_N^{i_N}$ for $\i_N \in \mathbb 2Z$ and
%		\item $\chi_i(K_j) = 0$ (but maybe we can drop this condition?).
%	\end{enumerate}
%	
%	Before we continue we need to describe what it means to take the power of any character by a real number.
%	Simply put let $\rho$ be any character and $r \in \mathbb R$.
%	then $\rho^r(K_i) := (\rho(K_i))^r$.
%	Let us quickly sketch why this is still a character:
%	\begin{enumerate}
%		\item $\rho^r(0) =0$.
%		\item $\rho^r(g+h) = (\rho(g+h))^r= (\rho(g)\rho(h))^r = \rho (g) ^\rho(h)^r$.
%	\end{enumerate}	
%	With this in mind we have the following relation between $\mathfrak h^*$ and the simple representations of $\mathbb Z^N$. 
%	\begin{lemma}
%		Let $\rho$ be a simple $\mathbb Z^N$ module. Then there exists a vector $v=(v_1,\dotsc, v_N)^T \in \mathfrak h^*$ such that 
%		$\chi^{v_1}\cdot \chi^{v_N} = \rho$
%		This vector is unique if we require $0 \leq v_i \leq |\langle \chi_i\rangle|$. 
%	\end{lemma}
%	\begin{proof}
%		First note that the product of finitely many characters is still a character and that real powers of characters are a well-defined concept. Let us therefore see how we might obtain the vector $v$.
%		Fix an $1 \leq i \leq N$ and let $w:= \rho(K_i)$. 
%		Now obviously we have $w:= e^{\frac{2 \pi i }{r}}$ with $r\in \mathbb R$.
%		Since $\chi_i(K_i) = e^{\frac{2 \pi i}{m}}$ simply choose $v_i := rm^{-1}$.
%	\end{proof}
%	Now let us define  $\Delta ^+ \subset \mathfrak h^*$ as 
%	the `dual' of the root system of $V$.
%	I.e. let $E_i$ be any positive root of $\Nichols V$. Then
%	$E_i$ is contained in a one-dimensional Yetter-Drinfeld module over $k \mathbb Z^N$ this means in particular there is a character $\xi_i$ implementing the action.
%	Then $\Delta^+$ is given by the minimal entry-wise positive vectors in $\mathfrak h^*$ representing these characters.
%	
%	We want to ensure that this spans an honest root system.
%	The axioms of an abstract root systems are as follows:
%	
%	
%
%
%
%	Problem 1: How do we see that $\Delta^+$ spans $\mathfrak h^*$ ? 
%	It even needs to suffice to consider the simple roots. 
%	
%	The second condition is easy.
%	Conditions three  and four require us to define a bilinear form on $\mathfrak h^*$ and we should use the Cartan matrix to do so.
%	
%	Now let us assume that condition 1 holds and that we have a fixed set of simple roots $\alpha_1, \dotsc , \alpha _N$.
%	We define an inner-product inductively using the Cartan matrix $A$. 	
%	First set $(\alpha_1, \alpha_1) =1$.
%	Next set 
%	$$
%		(\alpha_i, \alpha_i) = \frac {a_{i1}}{a_{1i}}.
%	$$
%	
%	Finally for $i \neq j$
%	$$
%		(\alpha_i, \alpha_j)= \frac{1}{2}(\alpha_i, \alpha_i) a_{ji}.
%	$$	
%	If our Cartan matrix isn't symmetric or symmetrisable we might run into a problem (?). 
%	
%	The fact that this is a non-degenerate symmetric bilinear form should now follow from some conditions of the Cartan matrix.
%	
%	Now fix any pair of simple  roots $\alpha_i ,\alpha_j$ then
%	$$
%		2 \frac{(\alpha_i ,\alpha_j)}{(\alpha_j, \alpha_j)} =
%		\frac{a_{j1}}{a_{1j}}a_{ji} \in \mathbb Z
%	$$
%	Next we want to define (possibly graded (?) ) Verma modules. 
%	
%----- After this line everything is bullshit to some degree -----
%
%
%In the following we fix an $n$-dimensional $k$-vector space $V$ together with an ordered basis $x_1,\dotsc x_n$ and a matrix of a braiding $\mathfrak q$.
%Let $N_i := \min\{ m \in \mathbb N_+ \mid \mathfrak  q_{ii}^{m}=1 \}$ and set
%$$
%	G:= \text{span}_\mathbb Z\{g_1, \dotsc g_n \}/ \langle  g_1^{2N_1}, \dotsc g_n^{2N_n} \rangle
%$$
%Now choose characters $\xi_1,\dotsc \xi_n$ such that
%$$
%	\xi_i(g_j) = \mathfrak q_{ij}.
%$$
%\begin{lemma}
%	The dual group $\widehat G$ is isomorphic to $G$. 
%	It has generators $\{\chi_1, \dotsc \chi_n\}$ defined by
%	$$
%	\chi_i(g_j^2) = e^{2\pi i/N_i }\delta_{ij}.
%	$$
%\end{lemma}
%
%\begin{proof}
%
%\end{proof}
%
%Next let us define another $n$-dimensional vector space $W:= \text{span}_\mathbb C \{ y_1,\dotsc y_n \}$.
%We equip $W$ with a matrix of the braiding $\widehat {\mathfrak q}$ 
%defined by $\widehat {\mathfrak q}_{ij}:= \mathfrak q_{ij}^{-1}$.
%
%Both, $V$ and $W$, are YD modules over $\widehat G$. 
%\begin{lemma}
%	If $\mathfrak q$ is of Cartan type then so is $\widehat{\mathfrak q}$ and the associated Cartan-matrices agree.
%	The root system of the Nichols algebra $\Nichols W$ of $W$ is isomorphic to the root system of $\Nichols V$.
%\end{lemma}
%\begin{proof}
%	First part is a calculation. Second part should follow from the fact that the root system is uniquely identified by its generalised Dynkin diagram.
%\end{proof}
%
%On a moral level we want to "split" the root system of $\Nichols V$ into a part generating the PBW-basis of $\Nichols V$, which we treat as the negative roots and a part $\Nichols W$ which are the positive roots.
%
%\begin{theorem}
%	The Drinfeld double $D(V)$ of the bosonosation of $\Nichols V$ is given 
%	by $T(V\oplus W)\# k(\tilde G \times G)$ by the ideal generated by $I(V)$, $I(W)$ and 
%	\begin{gather} \label{eq: RelInDouble}
%		x_i y_j = \xi_j^{-1}(g_i)y_j x_i -\delta_{ij}(\xi_ig_i-1) \qquad \qquad 1 \leq i,j\leq n.
%	\end{gather}	
%\end{theorem}
%
%In the following we will frequently consider the quotient 
%of $D(G)$ identifying $g_i$ with $\chi_i$.
%
%\begin{convention}
%	In the following we write
%	\begin{enumerate}
%		\item $\mathfrak h:= k(G)$. 
%		\item $\mathfrak n_- := \Nichols V$
%		\item $\mathfrak n_+ := \Nichols W$
%		\item $\mathfrak b := k(G) \# \Nichols W$
%	\end{enumerate}
%\end{convention}
%
%\begin{definition}
%	We call $\Gamma := \{ \text{iso classes of simple representations of } \mathfrak h \}$ the set of weights of $D(V)$. 
%\end{definition}
%	Any weight $\omega \in \Gamma$ is determined by a character $\xi = (\chi_1^{o_1}, \dotsc \chi_n^{o_n}) \in k\tilde G$.
%	
%\begin{lemma}
%
%	We can uniquely identify a weight with an element in $h^* := \text{span}_k\{ \chi_1, \dotsc \chi_n  \} $
%	In particular the image of $\Gamma$ generates a lattice $\Lambda$ in $h^*$.
%\end{lemma}
%	Note that here problems arise due to the fact that we consider finte groups. 
%	
%\begin{remark}
%	In Vay it is claimed that graded modules overcome this problem.
%\end{remark}
%
%\begin{lemma}
%	The rootsystem of $\Nichols V$ spans a sublattice  $\Xi \subset \Lambda$.
%	The monoid generated by the positive roots will be denoted $\Xi^+$.
%\end{lemma}
%
%\begin{definition}	
%	A Verma module of weight $\lambda\in \Lambda$ is defined as the induced $D(V)$ module
%	\begin{align}
%		M(\lambda):= D( V)\otimes _{\mathfrak b} \mathbb C_\lambda.	
%	\end{align}
%\end{definition}
%
%\begin{theorem}
%	Let $M:=M(\lambda)$ be a Verma-module of weight $\lambda \in \Lambda$. It is generated by a highest weight vector of weight $\lambda$, i.e. a vector $v_\lambda\in M$ such that
%	$h \triangleright v = \lambda(h) v$ for all $h \in \mathfrak h$ and 
%	$E_i \triangleright v =0$ for $1 \leq i \leq n$ and $D(V)\triangleright v = M$. Moreover $v_\lambda$ is unique up to scalar multiplication.
%\end{theorem}
%\begin{proof}
%	Consider $v_\lambda := 1 \otimes_{D(G)\# \Nichols V} 1\neq 0$. By definition $v_\lambda$ generates $M$ and is a highest weight vector of weight $\lambda$.
%	Suppose that $w\in M$ is another highest weight vector generating $M$.Then there exist elements $x, y \in D(V)$ such that $x\triangleright w = v_\lambda$ and $w = y \triangleright v_\lambda$.
%	In particular $1 = xy = yx$. 
%	Due to the relation \eqref{eq: RelInDouble} we can assume without loss of generality that
%	$x,y \in \mathfrak n_- \otimes \mathfrak h$.
%	It follows from the fact that $D(G)$ is a graded algebra that $x,y \in \mathfrak h$, i.e. in the degree zero component. This implies that there is a scalar $\lambda(x)$, such that $\lambda(x) w = x w = v_\lambda$.
%\end{proof}
%
%
%
%\begin{lemma}
%	Let $M$ be a $D(V)$ module and $v \in M$ of weight $\lambda$.
%	$$
%		h \triangleright(E_i\triangleright x)= (\lambda + \alpha)(h) E\triangleright x
%	$$ 
%	for all $h \in \mathfrak h$, where $\alpha$ is the weight of the PBW-generator $E_i$.
%\end{lemma}
%\begin{proof}
%	Note that in $D(V)$ we have for any  $E$ in $\Nichols V$ with weight $\alpha$ and $h \in \mathfrak h$ that
%	$hEh^{-1} = \alpha(h) E$. This implies the above claim
%\end{proof}
%
%\begin{definition}
%	We define a partial ordering on $\Lambda$ by
%	\begin{align}
%		\mu \leq \lambda \le \overset{\text{(Def)}}{\lambda - \mu \in \Xi^+}.
%	\end{align}
%	A weight $\lambda$ is called dominant if its contained in the fundamental Weyl chamber.
%\end{definition}
%
%
%In the following $W$ denotes the (finite!) Weyl group associated to the root system. We claim that there is a unique inner product such that wrt to this immer prod. the roots system is a root system in the sense of Lie-theory. We denote it by $(\cdot \mid \cdot)$.
%
%We define the  fundamental weights $\omega_1,\dotsc \omega_n$ in the same way we would define fundamental weights in Lie theory. 
%Claim: The fundamental weights are integral, i.e. actual weights.
%
%\begin{definition}
%	The affine action of $W$ on $h^*$ is defined by
%	$$
%		w\cdot \lambda = w(\lambda+\delta)-\delta
%	$$
%	where $\delta := \omega_1+\dotsc+ \omega_n$.
%\end{definition}
%
%\begin{lemma}
%	Let $w \in W$ and $\lambda \in \Lambda$ then 
%	$w\cdot \lambda \in \Lambda$.  
%\end{lemma}
%\begin{proof}
%	Let $\alpha$ be a root.
%	$$
%		(w\cdot \lambda, w\alpha)/(w\alpha,w \alpha) =		
%		(w(\lambda + \delta)- \delta, w\alpha)/(w\alpha,w \alpha)
%		=(w(\lambda + \delta), w\alpha)/(w\alpha,w \alpha) - (\delta, w\alpha)/(w\alpha,w \alpha) \in 1/2 \mathbb Z
%	$$
%\end{proof}
%
%\section{Kumar 9.2}
%We copy the exposition of Kumar to show the following statement.
%\begin{theorem}
%	Let $\lambda \in \Lambda$ be a weight and $v,w\in W$.
%	\begin{align}
%		\dim Hom(M(v\cdot \lambda),M(w\cdot \lambda)) =
%		\begin{cases}
%		0 & \text{if } v < w \\
%		1 & \text{else .}  
%		\end{cases}
%	\end{align}
%\end{theorem}
%
%Background on Weyl-groups and root systems
%
%\begin{definition}
%	An abstract root system in a finite-dimensional real inner product space $V, (\cdot, \cdot)$ is a finite set $\Delta\in V\setminus \{0\}$ such that
%	\begin{enumerate}
%		\item $\Delta$ spans $V$,
%		\item If $\alpha \in \Delta$ then $\mathbb R  \alpha \cap \Delta = \{\pm \alpha\}$
%		\item $\Delta$ is closed under reflections. That is
%		for any $\alpha,\beta \in \Delta$ the element
%		$$
%			s_\alpha (\beta) = \beta - 2\alpha \frac{(\alpha, \beta)}{(\alpha,\alpha)}
%		$$
%		is contained in $\Delta$.
%		\item For any two roots $ \alpha ,\beta \in \Delta $ the number $ \langle \beta ,\alpha \rangle :=2\frac {(\alpha ,\beta )}{(\alpha ,\alpha )}$ is an integer.
%	\end{enumerate} 
%\end{definition}
%
%\begin{lemma}
%	Let $s_i\in W$ be a simple reflection and $\lambda \in \Delta$ an integral weight such that 
%	$s_i \cdot \lambda < \lambda$.
%	Then 
%	$$ 
%		\lambda- s_i \cdot \lambda = n \alpha_i, \text{ with } n :=  2\frac{(\lambda, a_i)}{(a_i, a_i)} +1 
%	$$ where $\alpha_i \in \mathfrak h^*$ is the simple root corresponding to $\Lambda$.
%\end{lemma}
%\begin{proof}
%	We compute
%	\begin{align*}
%		\lambda- s_i \cdot \lambda 
%		&= \lambda -( s_i (\lambda+ \delta) -\delta) 
%		= \lambda + \delta - s_i(\lambda +\delta)
%		\\
%		&= \lambda + \delta - (\lambda +\delta) + 2\alpha_i \frac{(\alpha_i, \lambda + \delta)}{(\alpha_i, \alpha_i)}
%		= 2\alpha_i \frac{(\alpha_i, \lambda + \delta)}{(\alpha_i, \alpha_i)}
%		\\
%		& = 2\alpha_i 
%		\left( \frac{(\alpha_i, \lambda )}{(\alpha_i, \alpha_i)}
%		+\frac{(\alpha_i, \delta )}{(\alpha_i, \alpha_i)}
%		\right)
%		= \alpha_i 
%		\left( 2\frac{(\alpha_i, \lambda )}{(\alpha_i, \alpha_i)}
%		+ 2\frac{(\alpha_i, \delta )}{(\alpha_i, \alpha_i)}
%		\right) 
%		\\
%		&= \alpha_i 
%		\left( 2\frac{(\alpha_i, \lambda )}{(\alpha_i, \alpha_i)}
%		+ 2\frac{(\alpha_i, \delta )}{(\alpha_i, \alpha_i)}
%		\right)  = n \alpha_i.
%	\end{align*}
%	In the last step we used that $\delta$ is the sum of fundamental weights.
%\end{proof}
%
%For the next lemma we use notation as in  Kumar 9.2.4.
%\begin{lemma}
%	For a weight $\lambda \in \mathfrak h^*$ and a simple reflection $s_i\in W$ we have
%	\begin{align}
%		\dim \Hom_{D(V)}(M(\underline{s_i} \cdot \lambda), M(\lambda))=1
%	\end{align}
%\end{lemma}
%\begin{proof}
%	Assume $s_i \cdot \lambda = \lambda$.
%	Any non-zero endomorphism of $M(\lambda)$ maps a maximal vector onto a maximal vector.
%	As the space  generated by the maximal vectors is one-dimensional we have
%	$\dim \Hom_{D(V)}(M(\lambda), M(\lambda)) \leq 1$.
%	Since the identity is contained in $\Hom_{D(V)}(M(\lambda), M(\lambda))$ it needs to be at least one-dimensional. The claim follows.
%	
%	Next assume $s_i \cdot \lambda < \lambda$ and let $v_\lambda \in M(\lambda)$ be a maximal vector of weight $\lambda$.
%	We define $v:= F_i^n v_\lambda$.
%	ASSUMPTION for any grouplike $g\in \mathfrak h$ we have
%	$g F_i= \alpha_i^{-1}(g) F_i g$. 
%	(Note that $\alpha_i^{-1}$ is the weight of $F_i$). 
%	Now $g v = g F_i^n v_\lambda = \alpha_i^{-n} \lambda(g) F_i^n v_\lambda$
%	Thus $v$ is a weight vector of weight $\lambda - n\alpha_i= s_i \cdot \lambda$.
%	Moreover the claim is that this vector ismaximal. HERE WE NEED TO  UNDERSTAND THE RELATIONS IN $D(V)$ BETTER.
%	
%	As a consequence we have $\Hom_{D(V)}(M(\underline{s_i}\cdot \lambda), M(\lambda)) \geq 1$.
%	
%	To show that the weight space is at most one-dimensional we might use that we have an explicit basis (PBW) ?
%\end{proof}
%
%\begin{proof}
%	We try to copy the ideas of Kumar 9.2.3.
%	
%	Step zero: $\dim Hom(M(\lambda),M(\lambda)) =1$ (uses the fact that Verma modules are generated by a unique vector up to scalar mult and how their universal mapping property).
%		
%	Fix $s_i$ a simple refection.
%	First case: Let $s_i \cdot \lambda \leq \lambda$.
%	Then there is a $m\in \mathbb N$ such that $\lambda -s_i \cdot \lambda = m \alpha_i$. (This we need to proof in our case too but seems to be a general geometric fact of the affine action. Also we might have to think about an interpretation of this fact in the graded sense).
%	
%	We now want to show that $M(\lambda)$ contains a one-dimensional weight space of weight $s_i*\lambda$.	
%	We define 	
%	$$ 
%		v:= F_i^n \triangleright v_\lambda.
%	$$
%	A direct calculation shows that it is of the desired weight. 
%	Using the fact that $M(\lambda)$ is a free module over $\mathfrak n_-$ we obtain moreover that this is vector spans the weight space of weight $s_i*\lambda$. It follows that 
%	$Hom(s_i*\lambda, \lambda)$ is one-dimensional.
%	
%	Next case: Let $s_i\cdot \lambda >\lambda$:
%	Use that maps from Verma-modules into Verma modules are injective.
%	(Do we get problems if the group algebra has zero divisors?)
%	Here we want to copy arguments froms Kumar 9.2.6.
%	
%	Final step:
%	Let $v\geq w\in W$ we prove the claim via induction on $l(v)$.
%	If $l(v)=0$ we have $v=w$ thus the claim holds.
%	Now fix a simple reflection $s_i$ such that $v'= s_i v < v$.
%	Write $w' = w$ if $s_iw>w$ and $w' = s_iw$ otherwise.
%	Then 
%	$$
%		\dim Hom(v*\lambda, w*\lambda)\geq \dim Hom(v'*\lambda, w'* \lambda) = 1
%	$$	
%	Where the  inequallity follows by the two cases considered previously and the equality follows from the induction hypothesis.
%	TODO: TBC
%	
%\end{proof}
%
%
%CONSEQUENCE: THE BGG RESOLUTION IS AT LEAST A CHAIN COMPLEX OF PROJECTIVE $\mathfrak n_-$ modules.
%SHOW PROJECTIVITY IN THE SPIRIT OF HECKENBERGER.
%
%
%\begin{lemma}\label{lem: highestweightVectInducesMor}
%	Let $M$ be a $D(V)$ module such that there exists a vector $w\in M$ of weight $\lambda \in \Lambda$ and $E_i \triangleright w =0$ for $1 \leq i\leq n$. Then the assignment
%	$$
%		f: M(\lambda) \rightarrow M , x \otimes_\mathfrak b \gamma \mapsto  \gamma x\triangleright w
%	$$
%	defines a morphism of $D(V)$-modules.
%\end{lemma}
%\begin{proof}
%	We show that $f$ is well-defined.
%	Define the map
%	$\tilde f: D(V)\otimes \mathbb C \rightarrow M(\lambda), x\otimes \gamma \mapsto \gamma x \triangleright w$.
%	Observe that $\ker \tilde f:= \langle E_1\otimes 1, \dotsc E_n\otimes 1 \rangle.$ Given the projection $\pi:  D(V)\otimes \mathbb C \rightarrow M(\lambda)$ we see, that $\pi(\ker \tilde f) =\{0\}$. Thus $f$ is well-defined as a-linear map.
%	One checks that $f$ moreover is a morphism of $D(V)$ modules.
%\end{proof}
%
%In fact any morphism between Verma modules and modules is of the above form.
%
%\begin{lemma}
%	Let $M$ be a $D(V)$ module and $M(\lambda)$ such that there exists a morphism $f: M(\lambda) \rightarrow M$ of modules.
%	Then $w:=f(v_\lambda)$ is a weight vector of weight $\lambda$ and
%	$E_i \triangleright w = 0$ for $1\leq i \leq n$.
%\end{lemma}
%
%\begin{proof}
%	Let $h \in \mathfrak h$.
%	Then $h \triangleright v_\lambda = \lambda(h)  v_\lambda$ and
%	$$
%	h \triangleright w = h \triangleright f(v_\lambda) =
%	 f(h \triangleright v_\lambda) = f(\lambda(h)  v_\lambda)= \lambda(h) w.
%	 $$
%	 Moreover we have
%	 $$
%	 	0 = f(E_i \triangleright v_\lambda) = E_i \triangleright w.
%	 $$		  
%\end{proof}
%
%\begin{theorem}
%	Assume $M(\lambda)$ to be a Verma module of $D(V)$ whose weight is dominant. Then $M(\lambda)$ is projective.
%\end{theorem}
%
%\begin{proof}
%	Assume $M\xrightarrow{f}N$ to be an epimorphism of $D(V)$ modules and $\phi: M(\lambda)\rightarrow N$ to be a homomorphism.
%	Let $v_\lambda\in M(\lambda)$ be the highest weight vector and write $w:= \phi(v)\in N$.
%	Choose an element $u\in f^{-1}(w)$.
%	We claim that $u$ is of weight $\lambda$.
%	
%	Next we need to show that $E_i\triangleright u =0$ for all $1\leq i\leq n$.
%	Assume this not to be true.
%	And choose $\tilde u:= E_i \triangleright u \neq u$. 
%	We have 
%	$h \triangleright \tilde u = hE_i \triangleright u =  \lambda (h) L_i(h) E_i u =(\lambda L_i)(h) \tilde u.$
%	
%	Thus $\tilde u$ is of weight $\lambda+ \alpha$ with $\alpha \in \Xi^+$. This contradicts the dominance of $\lambda$. 
%	In other words $u$ is a hightest weight vector of weight $\lambda$ By Lemma \ref{lem: highestweightVectInducesMor} there exists a morphisn $g: M(\lambda)\rightarrow M$ which, by construction, satisfies 
%	$gf = \phi$.
%\end{proof}
%Now consider the Weyl group $W$ of the root system of $\Nichols V$.
%We claim that the action of $W$ on the lattice $\Lambda$ is well-defined.
%\begin{theorem}
%	Let $w\in W$ be an element and $\lambda \in \Lambda$ be a dominant weight.
%	Then $w*\lambda$ is dominant as well. 
%\end{theorem}
%\begin{proof}
%	Suppose that $w*\lambda$ is not dominant. Then
%	$w*\lambda - \mu \in \Xi^+$ for some $\mu< w*\lambda$.
%	Write $a= w*\lambda-\nu$
%	This implies $w*a \in \Xi$. 
%	By assumption $w*a \in \Xi^+$
%	
%\end{proof}
%
%\begin{theorem}
%	Morphisms between Verma modules associated to roots in the root system are determined by the Bruhat order of the Weyl group.
%\end{theorem}
%\begin{proof}
%	Uniqueness of the maps follows from the universal property. 
%\end{proof}
%
%\begin{theorem}
%	Every square is exact.
%\end{theorem}
%\begin{proof}
%	 Up to scalars there is a unique map $M_w \rightarrow M_w'$ since we have two different factorisations of these maps there are signs such that the square needs to be zero. 
%	 
%	 Exactness should follow from the injectivity of the morphisms.
%\end{proof}
%
%
%\begin{theorem}
%	The BGG resolution exists.
%\end{theorem}
%
%\section{Kumar 9.2}
%
%
%

\bibliographystyle{alpha}
\bibliography{hochschild_pointed_bib}

\end{document}