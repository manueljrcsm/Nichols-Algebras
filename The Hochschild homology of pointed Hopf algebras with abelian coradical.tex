\documentclass{amsart}
\usepackage{amsaddr}
\usepackage{amsthm}
\usepackage{amsmath}
\usepackage{amssymb}
\usepackage{tikz-cd}
\usepackage{tikz}
\usetikzlibrary{calc}



\usepackage[square,numbers]{natbib}
\bibliographystyle{abbrvnat}



\newtheorem{theorem}{Theorem}
\newtheorem{lemma}{Lemma}
\newtheorem{corollary}{Corollary}
\newtheorem*{conjecture}{Conjecture}


\theoremstyle{definition}
\newtheorem{definition}{Definition}
\newtheorem{question}{Question}
\newtheorem*{question*}{Question}
\newtheorem*{remark}{Remark}
\newtheorem*{convention}{Convention}
\newtheorem{example}{Example}

\newcommand{\YD}[1]{\ensuremath{{}^{#1}_{#1}\mathcal{YD}}}
\newcommand{\Nichols}[1]{\ensuremath{\mathcal{B}(#1)}}
\DeclareMathOperator{\Hom}{Hom}


\author{}
\address[]{
	Technische Universität Dresden,
	Institut f\"ur Geometrie,
	\\
	Zellescher Weg 12-14, 01062 Dresden}
\email{}
\date{\today}
\subjclass[2010]{}
\keywords{}	
\title{The Hochschild homology of pointed Hopf algebras with abelian coradical}
\begin{document}
	\begin{abstract}
		
	\end{abstract}
	\maketitle
	
	\section{Preliminaries - Nichols algebras of Cartan type}
	
	In this section main definitions of Nichols algebras of diagonal/Cartan type are recalled according to the Andruskiewitsch survey. 
	
	\begin{convention} $ $ \\
		\begin{enumerate}
			\item $k$ denotes an algebraically closed field of characteristic zero
			\item $G$ denotes a finite abelian group (finitely generated should be fine too!)
			\item $\YD G$ denotes the category of Yetter-Drinfeld modules over the group algebra of $G$.
			\item For a natural number $n\in \mathbb N$ we denote $[n]:= \{1,\dotsc, n\}$.
		\end{enumerate}
	\end{convention}
	
	\subsection{Nichols algebras}
	
		In short, Nichols algebras are braided Hopf algebras which can be seen as generalisations of exterior or symmetric algebras. Even though it is possible to define them abstractly, one usually considers them as Hopf algebra objects in the category of Yetter-Drinfeld modules over some Hopf algebra $H$.
	\begin{lemma}
		Let $H$ be some group. The category $\YD{H}$ of $H$-Yetter-Drinfeld modules is, as a braided monoidal category, equivalent (in fact, isomorphic) to the category of $H$-graded $H$-modules. That is the category $\mathcal C$ whose
		\begin{itemize}
			\item  objects are $H$-graded $H$-modules $V:= \oplus_{h\in H} V_h$ such that
			$n \triangleright V_h \subset V_{nhn^{-1}}$,
			\item morphisms are $H$-linear maps $f: \oplus_{h\in H} V_h \rightarrow \oplus{h\in H} W_h$, which respect the grading,
			\item tensor product is defined via
			$$
			\oplus_{h\in H} V_h \bigotimes \oplus_{n \in H} W_n := \oplus_{h \in H}(\oplus_{n \in H} V_{hn^{-1}} \otimes W_n).
			$$
			\item braiding is given by
			$$
			\sigma: \oplus_{h\in H} V_h \bigotimes \oplus_{n \in H} W_n \rightarrow 
			\oplus_{n \in H} W_n \bigotimes \oplus_{h\in H} V_h, \qquad  v\otimes  w \mapsto \left(h \triangleright w\right) \otimes v \text{ for } v\in V_h.
			$$	
		\end{itemize}
	\end{lemma}
%	\begin{proof}
%		A direct calculation shows that every $H$-graded $H$-module is a Yetter-Drinfeld module. This yields a functor of braided monoidal categories.
%		To prove the converse, consider any Yetter-Drinfeld module $M$ over $H$. 
%		Write $M_h :=\{m \in M \mid \delta(m) = h \otimes m\}$. One shows $M \cong \oplus M_h$ as comodules and modules and the Yetter-Drinfeld condition implies 
%		$n \triangleright M_h \subset M_{nhn^{-1}}$.
%		This defines another functor of braided monoidal categories. 
%		It is proven directly that these functors are mutually inverse to each other.
%	\end{proof}
	
	There exist various definitions of Nichols algebras. The most conceptually compelling one is given in terms of  their universal property.
	
	\begin{definition}
		Let $V \in \YD{H}$ be some Yetter-Drinfeld module over a Hopf algebra $H$.
		The Nichols algebra of $V$ is the braided graded Hopf algebra $\Nichols V= \oplus_{n \in \mathbb N_0} \Nichols V_n\in \YD{H}$ uniquely determined by
		\begin{enumerate}
			\item It is a connected algebra, i.e. $\Nichols V_0 = \{0\}$ and $\Nichols V_1 = V$.
			\item The set  $Pr(\Nichols V)$ of primitives is given by $V$. 
			\item It is as an algebra generated in degree one, i.e. it is generated by $V$. 
		\end{enumerate}
		The dimension of $V$ is called the rank of the Nichols algebra $\Nichols V$.
	\end{definition}
	
	There is a more hands-on description, which realises the Nichols algebra as a quotient of the tensor algebra of $V$. 
	
	\begin{lemma}
		Let $V \in \YD{H}$ be some Yetter-Drinfeld module over a Hopf algebra $H$.
		The Nichols algebra $\Nichols V$ of $V$ is, up to unique isomorphism, given by the quotient $T(V)/J$ of the tensor algebra $T(V)$ of $V$ by the maximal two-sided ideal $J$ which satisfies
		\begin{enumerate}
			\item $J$ is a Yetter-Drinfeld submodule of $T(V)$,
			\item $J$ is a two-sided coideal,
			\item $J$ is generated by homogenous elements of degree $\geq 2$.
		\end{enumerate}
	\end{lemma}
	
	
		\begin{theorem}[Bosonisation]
		Let $H$ be a Hopf algebra with invertible antipode and $R$ a braided Hopf algebra in $^H_H \mathcal{YD}$. 
		The bosonisation of $R$ by $H$ is the Hopf algebra $R\#H$ whose underlying vector space is $R \otimes H$ and whose multiplication, comultiplication and antipode are defined by
		\begin{align*}
		\begin{aligned}
		& (r\#g)(s\#h)=r(g_{(1)}\triangleright s) \# g_{(2)}h
		\\
		& \Delta(r\#g):= r^{(1)}\#r^{(2)}_{(-1)}g_{(1)} \otimes r^{(2)}_{(0)}\# g_{(2)}
		\\
		& S(r\# g) = S_H(r_{(-1)(2)} g_{(2)})\triangleright S_R(r_{(0)})\#S_H(r_{(-1)(1)} g_{(1)}).
		\end{aligned}
		\end{align*}		
	\end{theorem}
	
	A fact which will be of utter importance to us is that there is a section of Hopf algebras 
	$H\overset{\iota}{\underset{\pi}{\rightleftarrows}} R\#H$ given by
	$$
	\iota:H \rightarrow R\#H, \quad h \mapsto 1\#h, \qquad \pi: R\#H\rightarrow H, \quad r\#h \mapsto \varepsilon(r)h.
	$$
	
	Commonly, Nichols algebras are distinguished in terms of the braiding.
	
	\begin{definition}
		Let $V$ be a finite dimensional vector space and $c: V\otimes V \rightarrow V \otimes V$ be a solution of the YBE. We say that $(V,c)$
		\begin{enumerate}
			\item has \emph{group type} if there exists a basis $v_1, \dotsc, v_\theta$ of $V$ and elements $g_i(v_j)\in V$ such that
			$
			c(v_i \otimes v_j) = g_i(v_j) \otimes v_i.
			$
			Necessarily $g_i \in GL(V)$.
			\item has \emph{finite} (resp. \emph{abelian}) group type if it is of group type and the subgroup of $GL(V)$ spanned by $g_1,\dotsc g_\theta$ is finite (resp. abelian).
			\item is of \emph{diagonal} type if there there exists a basis $v_1, \dotsc, v_\theta$ of $V$ and scalars $q_{ij} \in k$ such that
			$
			c(v_i \otimes v_j) = q_{ij} v_j \otimes v_i.
			$
			The matrix $(q_{ij})$ is called the matrix of the braiding.
		\end{enumerate}
	\end{definition}

	\subsection{Nichols algebras of diagonal type}
	
	\begin{lemma}
		Let $(V,c)$ be of group type and $v_1,\dotsc v_\theta$ a basis such that there are $g_1,\dotsc, g_\theta \in GL(V)$ satisfying $c(v_i\otimes v_j) = g_i(v_j) \otimes v_i$. 
		Then $V$ becomes a Yetter-Drinfeld module over the subgroup of $GL(V)$ generated by $\{ g_1, \dotsc, g_\theta \}$ 
		by defining
		$$
		g_i \triangleright v_j := g_i(v_j), \qquad \delta(v_i)= g_i \otimes v_i.
		$$ 
		
		If $(V,c)$ is of diagonal type it is necessarily of abelian group type.
		It is also of finite group type if every entry of the matrix of the braiding is a root of unity.
	\end{lemma}
	
	\begin{example}
		Let $N\geq 2 $ and $C_N$ be the cyclic group of order $N$. Write $g$ for a generator of $C_N$.
		Fix some primitive $N$-th root of unity $q\in k$. 
		Then there exists a unique character $\xi: C_N \rightarrow k$, $g \mapsto  q$.
		We call integers $a_1, a_2, b_1, b_2 \in \mathbb Z_N$ satisfying 
		$$
		a_1 b_1 \neq 0 \quad a_2 b_2 \neq 0 , \quad a_1 b_2 + a_2 b_1 = 0 \mod N
		$$
		\emph{parameters of a generalised Taft algebra}.
		The vector space $V:=\langle v_1, v_2\rangle$ becomes a Yetter-Drinfeld module over $C_N$ by defining 
		$$
		g\triangleright v_i := q^{b_i} v_i,\qquad \delta(v_i) = g^{a_i} \otimes v_i.
		$$
		Its Nichols algebra $\mathcal B(V)$ is generated, as an algebra, in degree one by the primitive elements $v_1$ and $v_2$, together with the relations
		$$
		v_i^{N_i} = 0, \qquad v_1 v_2 = q^{a_1b_2} v_2 v_1,
		$$ 
		where $N_i:=|\langle a_i b_i \rangle |$ is the order of the subgroup of $\mathbb Z_N$ generated by $a_i b_i$.

		The bosonisation of the above described Nichols algebra is (the cooposite) of a generalised Taft algebra, i.e. a Hopf algebra $H_q(a_1, a_2, b_1, b_2)$ parametrised by some primitive $N$-th roof of unity $q$  and parameters of a generalised Taft algebra $a_1, a_2, b_1, b_2$.
		More explicitly, it is generated by three generators $g$, $x$ and $y$ and relations
		\begin{gather}
		\begin{gathered} \label{eq:AlgRelations}
		g^N = 1, \qquad
		x^{N_x} = 0, \qquad
		y^{N_y} = 0,	
		\\
		gx = q^{b_1}xg, \qquad
		gy = q^{b_2}yg, \qquad
		xy = q^{a_1b_2} yx,
		\end{gathered}
		\\
		\begin{gathered}\label{eq:CoalgRelations}
		\Delta(x) = x \otimes 1 + g^{a_1} \otimes x,  \qquad
		\Delta(y) = y \otimes 1 + g^{a_2} \otimes y,
		\\
		\Delta(g) = g \otimes g , \;\;
		\end{gathered}
		\\
		\begin{gathered}\label{eq:AntipodeRelations}
		S(g) = g^{-1}, \qquad
		S(x) = - q^{a_1b_2}xg^{-a_1}, \qquad
		S(y) = - q^{a_2b_2}yg^{-a_2}.
		\end{gathered}
		\end{gather}		
	\end{example}



	\begin{definition}
		Let $(V,q)$ be a braided vector space of diagonal type. That is there exists a fixed ordered basis
		$v_1,\dotsc v_n$ s of $V$ such that the braiding of $V$ is implemented by the $n \times n$-matrix $q$. The \emph{bilinear form associated to the braiding} is defined as the map 
		\begin{align}
		\begin{aligned}
		&\mathfrak{q}: \mathbb Z^{[n]} \times \mathbb Z^{[n]} \rightarrow k^\times \\
		&\mathfrak{q}(\alpha, \beta)= \prod_{i,j=1}^{n}q_{ij}^{\alpha_i\beta_j}, \quad \text{ where } \alpha= (\alpha_1,\dotsc,\alpha_n), \beta = (\beta_1, \dotsc,\beta_n)
		\end{aligned}
		\end{align} 	
	\end{definition}

	\begin{definition}
		Let $(V,q)$ be a braided vector space of diagonal type with dimension $\dim V =n$ and $R$ a quotient of $T(V)$ by a $\mathbb Z^{[n]}$ homogeneous ideal.
		Then the braiding $c: R\otimes R \rightarrow R \otimes R$ satisfies 
		\begin{align} 
			c(x\otimes y) = \mathfrak{q}(\alpha,\beta) y \otimes x,\qquad  \text{ for } x\in R_\alpha, y \in R_\beta \text{ homgenous elements of degree $\alpha$ and $\beta$}.
 		\end{align}	
	\end{definition}

	\begin{definition}
		Let $(R,c)$ be any braided Hopf algebra. The adjoint action of $R$ on itself is defined by
		\begin{align}
			ad: R\rightarrow GL(R),  ad(x)(y):= 	\mu(id \otimes \mu)(id \otimes id \otimes S)(id \otimes c)(\Delta \otimes id)(x \otimes y)
		\end{align}
		The braided commutator is defined as the map
		\begin{align}
			[-,-]_c: R \otimes R \rightarrow R, [x,y]_c := \mu (id-c)(x\otimes y)
		\end{align}
	\end{definition}

	We fix the ordered standard basis $e_1, \dotsc e_n \in\mathbb Z^n \subset k^n$.

	\begin{lemma}
		If $(V,q)$ is a braided vector space of diagonal type with dimension $\dim V =n$ and $R$ a quotient of $T(V)$ by a $\mathbb Z^{n}$ homogeneous ideal which is also a coideal (i.e. $R$ is a braided Hopf algebra then:
		\begin{enumerate}
			\item $[v_i,x]_c = v_ix - \mathfrak{q}(e_i,\alpha)x v_i \text{ for $v_i \in V$ and $x_\alpha \in R_\alpha$}$
			\item $[v_i,y]_c = ad(x)(y)$.
		\end{enumerate}
	\end{lemma}

	\begin{lemma}
		If $(V,q)$ is a braided vector space of diagonal type with dimension $\dim V =n$ and $R$ a quotient of $T(V)$ by a $\mathbb Z^{n}$ homogeneous ideal then the braided commutator satisfies:
		\begin{align}
			[u,vw]_c & = [u,v]_c w + \mathfrak{q}(\alpha, \beta) v[u,w] \\	
			[uv,w]_c & = u[v,w]_c + \mathfrak{q}(\beta, \gamma) [u,w]v \\
			[u,[v,w]]_c & = [[u,v]_c , w]_c  - \mathfrak q (\beta, \gamma) [u, w]_c w + \mathfrak q (\alpha, \beta) v [u, w]_c\\ 
		\end{align}
	\end{lemma}

	\subsection{Nichols algebras of Cartan type}
	
	We fix a braided vector space $(V,q)$ of dimension $n$.
	
	\begin{definition}
		Let $(V,q)$ be a braided vector space, $\Nichols V$ its Nichols algebra and $B$ a generating set of the  PBW basis of $\Nichols V$.
		We define the set of positive roots as
		\begin{align}
			\Delta^+_q := \{deg(b) \mid b \in B \} \subset k^n,
		\end{align}
		where $deg$ denotes the degree with respect to the $\mathbb Z^{[n]}$ braiding.
		The Nichols algebra $\Nichols B$ is called arithmetic if $|\Delta^+_q|$ is finite.
	\end{definition}

	\begin{remark}
		If $\Nichols B$ is arithmetic every root has multiplicity one, i.e.
		there is a bijection $\Delta^+_q \rightarrow B$, where $B$ is a generating set of the PBW basis of $\Nichols B$.
	\end{remark}
	
	\begin{definition}
		A generalised Cartan matrix is a matrix $A= (a_{ij})_{1\leq i,j \leq n}$ such that
		\begin{gather}
			a_{ii}=2, \qquad, a_{ij}\leq 0 \text{ for } i \neq j, \qquad a_{ij}=0 \Leftrightarrow a_{ji}=0.			
		\end{gather}
		(See Kumar)
	\end{definition}
	
	\begin{definition}
		We call $(V,q)$ of \emph{Cartan type}, if there exists a generalised Cartan matrix $A=(a_{ij})_{1\leq i,j \leq n}$ such that
		\begin{align}
			q_{ij}q_{ji}= q_{ii}^{q_{ij}} \qquad \text{ for } 1\leq i,j\leq n.
		\end{align}
	\end{definition}
	 If a generalised Cartan matrix implements the braiding we choose it such that 
	 $a_{ij} \neq ord(q_{ii})$ and $a_{ij}$ maximal.

	\begin{lemma}
		Let $(V,q)$ ne of Cartan type with generalised Cartan matrix $A$ associated to the braiding.
		Then $A$ induces a set of reflection on $\subset k^n$ via 
		\begin{align}
			s_i(\delta_j)= e_j - a_{ij}e_i
		\end{align}
		The group generated by $\{s_i \mid 1 \leq i \leq n\}$ is called the Weyl group $W$ associated to $A$.
		
		If $A$ is symmetrizable and indecomposable and finite the associated Weyl group is finite.
	\end{lemma}

	We will concentrate on the case of Nichols algebras of Cartan type whose associated Cartan matrix is symmetrizable(, indecomposable) and finite.
	
	\begin{definition}
		Given a positive root system $\Delta^+$  a convex order is defined to be a total order $<$ on $\Delta^+$ such that, if
		$a<b$ and $a+b \in \Delta^+$ we have $a< a+b< b$.
	\end{definition}
	
	\begin{lemma}
		Let $(V,q)$ be a braided vector space of Cartan type sand let $\Delta^+$ denote the set of positive roots associated to $\Nichols V$. Assume that $\Delta^+$ is finite (i.e.) that $\Nichols B$ is arithmetic then $\Nichols B$ is generated by $v_1, \dotsc v_n$ and the relations
		\begin{align}
			[v_i,v_j]c &= \sum_{n_{i+1},\dotsc, n_{j-1} \in \mathbb N_+} c_{n_{i+1},\dotsc, n_{j-1}} v_{j-1}^{n_{j-1}} \dotsc v_{i+1}^{n_{i+1}}, \quad v_i < v_j \in \Delta^+, \\
			v_i^{N_\beta} &= 0,
		\end{align}
		where $c_{n_{i+1},\dotsc, n_{j-1}} =0 $ if $n_{i+1}v_{i+1} + \dotsc n_{j-1} v_{j-1} \neq v_i + v_j$.
	\end{lemma}

	\paragraph{Outline} To obtain a projective resolution of a Nichols algebra of diagonal type $\Nichols(V)$ we proceed as follows:
	\begin{enumerate}
		\item We start with a braided vector space $(V,q)$ od dimension $n$
		\item Associate to it the pre-Nichols algebra $\Nichols{V}'$ (same relations except unrestricted heights of the PBW- generators)
		\item The PBW-generators (viewed as vectors in $\mathbb R^n$) form a root system $\Delta^+$
		For a positive root $\mu$ we write $x_\mu$ for the corresponding generator of the $PBW$-basis.
		\item We denote by $(W,S)$ the Weyl-group generated by the (reflections along) simple positive roots $S \subset \Delta^+$ (i.e. the generators of the Nichols algebra).
		\item Every element $w\in W$ has a well-defined length $l(w)$ - its minimal representation in terms of letters in $S$. We obtain a decomposition of $W$ in terms of the length function, i.e. 
		$W = \cup_{k=0}^m W_k$ with $W_k:= \{ w \in W \mid l(w)=k\}$.
		\item We claim that we obtain a projective resolution of the trivial $\Nichols V'$ module via 
		\begin{center}
			\begin{tikzcd}
				0 
				\arrow{r}& 
				P_m 
				\arrow{r}{d_m}&
				P_{m-1}
				\arrow{r}{d_{m-1}} &
				\dotsc
				\arrow{r}{d_1} &
				P_0 
				\arrow{r} & 
				k \arrow {r} & 
				0
			\end{tikzcd}
		\end{center}
		where  $P_j = \oplus_{w\in W_j }\Nichols V'$ and $d_j= \sum_{o \in W_j, p\in W_{j-1}} \iota_p d_k^{(o,p)}\rho_o$ where $\iota_p$ denotes the identification of $\Nichols V'$ wth the $p-th$ copy of itself inside of $P_{j-1}$ and $\rho_o$ the projection of $P_j$ onto its $o$-th component identified with $\Nichols V'$ and
		$$
			d_j^{(o,p)}: \Nichols V' \rightarrow \Nichols V', d_j^{(o,p)} = 
			\left\{ \begin{aligned}
				&\cdot \xi x_\mu^a  	& \text{with } \xi \in \mathbb C  \text{  and } \mu \in \Delta^+ \text{s.t. } o \tau_\mu =p \\
				&0 				&\text{else}
			\end{aligned} \right.
		$$
	\end{enumerate}
	Sketch of the proof:
	\begin{enumerate}
		\item We consider the Bruhat order on $(W,S)$.
			An arrow wrt to the bruhat order is a  pair $(o,p)\in W$ such that 
			$l(o)-1 = l(p)$ and there exists a transposition $t\in T\subset W$ such that $ot=p$.
			(Note that these arrows are in 1-to-1 correspondence with the non-zero $d^{(o,p)}_j$ maps.) 
		\item To obtain the integers $a \in \mathbb N$ we construct from the Bruhat order the Bruhat polytope.
		\item The Bruhat polytope $\mathcal P$ consists of a set of vertices $\{v_w \mid w \in W\}\subset \mathbb R^n$ and a set $E$ of (directed) straight lines such that there is an straight line $s$ between $v_o,v_p \in V$ if and only if there is an arrow $\tau_\mu$ wrt the Bruhat order between $o,p$ and
		$s$ is (up to translation) an itegral multiple of the root $\mu$ implementing the arrow between $o$ and $p$.
		\item Conjecture: The Bruhat polytope (as the minimal polytope satisfying this property) is uniquely defined by these building rules and does always exist.
		\item Given an arrow $\tau_\mu: o \rightarrow p$ in the Bruhat order we associate to it the integer $a$ which is determined as the unique integer which implements the line $a\mu$ between $o$ and $p$ in the Bruhat polytope.		
		\item Conjecture: Given two elements $u,w\in W$ such that $l(u)+2= l(w)$ there exist exactly $2$ or $0$ pairs of arrows	$w \rightarrow v \rightarrow u$. If two such arrows exist we call this a square in the Bruhat order.
		\item Conjecture: Given a square $(u,v,v',w)$ in the Bruhat the corresponding (closed) polygon chain in the Bruhat polytope is contained in a (affine) $2d-$ plane.
		\item Conjecture There exist scalars such that the maps in the complex corresponding to a square in the Bruhat order are zero (i.e. the complex is a chain complex.)
	\end{enumerate}


	
	
	\section{A projective resolution for generalised Taft algebras}
	
	In this section the construction of a projective resolution of generalised Taft algebras is reviewed.
	It is done as conceptually as possible in order to guide towards a more general principle.
	
	\subsection{Projective resolutions of Nichols algebras}
	
	In this section we discuss a projective resolution for the above mentioned Nichols algebras $\Nichols V$ and `lift' this to a projective resolution of the bosonisation.
	
	Let us start by defining the algebra $X$ which is generated by $v_1$ and $v_2$  and the relation
	$$
	v_1 v_2 = p v_2 v_1, \qquad \text{where } p:= q^{a_1 b_2}.
	$$
	Note that $X$ sits in the following exact sequence $T(V) \twoheadrightarrow X\twoheadrightarrow \Nichols V$ of graded vector spaces.
	
	We define a projective resolution of $k$ as an $X_1$-module via
	\begin{center}
		\begin{tikzcd}
		0
		\arrow[r]
		& X_1
		\arrow[r, "f_2"]
		&
		X_1^2
		\arrow[r, "f_1"]
		& 
		X_1 \arrow[r, "\varepsilon"]
		&
		k \arrow[r]
		& 
		0,
		\end{tikzcd}
	\end{center}
	with the morphisms
	\begin{align*}
	f_2: & X_1 \rightarrow X_1^2,  && x \mapsto
	x\begin{pmatrix}
	p v_2 \\
	- v_1
	\end{pmatrix}
	\\
	f_1: & X_1^2\rightarrow X_1,  && \begin{pmatrix} x \\ y\end{pmatrix} \mapsto
	x v_1 + y v_2
	\end{align*}
	Let us give a graphical representation of this resolution that will be illustrative of what we will do next.
	\begin{center}
		\begin{tikzcd}
		&& X_1 \arrow[rd, "\cdot v_1"]
		\\
		0
		\arrow[r]
		& X_1
		\arrow[ru, "\cdot pv_2"]
		\arrow[rd, "\cdot(-v_1)"']
		&& 
		X_1 \arrow[r, "\varepsilon"]
		&
		k \arrow[r]
		& 
		0,
		\\
		&& X_1 \arrow[ru,"\cdot v_2"']
		\end{tikzcd}
	\end{center}
	where we read each column as a direct sum \footnote{
		\begin{conjecture}
			For these types of algebras (i.e.) disconnected Dynkin diagrams we get resolution that look like Hasse-diagrams of partitions.
			
		\end{conjecture} 
	}.
	To get a resolution of $\Nichols V$, we stitch this resolution together with  the relations $v_1^{N_1} = v_2^{N_2} =0$. This yields
	\begin{center}
		\begin{tikzcd}
		\mathcal{B}(V_1) \arrow[rd, "\cdot v_1^{N_1-1}"]
		\\
		&
		\mathcal{B}(V_1) \arrow[rd,"\cdot v_1"]
		\\
		\mathcal{B}(V_1) \arrow[ru, "\cdot p v_2"] \arrow [rd,"\cdot(-v_1)"']		&&
		\mathcal{B}(V_1) \arrow[rd,"\cdot v_1^{N_{1}-1}"]
		\\
		&
		\mathcal{B}(V_1) \arrow[rd,"\cdot (-v_1^{N_{1}-1})"'] \arrow[ru,"\cdot v_2"]	&&\mathcal{B}(V_1) \arrow[rd,"\cdot v_1"] 
		\\
		\mathcal{B}(V_1) \arrow[ru, "\cdot v_2^{N_2-1}"] \arrow[rd,"\cdot v_1^{N_1-1}"']		&&
		\mathcal{B}(V_1) \arrow[ru,"\cdot p v_2"] \arrow[rd,"\cdot(-v_1)"'] 	&& 
		\mathcal{B}(V_1) \arrow[r,"\varepsilon"]  & 
		k \arrow[r]& 
		0.
		\\
		& 
		\mathcal{B}(V_1)\arrow[rd,"\cdot v_1"'] \arrow[ru,"\cdot p^{-1}v_2^{N_2-1}"]&& \mathcal{B}(V_1) \arrow[ru,"\cdot v_2"']
		\\
		\mathcal{B}(V_1)\arrow[ru, "\cdot pv_2"] \arrow[rd,"\cdot(-v_1)"']		&&
		\mathcal{B}(V_1) \arrow[ru,"\cdot v_2^{N_{2}-1}"']
		\\
		&
		\mathcal{B}(V_1) \arrow[ru,"\cdot v_2"']
		\\
		\mathcal{B}(V_1) \arrow[ru,"\cdot v_2^{N_2-1}"']
		\end{tikzcd}
	\end{center}
	Notice that the resolution of $X$ still appears as `tiles' in this resolution.
	
	\textbf{Question:} How can we generalise this `stitching' process? Gut feeling is to look at the canonical long exact sequence in homology. 
	
	
	\subsection{Lifting resolutions from the Nichols algebra to its bosonisation}
	
	Let $H$ be Hopf algebra (over any field $k$) and $R$ a braided Hopf algebra in $\YD H$. Denote by $\YD{H}$ the category of $H$-Yetter-Drinfeld modules and by $R$-$H$-Mod the category of modules over the ($H$-module algebra) $R$. Finally, denote by 
	$R$-$\YD{H}$ the subcategory of $R$-modules in $\YD{H}$. That is, objects of 
	$R$-$\YD{H}$ are pairs $(\mathbb M,\blacktriangleright)$ comprising a $H$-Yetter-Drinfeld module $\mathbb M := (M,\triangleright,\delta)$ and a $H$-linear, $H$-colinear map $\blacktriangleright: R \otimes M \rightarrow M$, which implements an action of $R$ on $\mathbb M$.
	
	Similarly objects in $R$-$H$-Mod are pairs $(\mathbb M,\blacktriangleright)$ comprising a $H$-module $\mathbb M := (M,\triangleright)$ and a $H$-linear map $\blacktriangleright: R \otimes M \rightarrow M$, which implements an action of $R$ on $\mathbb M$.
	Note that there is a forgetful functor $R$-$\YD H\rightarrow R$-$H$-Mod.
	This functor is exact (since kernels and cokernels come from the kernels and cokernels of the underlying $k$-linear maps). 
	
	
	\begin{lemma}
		Let $H$ be a f.d. Hopf algebra such that $\text{tr } S^2 \neq 0$. TFAE
		\begin{enumerate}
			\item $H$ is semisimple.
			\item $D(H)$ is semisimple.
			\item The trivial $H$-module $k_\varepsilon$ is projective.
			\item The trivial $D(H)$-module $k_\varepsilon$ is projective.
		\end{enumerate}
	\end{lemma}
	\begin{proof}
		We prove $(1) \Leftrightarrow (3)$ and $(2) \Leftrightarrow (4)$:
		Assume that $L$ is some f. d.  semisimple Hopf algebra. By Maschke's theorem, there exists a normalised integral $\Lambda \in L$ (i.e. $\varepsilon(\Lambda) =1$). 
		One shows that $L:= \Lambda L \oplus (1-\Lambda)L$ thus $\Lambda L$ is projective.  Moreover $\Lambda L\cong k_\varepsilon$ implying that the trivial module is projective.
		Conversely, assume $k_\varepsilon$ to be projective.
		Now consider the surjective morphism of $L$ modules $\varepsilon: L \rightarrow k_\varepsilon$ . The projectivity implies that there exists a morphism of $L$ modules $\xi: k_\varepsilon \rightarrow L$ such that $\varepsilon \xi = id_k$. Define $\Lambda:= \xi(1)$. By 
		definition we have $\varepsilon(\Lambda) = \varepsilon\xi (1)= 1$.
		Moreover for any $l\in L$ we have $l \Lambda = l \xi(1) = \xi (l\triangleright 1) = \varepsilon(l) \Lambda$. 
		
		The equivalence between $(1)$ and $(2)$ follows from the following consideration.
		If $H$ is semisimple and $\text{tr } S^2 \neq 0$, its dual is semisimple. One then shows, using integrals, that $D(H)$ is semisimple. Conversely, if $D(H)$ is semisimple, there exists a non-trivial integral $\mu \in D(H)$.  Define $\Lambda:= (\varepsilon \otimes id)(\mu) \in H$. As $\varepsilon(\Lambda)= \varepsilon\otimes \varepsilon (\mu)=1$, we have $\Lambda \neq 0$. Moreover, a direct computation shows that it is a (non-trivial) integral for $H$. By Maschke's theorem, we then know that $H$ is semisimple.
		
	\end{proof}
	
	\begin{lemma}
		Let $H$ be finite dimensional and semisimple.The algebra $R$ is projective in $R$-$\YD H$ (and in $R$-$H$-Mod).
	\end{lemma}
	\begin{proof}
		We show that $R$ is projective in $\YD H$-Mod.
		There is a natural isomorphism 
		$\Hom_{R\text{-}\YD H}(R, -) = \Hom_{R\text{-}\YD H}(R\otimes k_\varepsilon, -) = \Hom_{\YD H}(k_\varepsilon, -)$ coming from an adjoint pair of functors, i.e., ``free" and ``forgetful" functors.
		As $k_\varepsilon$ is projective in $\YD H$ by the previous lemma,  $Hom_{\YD H}(k_\varepsilon, -)$ is exact, implying that $\Hom_{R-\YD H}(R, -)$ is exact. Hence, $R$ is projective in {{$R\text{-}\YD H$-Mod}}.
	\end{proof}
	
	\begin{lemma}
		Let $H$ f.d. and semisimple and let $M$ be a projective object in $R\text{-}\YD H$-Mod. Then there exists a $Q\in R\text{-}\YD H$-Mod, such that $M \oplus Q  = \oplus_{i \in I} R$.
	\end{lemma}
	\begin{proof}
		Consider the free $R$-module generated by $M$, $R^M:= \oplus_{m \in M} R$
		Let $\pi: R^M \rightarrow M$, $r_m \rightarrow r\triangleright m$ . Since $M$ is projective there exists a map $\iota: M \rightarrow R^M$, such that $\pi \iota  = id_M$.
		We claim that $R^M = \text{im } \iota \oplus \ker \pi$.
		Let $r \in R^M$ and write $r= s + t$, with $s= \iota \pi (r)$ and $t= r-s$.  Then we have $s \in \text{im } \iota$ and $\pi(t) = \pi(r) - \pi \iota \pi(r)= \pi(r) -\pi(r) = 0$. Thus $t \in \ker \pi$.
		To see that the sum is direct, assume that there exists some $x \in R^M$ such that $x\in \ker \pi$ and $x \in \text{im } \iota$.  As $x \in \text{im } \iota$, there exists $y \in M $ such that $\iota y= x$.
		As $y= \pi\iota (y) = \pi(x) =0$, it follows that $x=0$.
	\end{proof}
	
	\begin{theorem}
		Let $H$ be a finite dimensional semisimple Hopf algebra  such that $\text{tr } S^2 \neq 0$ and let $R$ be a braided Hopf algebra over $H$. Write $\Lambda$ for the unique left integral in $H$ such that $\varepsilon(\Lambda)=1$.
		For any $H$-module $N$, the following constitutes an exact functor:
		\begin{align}
		\begin{aligned}
		F_N\colon &R\text{-}\YD{H} \rightarrow R\#H\text{-}\text{Mod} \\
		& ( M,\blacktriangleright) \mapsto 
		\left(M\#N\right)^\Lambda:=((R\# \Lambda)\triangleright(M\#N),\triangleright)\\
		&f \mapsto \left(f\# id_N\right) |_{	\left(M\#N\right)^\Lambda},
		\end{aligned}
		\end{align}
		with $\triangleright$ given by $\left(r\#h\right) \triangleright \left(m\#n\right) :=  
		r\blacktriangleright(h_{(1)}\triangleright m) \otimes h_{(2)}\triangleright n$, for ${r \in R}, {h \in H}, {m\in M}$ and  $n \in N$.
		
		Moreover, if $N$ is such that there exists a surjective $H$-linear map $N \rightarrow k$, then $F_N$ preserves projective modules.
	\end{theorem}
	\begin{proof}
		To prove that $F_N$ is well-defined, we first define the functor
		\begin{align}
		\begin{aligned}
		F_N'\colon &R\text{-}\YD{H} \rightarrow R\#H\text{-}\text{Mod} \\
		& (\mathbb M,\blacktriangleright) \mapsto 
		M\#N:=(M\otimes N,\triangleright)\\
		&f \mapsto f\# id_N:= f\otimes id_N.
		\end{aligned}
		\end{align}
		where $\triangleright$ is given by $\left(r\#h\right) \triangleright \left(m\#n\right) :=  
		r\blacktriangleright(h_{(1)}\triangleright m) \otimes h_{(2)}\triangleright n$.
		
		Let $M$ be in $R\text{-}\YD H$. We compute, for $r,s \in R$, $g,h,\in H$, $m\in M$ and $n\in N$
		\begin{align*}
		(r\#g)&\triangleright((s\#h) \triangleright(m\# n))
		=(r\# g) \triangleright (s\blacktriangleright (h_ {(1)}\triangleright m) \# h_{(2)} \triangleright n) \\
		& = r\blacktriangleright( g_{(1)}\triangleright(s\blacktriangleright (h_{(1)}\triangleright m))) \#  g_{(2)}h_{(2)}\triangleright n \\
		& =  r\blacktriangleright((g_{(1)}\triangleright s)\blacktriangleright (g_{(2)} h_{(1)}\triangleright m)) \#  g_{(3)}h_{(2)}\triangleright n \\
		& =  (r(g_{(1)}\triangleright s))\blacktriangleright (g_{(2)(1)} h_{(1)}\triangleright m) \#  g_{(2)(2)}h_{(2)}\triangleright n \\
		& = (r(g_{(1)}\triangleright s)\# g_{(2)}h)\triangleright (m\#n) \\
		& = (r\# g)(s \# g) \triangleright (m \# n).
		\end{align*}
		This shows that $(F_N'(M),\triangleright)$ is a $R\#H\text{-}$module. Moreover, one checks that the action is unital. 	We also compute, given a morphism $f: M \rightarrow  M'$ in $R\text{-}\YD H$,
		\begin{align*}
		(f\#\text{id}_N)((r\#g) \triangleright (m \# n)) &= (f\#\text{id}_N)(r\blacktriangleright(g_{(1)}\triangleright m) \otimes g_{(2)}\triangleright n) \\
		& = r\blacktriangleright(g_{(1)}\triangleright f(m)) \otimes g_{(2)}\triangleright n \\
		&  = \left(r\#g\right) \triangleright (f\#\text{id}_N)(m\#n).
		\end{align*}
		Hence, we conclude that the functor $F_N'$ is well-defined.
		
		Next, define the endofunctor
		\begin{align}
		\begin{aligned}
		(-)^\Lambda \colon &R\# H\text{-Mod} \rightarrow R\#H\text{-Mod} \\
		&X \mapsto X^\Lambda :=(R\#\Lambda) \triangleright X \\
		&f:X\rightarrow Y \mapsto f|_{X^\Lambda} \colon X^\Lambda \mapsto Y^\Lambda
		\end{aligned}
		\end{align}
		
		We compute, for some $x\in X$, $r,s\in R$ and $g\in H$,
		$$
		(r\#g)\triangleright ((s\#\Lambda) \triangleright x)
		= (r(g\triangleright s)\# \Lambda)\triangleright x
		$$
		and further, for a morphism $f \colon X \to Y$ in $R\#H$-Mod,
		$$
		f((s\# \Lambda)\triangleright x) = (s\# \Lambda) f(x),
		$$
		hence, $\text{im } f|_{X^\Lambda} \subseteq Y^\Lambda$ and thus, $(-)^\Lambda$ is well defined and we conclude that so is $F_N$, as $F_N= (-)^\Lambda  \circ F_N' $.
		
		To prove that $F_N$ preserves exactness, note that $F_N'$ does, since on the level of  morphisms $F_N'(f)= f\otimes \text{id}_N$ and the tensor product over $k$ is exact.
		By definition, $(-)^\Lambda(f)$ preserves kernels. Moreover, if  that ${f\colon X\rightarrow Y}$  is surjective, then $(-)^\Lambda(f) = f|_{X^\Lambda}$ is surjective too, since: if $y\in Y^\Lambda$, there is $r\in R$ such that $y= (r\#\Lambda) \triangleright y'$. 
		Letting $x'\in X$ such that $f(x') =y'$, then $f((r\# \Lambda) \triangleright x')=
		r\# \Lambda \triangleright y' = y$.	Therefore, we conclude that $(-)^\Lambda$ is exact and it follows that $F_N$ is exact too.
		
		It remains to see that $F_N$ preserves projectivity, if there exists a surjective $H$-linear map $\widehat \varepsilon: N \rightarrow k$.		 		
		Let $M$ be  projective in $R\text{-}\YD H$. Then there exists a $Q$ in $R\text{-}\YD H$ such that ${M \oplus Q \cong \oplus_{j\in J} R}$.
		Applying the forgetful functor $\text{Forg} \colon  {R\text{-}\YD H} \rightarrow R\text{-}H\text{-Mod}$, we see that $M$ is projective as an $R$-$H$-module.
		
		
		
		Fix a surjection $\beta: X \rightarrow Y$ of $R\#H$-modules and a morphism  $\gamma: F(M) \rightarrow Y$. As there are algebra inclusions of $R$ and $H$ in  $R\#H$, we have that $\beta$ and $\gamma$ are $H$- and $R$-linear.
		Thus, we can consider the following diagram in $R$-$H$-Mod:		
		\begin{center}
			\begin{tikzcd}
			& M 
			\arrow[d,dashed, bend left =10,"\lambda"] 
			\arrow[ddl, dashed, bend right =30 , "\alpha"]\\
			& F_N(M)  \arrow[d,"\gamma"] 
			\arrow[u, dashed, bend left =10,"\delta"] \\
			X \arrow[r,two heads, "\beta"]
			& Y
			\end{tikzcd},
		\end{center}
		where  $\delta: F_N(M) \rightarrow M$, $m \#  n \mapsto 
		\widehat \varepsilon (n)  m$. 
		IF(?) it is a surjective morphism of $R\text{-}H$-modules, there exists a section $\lambda$ since $M$ is projective in $R$-$H$-Mod.  Again by the projectivity of $M$, there exists $\alpha \colon M \to X$ in $R$-$H$-Mod which makes the outer triangle commute\footnote{
			IDEA for surjectivity.
			
			Let $Q$ be such that $M\oplus Q = \oplus_{i\in I} R$ consider the $\YD H$ map $\eta: k \rightarrow \oplus_{i\in I}$, $1_k \mapsto (1,0,...,0)$. 
			Let $\pi: \oplus R \rightarrow M$ be the projection onto  $M$ and write $\eta' = \pi \circ \eta$.
			Now let $n\in N$ be such that $\hat \varepsilon(n) =1$ and choose some $m\in M$.
			Then $m = \pi(r_1,\dotsc r_k)$. Let wlog $r_2\dotsc r_k =0$. Then consider the element
			$m'= \eta'(1) \in M$. We compute
			$$
			\delta( r_1\Lambda_{(1)}m' \# \Lambda_{(2)}n ) = \varepsilon(\Lambda n) r_1 m' = m.
			$$
		}.
		We show that the composition $\alpha \delta$ extends in fact to a morphism of $R\# H$-modules. We compute 
		\begin{align*}
		(\alpha \delta) ((r\#g) \triangleright (s \blacktriangleright (\Lambda_{(1)} \triangleright m)\#\Lambda_{(2)}n)) 
		& = (\alpha \delta) (r(g_{(1)} \triangleright s) \blacktriangleright ((g_{(2)}\Lambda_{(1)})\triangleright m) \# g_{(3)} \Lambda_{(2)} n)	\\
		&= \widehat \varepsilon(n) r(g \triangleright s) \blacktriangleright(\Lambda \triangleright \alpha (m)) \\
		%			& = \widehat \varepsilon(n) (r(g \triangleright s)\# 1)(1 \# \Lambda) \triangleright \alpha(m) \\
		& = \widehat\varepsilon(n)  (r(g \triangleright s)\# \Lambda) \triangleright \alpha (m)).
		\end{align*}
		and
		\begin{align*}
		(r\#g)  \triangleright ((\alpha \delta)(s \blacktriangleright (\Lambda_{(1)} \triangleright m)\#\Lambda_{(2)}n))
		& = \widehat\varepsilon(n) (r\#g)  \triangleright (s \blacktriangleright (\Lambda \triangleright \alpha(m) )) \\
		& = \widehat\varepsilon(n)  (r\# g)(s\# \Lambda) \triangleright \alpha (m)	 \\
		& = \widehat\varepsilon(n)  (r(g \triangleright s)\# \Lambda) \triangleright \alpha (m)).
		\end{align*}
		Since $\lambda \delta = \text{id}_M$ and $\beta \alpha =  \gamma\lambda$, we have
		$\beta \alpha\delta =  \gamma \lambda \delta = \gamma$.
		Thus, $F_N(M)$ is projective.
	\end{proof}
	
	
	---TODO:CHECK WHETHER THE ARGUMENTS REALLY HOLD AND HOW THEY CAN BE GENERALISED ---
	
	\begin{theorem}
		Let $H$ be a fd semisimple Hopf algebra over some field $k$ such that $\text{tr }S^2 \neq 0$ and $R$ a braided Hopf algebra over $H$. 
		Write $\Lambda$ for the unique left integral of $H$ such that $\varepsilon(\Lambda) =1$.
		Given any projective resolution $(P_n, d_n)$ of $k$ viewed as the trivial $R$ module and any simple $H$ module $N$ we obtain a projective resolution $F_N(P_n, d_n)$ of the trivial $R\#H$-module $k$.
	\end{theorem}
	\begin{proof}
		As simplicity implies projectivity $N$ is in particular projective.
		Choose any $n\in N$ such that $Hn =N$, and define $\widehat \varepsilon(g n):=\varepsilon(g)$. This is a morphism of $H$ modules and  as $\widehat \varepsilon(n) \neq 0$ it is surjective. 
		By the previous theorem we know that $F_N(P_n,d_n)$ is a projective resolution of $F_N(k\varepsilon)$. 
		Let us therefore study $F_N(k_\varepsilon)$
		As 
		$
		\widehat \varepsilon(\Lambda \triangleright n)= \varepsilon(\Lambda) \widehat \varepsilon(n) = \varepsilon(n) \neq 0.
		$
		we have $\Lambda \triangleright n \neq 0$.
		Now consider any $n' \in N$. By definition of $n$ there exists a $g\in H$ such that $n' = g \triangleright n$ and we have
		$\Lambda \triangleright n' = \Lambda g \triangleright n = \alpha (g) \Lambda \triangleright n$, where $\alpha \in H^*$ is the distinguished group-like of $H^*$.  Thus $\Lambda \triangleright N \cong k_\varepsilon$.
		
		Note: We did not use simplicity but being generated by one element for the most part. Does being generated by one element imply projectivity?
	\end{proof}
	
	
	SKETCH  TO FURTHER THIS PROJECT:
	Let $(V,c)$ be a braided vector space of diagonal type with $q$ as matrix of the braiding 
	Alter the diagonal entries of $q$ such that the newly obtained matrix $q'$ is of Cartan type.
	Denote by $V'$ the corresponding braided  vector space. Produce a proj res of the trivial module $\Nichols {V'}$ show that it can be turned into a projective resolution  of the trivial $\Nichols V$ module.
	Use the above thm to obtain a proj res of the trivial module over the bosonisation.
	
	
	\begin{enumerate}
		\item Let $\mathfrak q$ be the matrix of a braiding. Associate to it the generalised Cartan matrix $C$ as constructed in Definition 2.22 in the Andrus-Angiono survey.
		\item BGG resolution is build around Verma module. We need to understand those in our setting. This we should do on the level of a Pre-Nichols algebra $\tilde B(V)$.
		\item There exists a short exact sequence of (?) $k$-vector spaces
		$$
			0 \rightarrow X \rightarrow \tilde B(V) \rightarrow \Nichols V \rightarrow 0.
		$$
		Where $X$ is generated by the set $x_1^{N_1} \dotsc x_n^{N_n}$.
		Assume we have a nice resolution of the algebras $X$ (coming from $\tilde B(V)$ ?) and $\tilde B(V)$ (here should be the BGG part) we obtain a long exact sequence of homology groups via the snake lemma.
	\end{enumerate}



	Idea: Consider a braided vector space $(V,c)$, such that $\Nichols V$ is of Cartan type.
	Consider its root system and denote by $W$ its Weyl-group. Its Weyl group is a coxeter group generated by $S$, the set of reflections which is indexed by the elements constituting a PBW basis of $\Nichols V$.
	
	Let $w\in W$ be the longest word of $W$.
		
	\begin{conjecture}
		Let $w\in W$ as before and denote by $l(w)$ the length of $w$. 
		Then
		\begin{center}
		\begin{tikzcd}
			0 \arrow[r]&
			B \arrow[r, "d_{l-1}"]&
			\oplus_{w_1} B &
			\dotsc
		\end{tikzcd}
		\end{center}
	Yields a free resolution of the pre-Nichols algebra of $V$.
	\end{conjecture}
	\begin{proof}
		Freeness is clear it is exactness we need to care about
		
	\end{proof}

	Idea: The distinguished pre-Nichols algebra plays the $n$-part in the triangular decompostion. Now  the $h$ part coms from a vector space which is (if its f.d) dual to a space containing the roots.
	So we should get it out of our Cartan matrix $C$.
	We should try to define the borel algebra $U(b)$ in terms of $h$ and the distinguished Nichols algebra.
	\begin{lemma}
		Let $(V,c)$ be a finite-dimensional braided vector space of diagonal type with matrix of the brading $(q_{ij})_{1 \leq i,j \leq \theta}$.
		Let $G$ be any group such that $(V,c)$ is a Yetter-Drinfeld module over $kG$.
		Then
		\begin{itemize}
			\item There exists a Yetter-Drinfeld module $(V',c')$ over $G$ such that
			the matrix of the braiding $q'$ satisfies:
			$q_{ij} = q'_{ij}$ for $i \neq j$ and $q'$ is of Cartan type i.e. there exists a generalised Cartan matrix $A$ such that $q'_{ij}= q{ii}^{A_{ij}}$.
			\item There exists (unique?) pre-Nichols algebras $B(V)$ and $B(V')$ such that $B(V)$ and $B(V')$ are isomorphic as $k$-algebras. 
		\end{itemize}		
	\end{lemma}
	
	\begin{proof}
		Consider the matrix of the braiding $q_{ij}$ and let $Q_i:=\{q_{ij}|1 \leq j \leq \theta\}$.
		As $Q_i$ generates acyclic subgroup of $k^\times$ there exists some generator $x_{i}$. Now set $q'_{ij}= q_{ij}$ if $i\neq j$ and $q'_{ii}=x_i$ and fix 
		$A_{ij}= \max{n\in -\mathbb N | x_i^n=q_{ij} }$ if $i \neq j$ and  $A_{ii}=2$. This is by construction a generalised Cartan matrix.
		Moreover $(V',c')$ is constructed in the obvious way.
	\end{proof}

\section{Coxeter groups}

\begin{definition}
	Let $S$ be a set of order $N>1$ and $M:=(m_{ij})$ an integral $N\times N$ matrix such that
	\begin{enumerate}
		\item $m_{ii}=1$
		\item $m_{ij} = m_{ji} \geq 2 $ for $i \neq j$.
	\end{enumerate} 
	The group $W$ is the quotient of the free group generated by $S$ by the relations
	$(s_i s_j)^{m_{ij}}=e$. The pair $(W, S)$ is called a Coxeter system.
	
	A reflection in $W$ is an element $t:= wsw^{-1}\in W$, with $w \in W$ and $s\in S$. We write $T$ for the set of reflections of $W$. 
\end{definition}

\begin{definition}
	Let $(W,S)$ be a Coxeter system and $s_1\dotsc s_n \in W$. Write
	$t_{n-i}:= s_n s_{n-1} \dotsc s_{n-i} \dotsc s_n \in T$ for $1 \leq i \leq n$. The inversion order of $s_1\dotsc s_k$ is given by the tuple $(t_n, \dotsc , t_1)$.
\end{definition}

\begin{remark}
	It holds that
	$$
		s_1 \dotsc s_n t_{n-i} = s_1 \dotsc s_{i-1} s_{i+1} \dotsc s_n.
	$$
	and
	$$
		s_1 \dotsc s_i = t_i \dotsc t_1.
	$$	
	If $w= s_1\dotsc s_k$ with $k$ minimal then $t_i \neq t_j$ for $i\neq j$.
\end{remark}
	
	
\begin{definition}
	Let $(W,S)$ be a Coxeter system.
	We define the length of an element $w\in W$ via
	$$
		l_S(w)= \min_{k\in \mathbb N}\{ s_1 \dotsc s_k= w \mid s_1,\dotsc s_k \in S\}.
	$$
	We call a word $s_1\dotsc s_k$ reduced if it is of minimal length.
\end{definition}

\begin{remark}
	It holds that
	$$
		l_S(u)+l_S(v) \geq l_S(uv), \qquad l_S(u) = l_S (u^{-1}).
	$$
\end{remark}

\begin{definition}
	Consider a Coxeter system $(W,S)$ and $u,v \in W$.
	We write
	\begin{enumerate}
		\item $u \leftarrow v$ iff $l_S(u) +1= l_S(v)$ and there exists a $t\in T$ such that $ut=v$.
		\item $u\leq v$ if $u=u_0 \leftarrow u_1 \leftarrow \dotsc \leftarrow u_n =v$.
	\end{enumerate}
\end{definition}

\begin{definition}
	Let $(W,S)$ be a Coxeter system.
	Its Bruhat graph is the direct labeled graph which is defined by:
	\begin{enumerate}
		\item Its vertices are the elements of $W$.
		\item There is an edge form $v$ to $u$ whenever $u \leftarrow v$. It is labeled by the transposition $t$ such that $ut=v$
	\end{enumerate}
\end{definition}

\begin{example}
	Consider the Coxeter-group $S_4$ with three generators $a,b,c$ and relations
	$$
		(ab)^3=(bc)^3=(ac)^2 = e \quad\quad \text{ and } \quad\quad a^2 = b^2 = c^2 = e. 
	$$
	The 24 elements of $S_4$ are
	\begin{gather*}
		\begin{matrix}
			e 		&			&			&			&			& 		\\
			a 		& b 		& c 		&			&			&		\\
			ab 		& ac 		& ba		& bc		& cb 		&		\\
			aba		& abc		& acb		& bac		& bcb		& cba	\\
			abac	& abcb	 	& acba 		& bacb 		& bcba		&		\\
			abacb	& abcba		& bacba		&			&			&		\\
			abacba
		\end{matrix}
	\end{gather*}
	
	Let us now determine all edges in the bruhat graph:
	\begin{gather*}
		e \overset{a}\rightarrow a, \quad
		e \overset{b}\rightarrow b, \quad
		e \overset{c}\rightarrow c, \quad
		\\
		\\
		a \overset{b}\rightarrow ab, \quad
		a \overset{aba}\rightarrow ba, \quad
		a \overset{c}\rightarrow ac, \quad
		\\
		b \overset{a}\rightarrow ba, \quad
		b \overset{aba}\rightarrow ab, \quad
		b \overset{c}\rightarrow bc, \quad
		b \overset{bcb}\rightarrow cb, \quad
		\\
		c \overset{a}\rightarrow ac, \quad
		c \overset{b}\rightarrow cb, \quad
		c \overset{bcb}\rightarrow bc, \quad
		\\
		\\
		ab \overset{a}\rightarrow aba, \quad
		ab \overset{c}\rightarrow abc, \quad
		ab \overset{bcb}\rightarrow acb, \quad
		\\
		ac \overset{b}\rightarrow acb, \quad
		ac \overset{aba}\rightarrow cba, \quad
		ac \overset{bcb}\rightarrow abc, \quad
		ac \overset{abcba}\rightarrow bac, \quad
		\\
		bc \overset{abcba}\rightarrow abc, \quad
		bc \overset{a}\rightarrow bac, \quad
		bc \overset{b}\rightarrow cbc, \quad 
		\\
		cb \overset{aba}\rightarrow acb, \quad
		cb \overset{a}\rightarrow cba, \quad
		cb \overset{c}\rightarrow bcb, \quad 
	\end{gather*}
	The squares in this graph are
	
	\begin{center}
		\begin{tikzcd}
			&\circ
			\arrow{dr}{b}\\
			\circ 
			\arrow{ur}{a}
			\arrow{dr}{b}
			&& \circ \\
			& \circ 
			\arrow{ur}{aba}
			\\
			&\circ
			\arrow{dr}{aba}\\
			\circ 
			\arrow{ur}{a}
			\arrow{dr}{b}
			&& \circ \\
			& \circ 
			\arrow{ur}{a}
			\\
			&\circ
			\arrow{dr}{bcb}\\
			\circ 
			\arrow{ur}{a}
			\arrow{dr}{bcb}
			&& \circ \\
			& \circ 
			\arrow{ur}{abcba}\\
			&\circ
			\arrow{dr}{abcba}\\
			\circ 
			\arrow{ur}{aba}
			\arrow{dr}{c}
			&& \circ \\
			& \circ 
			\arrow{ur}{aba}
		\end{tikzcd}
		\begin{tikzcd}
			&\circ
			\arrow{dr}{c}\\
			\circ 
			\arrow{ur}{a}
			\arrow{dr}{c}
			&& \circ \\
			& \circ 
			\arrow{ur}{a}
			\\
			&\circ
			\arrow{dr}{c}\\
			\circ 
			\arrow{ur}{aba}
			\arrow{dr}{c}
			&& \circ \\
			& \circ 
			\arrow{ur}{abcba}\\
			&\circ
			\arrow{dr}{abcba}\\
			\circ 
			\arrow{ur}{a}
			\arrow{dr}{bcb}
			&& \circ \\
			& \circ 
			\arrow{ur}{a}
		\end{tikzcd}
		\begin{tikzcd}
			&\circ
			\arrow{dr}{c}\\
			\circ 
			\arrow{ur}{b}
			\arrow{dr}{c}
			&& \circ \\
			& \circ 
			\arrow{ur}{bcb}
			\\
			&\circ
			\arrow{dr}{bcb}\\
			\circ 
			\arrow{ur}{b}
			\arrow{dr}{c}
			&& \circ \\
			& \circ 
			\arrow{ur}{b}
			\\
			&\circ
			\arrow{dr}{bcb}\\
			\circ 
			\arrow{ur}{aba}
			\arrow{dr}{bcb}
			&& \circ \\
			& \circ 
			\arrow{ur}{aba}
		\end{tikzcd}
	\end{center}
	
	
	
	
	The Bruhat graph then is
	\begin{center}
		\begin{tikzcd}[column sep= tiny, row sep= 45pt]
			&&&&& abacba 
				\arrow[blue]{dll}[near start, description]{a}
				\arrow[red]{d}[near start, description]{b}
				\arrow[teal]{drr}[near start, description]{c}
			\\
			&&& abacb
				\arrow[red]{dll}[near start, description]{b}
				\arrow[orange]{d}[near start, description]{aba}
				\arrow[teal]{drr}[near end, description]{c}
			&& abcba
				\arrow[orange]{dllll}[near start, description]{aba}
				\arrow[blue]{dll}[near start, description]{a}
				\arrow[purple!70!black]{drrrr}[near end, description]{bcb}
				\arrow[teal]{drr}[near start, description]{c}
			&& bacba
				\arrow[blue]{dll}[near start, description]{a} 
				\arrow[red]{drr}[near start, description]{b}
				\arrow[purple!70!black]{d}[near start, description]{bcb}\\
			& abac
				\arrow[teal]{ddl}[near start, description]{c}
				\arrow[blue]{ddr}[near start, description]{a}
				\arrow[purple!70!black]{ddrrrrr}[near start, description]{bcb}
			&& abcb
				\arrow[red]{ddl}[near end, description]{b}
				\arrow[teal]{ddr}[near start, description]{c}
				\arrow[magenta!70!green]{ddrrrrrrr}[near start, description]{abcba}
			&& bacb 
				\arrow[purple!70!black]{ddlllll}[near start, description]{bcb}
				\arrow[magenta!70!green]{ddl}[near end, description]{abcba}
				\arrow[red]{ddr}[near end, description]{b}
				\arrow[orange]{ddrrrrr}[near start, description]{aba}
			&& acba 
				\arrow[magenta!70!green]{ddlllllll}[near end, description]{abcba}
				\arrow[blue]{ddlll}[near end, description]{a}
				\arrow[red]{ddr}[near start, description]{b}
			&& bcba
				\arrow[blue]{ddr}[near start, description]{a}
				\arrow[teal]{ddl}[near start, description]{c}
				\arrow[orange]{ddlll}[near start, description]{aba}
			\\ \\
			aba 
				\arrow[blue]{ddr}[near start, description]{a} 
				\arrow[red]{ddrrr}[near start, description]{b}
%				
			&& abc 
				\arrow[teal]{ddl}[near start, description]{c} 
				\arrow[purple!70!black]{ddrrr}[near start, description]{bcb}
				\arrow[magenta!70!green]{ddrrrrr}[near start, description]{abcba}
			&& acb
				\arrow[purple!70!black]{ddlll}[near end, description]{bcb}
				\arrow[red]{ddr}[near start, description]{b} 
				\arrow[orange]{ddrrrrr}[near end, description]{aba}
%				
		 	&& bac 
		 		\arrow[teal]{ddlll}[near start, description]{c} 
		 		\arrow[magenta!70!green]{ddl}[near start, description]{abcba}
		 		\arrow[blue]{ddr}[near start, description]{a}
		 	&& cba
		 		\arrow[blue]{ddr}[near start, description]{a} 
		 		\arrow[orange]{ddlll}[near start, description]{aba}
		 		\arrow[magenta!70!green]{ddlllll}[near start, description]{abcba}
		 	&& bcb 
		 		\arrow[red]{ddlll}[near start, description]{b} 
		 		\arrow[teal]{ddl}[near start, description]{c} \\ \\
			& ab 
				\arrow[orange]{drrrr}[near start, description]{aba}
				\arrow[red]{drr}[near start, description]{b}
			&& ba 
				\arrow[blue]{drr}[near start, description]{a}
				\arrow[orange]{d}[near start, description]{aba}
			&& ac 
				\arrow[teal]{dll}[near start, description]{c}
				\arrow[blue]{drr}[near start, description]{a}
			&& bc 
				\arrow[purple!70!black]{d}[near start, description]{bcb}
				\arrow[teal]{dll}[near start, description]{c}
			&& cb 
				\arrow[red]{dll}[near start, description]{b}
				\arrow[purple!70!black]{dllll}[near start, description]{bcb}\\
			&&& a 
				\arrow[blue]{drr}[near start, description]{a}
			&& b
				\arrow[red]{d}[near start, description]{b}
			 && c 
				\arrow[teal]{dll}[near start, description]{c}\\
			&&&&& e
		\end{tikzcd}
	\end{center}

	\begin{center}
		\begin{tikzpicture}[x=2cm, y=2cm, z=1cm]
		\tikzset{-Circle}
			% Axes
			\draw [->, dashed] (-4,0,0) -- (4,0,0) node [at end, right] {$x$};
			\draw [->, dashed] (0,-4,0) -- (0,4,0) node [at end, left] {$y$};
			\draw [->, dashed] (0,0,-4) -- (0,0,4) node [at end, left] {$z$};
			
			\coordinate (e) at (0,0,0);
			 
			\coordinate (a) at (1,0,0); 
			\coordinate (b) at (-0.5,  0.7071, -0.5); 
			\coordinate (c) at (0,0,1); 
			
			\coordinate (a+b) at (0.5,  0.7071, -0.5);
			\coordinate (a+b+c) at at (0.5,  0.7071, 0.5);
			\coordinate (b+c) at (-0.5,  0.7071, 0.5);
			
			\coordinate (ab) at ($(b)+(a+b)$);
			\coordinate (ba) at ($(a)+(a+b)$);
			\coordinate (ac) at ($(a)+(c)$);
			\coordinate (bc) at ($(c)+(b+c)$);
			\coordinate (cb) at ($(b)+(b+c)$);
			
			\coordinate (aba) at ($(ab)+(a)$);
			\coordinate (acb) at ($(ab)+(b+c)$);
			\coordinate (abc) at ($(bc)+(a+b+c)$);
			\coordinate (cba) at ($(ac)+(a+b)$);
			\coordinate (bac) at ($(ac)+(a+b+c)$);
			\coordinate (bcb) at ($(cb)+(c)$);
			
			
			\draw[blue] (e) -- (a) node [at end, right] {$a$};
			\draw[red] (e) -- (b) node [at end, right] {$b$};
			\draw[teal] (e) -- (c) node [at end, right] {$c$};
			
			\draw[blue!50!red] (b) -- (ab) node [at end, right] {$ab$};
			\draw[red] (a) -- ($(a)+(b)+(b)$);
			
			\draw[blue!50!red] (a) -- (ba) node [at end, right] {$ba$};
			\draw[blue] (b) -- ($(b)+(a)+(a)$);
			
			\draw[teal] (a) -- (ac) node [at end, right] {$ac$};
			\draw[blue] (c) -- ($(c)+(a)$);
			
			\draw[red!50!teal] (c) -- (bc) node [at end, right] {$bc$};
			\draw[teal] (b) -- ($(b)+(c)+(c)$);
			
			\draw[red!50!teal] (b) -- (cb) node [at end, right] {$cb$};
			\draw[red] (c) -- ($(c)+(b)+(b)$);
			
			\draw[blue] (ab) -- (aba) node [at end, right] {$aba$};
			\draw[red] (ba) -- ($(ba)+(b)$);
			
			\draw[red!50!teal] (ab) -- (acb) node [at end, right] {$acb$};
			\draw[red] (ac) -- ($(ac)+3*(b)$);
			\draw[blue!50!red] (cb) -- ($(cb)+(a+b)$);
			
			\draw[orange] (bc) -- (abc) node [at end, right] {$abc$};
			\draw[teal] (ab) -- ($(ab)+3*(c)$);
			\draw[red!50!teal] (ac) -- ($(ac)+2*(b+c)$);
			
			
			
%			\draw (a) -- ($(a)+(a+b+c)$)node [at end, right] {$x$};
		\end{tikzpicture}
	\end{center}
\end{example}



%\begin{definition}
%	Given any $w \in W$ we define the inversion set and descending set of $w$ via
%	$$
%		Inv(w) = \{t \in T | l_S(tw) < l_S(w) \}, \qquad Des(w) = Inv(w) \cap S.
%	$$
%\end{definition}
%
%\begin{theorem}
%	Given a Coxeter system $(W,S)$ and a reduced word $s_1 \dotsc s_i$ we have
%	$t\in Inv(w)$ iff $t= s_1\dotsc s_{i-1} s_i s_{i-1} \dotsc s_1$.
%\end{theorem}
%\begin{proof}
%	$\Leftarrow$ is clear.
%	$\Rightarrow$ is a bit more involved, see below.
%\end{proof}
%
%\begin{definition}
%	The root system assigned to $(W,S)$ is the set $R:= T \times\{\pm 1\}$.
%\end{definition}
%
%\begin{definition}
%	Given any word $s_1 \dotsc s_k$ and any reflection $t\in T$ we define 
%	$$
%		\#(s_1\dotsc s_k ; t):= \{ 1\leq i \leq k \mid t_i = t\}, \qquad \eta(s_1\dotsc s_k; t) := (-1)^{\# (s_1\dotsc s_k ;t)}.
%	$$.
%\end{definition}
%
%\begin{lemma}
%	There is a unique embedding of groups $\pi: W \rightarrow Sym(R)$ such that
%	$\pi_s((t,\epsilon)) = (sts, \epsilon \eta (s,t))$.
%\end{lemma}
%\begin{proof}
%	We first check that $\pi$ is well defined on $S$.
%	Note that $\eta(s, sts)= \eta(s,t))$ thus
%	$$
%		\pi_s^2(t,\epsilon) = \pi_s(sts, \epsilon \eta(s,t)) =(t, \epsilon).
%	$$
%	Now consider $u, v\in S$ and let $p:= \min_{n\geq 1}\{(uv)^n = 0\}$.
%	Observe that $(uv)^p$ gives rise to a word of length $2p$ with alternating letters $u$ and $v$ and any $t_i \in T((uv)^p)$ is of the form $(uv)^{i-1}u$.
%	Moreover $(uv)^p=0$ implies $t_{i+p} =(uv)^{p+i-1}u =(uv)^{i-1}u = t_i$ for $1\leq i \leq p$, implying that $\eta((uv)^p; t)\in 2 \mathbb N$.
%	Thus, by the universal property of the quotient, 
%	$\pi$ is a morphism of groups.
%	
%	To see that $\pi$ is injective consider any reduced word $W:=s_1\dotsc s_k$.
%	Fix an element $(s_1,\epsilon) \in R$. As $\#(s_1\dotsc s_k ; s_1) =1$ it follows that $\pi_w(s_1,\epsilon) = (w s_1 w, -\epsilon) \neq (s_1, \epsilon)$.
%\end{proof}
%
%\begin{remark}
%	For any $t\in T$ we have $\pi_t (t,\epsilon) = (t,-\epsilon)$.
%\end{remark}
%
%We can now proof the theorem
%\begin{proof}
%	Assume $t\in Inv(w)$, that is $l_S(tw)< l_S(w)$. One shows that $\eta(w,t) = -1$, i.e. that $t = t_i$ for some $i\in \{1\dotsc ,k\}$ thus $l_S(tw)< l_S(w)$.
%\end{proof}
%
%\begin{remark}
%	It holds that $\# Inv(w) = l_S(w)$ since if $w= s_1\dotsc s_k$ is a reduced word $Inv(w)$ consists of $k$ reflections. Conversely let $\# Inv(w) =k$ then $w$ has a least  $k$ reflections and in a reduced word $t_i \neq t_j$ for $i\neq j$ implies it has at most $k$ reflections.
%\end{remark}
%
%
%TODO----


\begin{definition}
	Let $(W,S)$ be a Coxeter system and $u,v\in W$ we write
	$u < v$ iff $l(u) < l(v)$ and there exists a sequence of reflections $t_1,\dotsc t_n$ such that
	$u t_1\dotsc t_n =v$. This partial order is called the Bruhat order.
	An arrow $u\rightarrow v$ is a tuple $(u,v)$ with $u<v$ such that there exists a $t\in T$ with $ut=v$
\end{definition}

\begin{conjecture}
	Let $u$, $v$ be such that $l(u)+2 = l(v)$ then 
	$$ \{ w\in W | u\rightarrow w \rightarrow v \} $$ has cardinality either two or zero.
\end{conjecture}


\begin{example}
	Fix the generators $a= (12)$, $b=(13)$ of the group $S_3$ and observe that 
	$(ab)^3=id$.
	The elements of $S_3$ are
	$$
		id, \quad a, \quad b, \quad ab, \quad ba, \quad aba=bab.
	$$
	Observe that $a < ab$ and $a< ba$ since $a(aba) = ba$.
	\begin{center}
	\begin{tikzcd}
		& aba = bab	\\
		ab 		\arrow[ur,"a"]				
		&& ba	\arrow[ul,"b"'] \\
		a 		\arrow[u,"b"]
				\arrow[urr,"aba"']			
		&& b	\arrow[u,"a"']
				\arrow[ull,"aba"] \\
		&id		\arrow[ul,"a"]
				\arrow[ur,"b"']
	\end{tikzcd}
	\end{center}
\end{example}	

Now consider a Nichols algebra $B$ of diagonal type corresponding to type $A_2$ i.e. whose Weyl group is the above group $S_3$.
Let it be generated by $x_1, x_2$ by AngAndr the PBW basis is generated by $x_1, x_{12}, x_2$ and the relations are
$$
	[x_1,x_{12}]= 0, \quad
	[x_1,x_2] = x_{12} \quad
	[x_{12},x_2]= 0.
$$
The following complex is a projective resolution.
\begin{center}
	\begin{tikzcd}
		& & B \arrow[r, "\cdot x_1^2"] \arrow[rdd, "\cdot x_{12}"'] & B \arrow[rd,"\cdot x_2"]\\
		0 \arrow[r] & B\arrow[ur,"\cdot x_2"] \arrow[dr,"\cdot x_1"'] & && B \arrow[r,"\epsilon"] & k \arrow[r] & 0\\
		& & B \arrow[r, "\cdot x_2^2"'] \arrow[ruu, "\cdot x_{12}"'] & B \arrow[ru,"\cdot x_1"']\\
	\end{tikzcd}
\end{center}
\begin{proof}
	First note that it is a free complex therefore projective.
	Exactness can be checked for each square (in the inner). that the last map is injective is clear and that the first map generates the kernel of $\epsilon$ is clear as well.
	
	Idea for the squares:  Multiplication with a root vector corresponds in some way to the cooresponding reflection in the Weyl group.

 	Roughly speaking:  ?
	
\end{proof}


A more involved example

Let $B$ be the Nichols algebra corresponding to $G_2$-type.
That is the root system looks like
\begin{center}
	\begin{tikzpicture}
	\foreach\ang in {60,120,...,360}{
		\draw[->,black,thick] (0,0) -- (\ang:2cm);
	}
	\foreach\ang in {30,90,...,330}{
		\draw[->,black,thick] (0,0) -- (\ang:3cm);
	}
	\node[anchor=south west,scale=0.6] at (2,0) {$\alpha$};
	\node[anchor=south east,scale=0.6] at (150:3cm) {$\beta$};
	\node[anchor=south east,scale=0.6] at (120:2cm) {$\alpha+\beta$};
	\node[anchor=south west,scale=0.6] at (60:2cm) {$2\alpha+\beta$};
	\node[anchor=south west,scale=0.6] at (30:3cm) {$3\alpha+\beta$};
	\node[anchor=south,scale=0.6] at (90:3cm) {$3\alpha+2\beta$};
	\end{tikzpicture}
\end{center}
We fix an convex ordering on the positive roots, i.e. an ordering such that
$x< x+y<y$ whenever $x+y\in \Delta^+$.
Such an  ordering is given for example by
$\alpha < 3 \alpha + \beta < 2\alpha + \beta < 3\alpha +2 \beta  <  \alpha+\beta < \beta$.
Let us compute its Weyl group. We denote its generators by $s_a$ and $s_b$.
We write the reflections along simple roots in a matrix form
$$ 
	s_a:=
	\begin{pmatrix}
		\alpha &
		3 \alpha + \beta & 
		2 \alpha + \beta & 
		3 \alpha + 2 \beta & 
		\alpha + \beta & 
		\beta \\
		-\alpha &
		\beta &
		\alpha + \beta &
		3 \alpha + 2 \beta &
		2 \alpha + \beta &
		3 \alpha + \beta
	\end{pmatrix}
$$
$$ 
	s_b:=
	\begin{pmatrix}
		\alpha &
		3 \alpha + \beta & 
		2 \alpha + \beta & 
		3 \alpha + 2 \beta & 
		\alpha + \beta & 
		\beta \\
		\alpha + \beta &
		3 \alpha+ 2 \beta &
		2 \alpha + \beta &
		3 \alpha +  \beta &
		 \alpha &
		- \beta
	\end{pmatrix}
$$

From this we compute every element of the Weyl group explicitly: 
\begin{center}
	\begin{tikzcd}
		& a \arrow[r, "b"] \arrow[rdd, bend left =10, "aba"] &
		ab  \arrow[r, "a"] \arrow[rdd, bend right =10, "babab"']  &
		aba \arrow[r, "b"] \arrow[rdd, bend left =10, "babab"] &
		abab \arrow[r, "a"] \arrow[rdd, bend right =10, "aba"'] &
		ababa \arrow[rd,"b"]
		\\
		e \arrow[ru, "a"]\arrow[rd, "b"'] & & & & & & ababab = bababa 
		\\
		& b \arrow[r, "a"'] \arrow[ruu, bend left =10, "bab"] & 
		ba \arrow[r, "b"'] \arrow[ruu, bend right =10, "ababa"'] & 
		bab \arrow[r, "a"'] \arrow[ruu, bend left =10, "ababa"]& 
		baba \arrow[r, "b"'] \arrow[ruu, bend right =10, "bab"']&
		babab \arrow[ru,"a"'] 
	\end{tikzcd}
\end{center}


\section{Verma modules for the Drinfeld double}

Let $G$ be a finite abelian group $V$ a Yetter-Drinfeld module of diagonal type over $G$ such that $\Nichols V$ is finite dimensional.

Set
\begin{definition}
	We call $\Lambda := \{ \text{iso classes of simple represeentations of } D(G)\}\cong D(G)$ the space of weights.
	
	A Verma module of weight $\lambda\in \Lambda$ is defined as the (induced) $D(\Nichols V)$ module
	$$
		M(\lambda):= D(\Nichols V)\otimes _{D(G)\# \Nichols V} \mathbb C_\lambda.
	$$
\end{definition}

\begin{theorem}
	Let $M:=M(\lambda)$ be a Verma-module. It is generated by a heighest degree vector, i.e. a vector $v\in M$ such that $\Nichols V \triangleright v =0 $ and $D(\Nichols V)\triangleright v = M$.
\end{theorem}
\begin{proof}
	Consider $v:= 1 \otimes_{D(G)\# \Nichols V} 1\neq 0$ then obviously $v$ generates the Verma module and every term in the ' positive' Nichols algebra vanishes.
\end{proof}

\begin{definition}
	A nice partial order on the space of weights satisfies
	\begin{enumerate}
		\item if $n\in \Nichols V$ and $\lambda \in \Lambda$ then $n\cdot \lambda > \lambda$.
	\end{enumerate}
\end{definition}

\begin{theorem}
	Let $M$ be a $D$ module such that there exists a $w$ in the weight space of the weight $\lambda\in \Lambda$ and $\Nichols V \triangleright w =0$ then there is a unique morphism 
	$M(\lambda)\rightarrow M$ such that the maximal weight vector $v_\lambda \in M(\lambda)$ is mapped to $w$.
\end{theorem}
\begin{proof}
	We claim that the following defines a morphism of $D$ modules:
	\begin{align*}
		f:M(\lambda) \rightarrow M, x \otimes_B 1 \mapsto x \triangleright w 
	\end{align*}
	As $\mathbb C_\lambda $ is generated by $1$ we can write any element $x\otimes_B \xi$  in $M(\lambda)$ wlog as $x \otimes_B 1$.
	Now let $x\otimes_B 1 = y \otimes_B 1$ then there exists a $b\in B$ such that 
	$x b \otimes_B 1 = y \otimes_B b1 = y \otimes_B \chi(b)1 = \chi(b) y\otimes_B 1$
\end{proof}

\begin{theorem}
	Assume 
\end{theorem}

\begin{proof}
	Assume $M\xrightarrow{f}N$ to be an epimorphism of $D$ modules and $\phi: M(\lambda)\rightarrow N$ a homomorphism.
	Let $v\in M(\lambda)$ denote a highest weight vector and write $w:= \phi(v)\in N$.
	Set $u\in f^{-1}(w)$. Write $M'$ for the $M$ submodule generated by $u$.
	
	We need to show that $\Nichols V\triangleright u =0$. Here we want to use that $\lambda$ is dominant. 
	General observation. If $\Nichols V$ does not act trivially then there's a finite sequence 
	$u \rightarrow u_1 \rightarrow u_2 \rightarrow \dotsc u_m$ such that $u_{n-1}$ is obtained from $u_n$ by an action of $\Nichols V$ and $u_n \neq 0$ and  $\Nichols V u_m = 0$ (since $\Nichols V$ is finite dimensional and graded). Now $u_m$ has some weight $\lambda'$ and $\lambda'>\lambda$ which is a contradiction to the dominance of $\lambda$.
	
	Then we obtain by a `universal property' that there exists a morphism from $M(\lambda)$ to $M$
\end{proof}

\begin{theorem}
	To every positive  in the root system associated to $\Nichols V$ we can associate a weight $\lambda$ which is moreover dominant.
\end{theorem}

\begin{theorem}
	Morphisms between Verma modules associated to roots in the root system are determined by the Bruhat order of the Weyl group.
\end{theorem}
\begin{proof}
	Uniqueness of the maps follows from the universal property. 
\end{proof}

\begin{theorem}
	Every square is exact.
\end{theorem}
\begin{proof}
	 Up to scalars there is a unique map $M_w \rightarrow M_w'$ since we have two different factorisations of these maps there are signs such that the square needs to be zero. 
	 
	 Exactness should follow from the injectivity of the morphisms.
\end{proof}


\begin{theorem}
	The BGG resolution exists.
\end{theorem}

\end{document}